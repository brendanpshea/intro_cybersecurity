\documentclass{beamer}
\usetheme{Madrid}
\usecolortheme{whale}
\usepackage{graphicx}
\usepackage{amsmath}
\usepackage{hyperref}

\title{Applying Security Techniques to Computing Resources}
\author{Security Fundamentals Course}
\date{\today}

\begin{document}

\begin{frame}
    \titlepage
\end{frame}

\begin{frame}
    \frametitle{Introduction: Securing Computing Resources in the Modern World}
    
    \begin{itemize}
        \item Computing resources face continuous and evolving threats in today's interconnected environment.
        \item Security must be integrated into every aspect of information technology, not added as an afterthought.
        \item Effective security requires a balanced approach that addresses technical, administrative, and physical controls.
        \item This lecture covers fundamental security techniques applicable across various computing environments.
    \end{itemize}
    
    \begin{alertblock}{Key Challenge}
        Organizations must balance security needs with usability, performance, and cost constraints while addressing an ever-expanding threat landscape.
    \end{alertblock}
\end{frame}

\begin{frame}
    \frametitle{Security Fundamentals: Why Establish Secure Baselines?}
    
    \begin{itemize}
        \item \textbf{Secure baselines} provide a consistent, documented starting point for deploying secure systems.
        \item Baselines ensure that security is applied systematically rather than haphazardly across an organization.
        \item They establish a measurable standard against which compliance can be regularly verified.
        \item Baselines reduce attack surface and minimize common configuration vulnerabilities.
    \end{itemize}
    
    \begin{block}{Benefits of Secure Baselines}
        \begin{itemize}
            \item Predictable security posture
            \item Easier troubleshooting
            \item Streamlined auditing
            \item Simplified compliance
        \end{itemize}
    \end{block}
\end{frame}

\begin{frame}
    \frametitle{Establishing Secure Baselines: Key Principles}
    
    \begin{itemize}
        \item Begin by identifying critical security requirements and compliance standards relevant to your organization.
        \item \textbf{Principle of least privilege} ensures users and systems have only the access necessary to perform their functions.
        \item Document all configuration decisions, including rationale and exceptions to standard practice.
        \item Collaborate across security, operations, and business units to develop realistic, implementable baselines.
    \end{itemize}
    
    \begin{exampleblock}{Example Baseline Components}
        System hardening specifications, approved software lists, required security controls, patch management requirements, and authentication standards.
    \end{exampleblock}
\end{frame}

\begin{frame}
    \frametitle{Deploying Secure Baselines: Implementation Strategies}
    
    \begin{itemize}
        \item Create a staged deployment plan with testing phases to identify potential compatibility issues.
        \item Utilize automation tools to ensure consistent application of baseline configurations.
        \item Establish verification procedures to confirm that systems meet baseline requirements.
        \item Document exceptions with formal risk assessment and management approval.
    \end{itemize}
    
    \begin{table}
        \begin{tabular}{|l|l|}
            \hline
            \textbf{Deployment Method} & \textbf{Best Use Case} \\
            \hline
            Golden Images & New system deployment \\
            Configuration Management & Existing system maintenance \\
            Group Policy & Windows environment control \\
            Scripted Configuration & Cross-platform standardization \\
            \hline
        \end{tabular}
        \caption{Common Baseline Deployment Methods}
    \end{table}
\end{frame}

\begin{frame}
    \frametitle{Maintaining Secure Baselines: Ongoing Requirements}
    
    \begin{itemize}
        \item Regularly review baselines to ensure they address current threats and organizational needs.
        \item \textbf{Configuration drift} occurs when systems gradually deviate from their secure baseline state.
        \item Implement continuous monitoring to detect unauthorized changes or configuration drift.
        \item Establish a formal change management process for baseline updates to maintain security and stability.
    \end{itemize}
    
    \begin{block}{The Baseline Lifecycle}
        Create → Deploy → Monitor → Update → Redeploy → Verify → Monitor
    \end{block}
\end{frame}

\begin{frame}
    \frametitle{Hardening Techniques: Overview and Importance}
    
    \begin{itemize}
        \item \textbf{System hardening} refers to the process of securing a system by reducing its attack surface.
        \item Hardening involves removing unnecessary software, disabling unneeded services, and applying secure configurations.
        \item Different types of systems require specialized hardening techniques based on their function and exposure.
        \item Hardening should be applied in layers as part of a defense-in-depth security strategy.
    \end{itemize}
    
    \begin{alertblock}{Critical Hardening Categories}
        Operating system hardening, network hardening, application hardening, database hardening, and physical hardening all work together to protect systems.
    \end{alertblock}
\end{frame}

\begin{frame}
    \frametitle{Mobile Device Hardening: Common Vulnerabilities \& Solutions}
    
    \begin{itemize}
        \item Mobile devices face unique security challenges due to their portability, connectivity options, and diverse app ecosystems.
        \item \textbf{Mobile hardening} includes enforcing strong authentication, encryption, and remote wipe capabilities.
        \item Limit application installations to trusted sources and implement application vetting procedures.
        \item Configure automatic updates and establish secure connectivity requirements.
    \end{itemize}
    
    \begin{itemize}
        \item[] \textbf{Key Mobile Hardening Controls:}
        \begin{itemize}
            \item Screen locks with biometrics or strong passcodes
            \item Full-device encryption
            \item OS and app update enforcement
            \item Controlled app permissions
        \end{itemize}
    \end{itemize}
\end{frame}

\begin{frame}
    \frametitle{Workstation Hardening: Protecting the User Environment}
    
    \begin{itemize}
        \item Workstations are primary targets for attackers as they are direct user interfaces to organizational resources.
        \item Implement \textbf{application whitelisting} to prevent execution of unauthorized programs on workstations.
        \item Restrict administrative privileges and apply the principle of least privilege for user accounts.
        \item Configure host-based firewalls and enable disk encryption to protect data at rest.
    \end{itemize}
    
    \begin{exampleblock}{Example: Windows 11 Hardening}
        Use AppLocker for application control, BitLocker for disk encryption, Windows Defender for malware protection, and Group Policy to enforce secure configurations.
    \end{exampleblock}
\end{frame}

\begin{frame}
    \frametitle{Network Hardware Security: Hardening Switches \& Routers}
    
    \begin{itemize}
        \item Network devices manage the flow of all organizational data and require specialized hardening techniques.
        \item Disable unnecessary services and protocols on network devices to reduce attack surface.
        \item Implement \textbf{access control lists (ACLs)} to filter traffic based on security policies.
        \item Secure the management interfaces with strong authentication and encrypted connections.
    \end{itemize}
    
    \begin{block}{Essential Switch Hardening Steps}
        \begin{enumerate}
            \item Disable unused ports
            \item Implement port security
            \item Secure spanning tree protocol
            \item Configure VLAN segregation
        \end{enumerate}
    \end{block}
\end{frame}

\begin{frame}
    \frametitle{Cloud Infrastructure Hardening: Securing Virtual Environments}
    
    \begin{itemize}
        \item Cloud environments introduce shared responsibility models where security duties are split between providers and customers.
        \item Apply \textbf{infrastructure as code (IaC)} principles to ensure consistent, secure deployments.
        \item Implement strong identity and access management with multi-factor authentication for cloud resources.
        \item Monitor cloud configurations for drift and unauthorized changes using cloud security posture management tools.
    \end{itemize}
    
    \begin{alertblock}{Cloud Security Misconception}
        Cloud providers secure the infrastructure, but customers remain responsible for securing their data, access controls, and application configurations.
    \end{alertblock}
\end{frame}

\begin{frame}
    \frametitle{Server Hardening: Best Practices and Critical Controls}
    
    \begin{itemize}
        \item Servers host critical applications and data, making them high-value targets requiring robust protection.
        \item Install only necessary components and remove or disable unused services, roles, and features.
        \item Implement \textbf{file integrity monitoring (FIM)} to detect unauthorized modifications to critical system files.
        \item Apply regular patches and updates through a formal testing and deployment process.
    \end{itemize}
    
    \begin{table}
        \begin{tabular}{|l|p{5cm}|}
            \hline
            \textbf{Server Type} & \textbf{Special Hardening Considerations} \\
            \hline
            Web Servers & Web application firewalls, TLS configuration \\
            Database Servers & Query restrictions, data encryption \\
            Email Servers & Content filtering, anti-spoofing \\
            Domain Controllers & Privileged access management \\
            \hline
        \end{tabular}
    \end{table}
\end{frame}

\begin{frame}
    \frametitle{ICS/SCADA Security: Protecting Critical Infrastructure}
    
    \begin{itemize}
        \item \textbf{Industrial Control Systems (ICS)} and \textbf{SCADA} environments control physical processes and critical infrastructure.
        \item These systems often use specialized protocols and legacy components with unique security requirements.
        \item Implement network segmentation to isolate ICS/SCADA networks from corporate IT networks.
        \item Apply security patches cautiously with extensive testing due to potential operational impacts.
    \end{itemize}
    
    \begin{block}{ICS/SCADA Security Challenges}
        Long lifecycles (20+ years), real-time requirements, proprietary protocols, and physical safety implications create unique security considerations.
    \end{block}
\end{frame}

\begin{frame}
    \frametitle{Embedded Systems \& RTOS Hardening: Special Considerations}
    
    \begin{itemize}
        \item \textbf{Embedded systems} are specialized computing devices with dedicated functions built into larger mechanical or electrical systems.
        \item \textbf{Real-Time Operating Systems (RTOS)} have strict timing requirements that security measures must not disrupt.
        \item Resource constraints (memory, processing power) limit the security mechanisms that can be implemented.
        \item Secure the boot process to prevent unauthorized firmware modifications.
    \end{itemize}
    
    \begin{alertblock}{Security vs. Performance Trade-offs}
        Security controls for embedded systems must be carefully balanced against performance requirements, especially in time-critical applications.
    \end{alertblock}
\end{frame}

\begin{frame}
    \frametitle{IoT Device Security: Challenges and Solutions}
    
    \begin{itemize}
        \item \textbf{Internet of Things (IoT)} devices often combine limited computing resources with network connectivity.
        \item Many IoT devices lack basic security features like strong authentication or encryption capabilities.
        \item Default credentials present a major vulnerability—always change manufacturer passwords.
        \item Network segmentation is critical to contain potential compromises of vulnerable IoT devices.
    \end{itemize}
    
    \begin{itemize}
        \item[] \textbf{IoT Security Best Practices:}
        \begin{itemize}
            \item Maintain an inventory of all connected devices
            \item Implement dedicated IoT networks
            \item Disable unnecessary features
            \item Apply firmware updates regularly
        \end{itemize}
    \end{itemize}
\end{frame}

\begin{frame}
    \frametitle{Wireless Device Security: The Foundation of Mobile Computing}
    
    \begin{itemize}
        \item Wireless networks extend connectivity beyond physical boundaries, creating additional attack vectors.
        \item Implement \textbf{defense in depth} with multiple security layers for wireless environments.
        \item Secure both the wireless infrastructure (access points, controllers) and client devices.
        \item Regular wireless scanning helps identify unauthorized access points and potential attacks.
    \end{itemize}
    
    \begin{exampleblock}{Wireless Attack Types}
        Rogue access points, evil twin attacks, deauthentication attacks, and wireless packet sniffing represent common threats to wireless networks.
    \end{exampleblock}
\end{frame}

\begin{frame}
    \frametitle{Site Surveys: Mapping Your Wireless Environment}
    
    \begin{itemize}
        \item \textbf{Site surveys} are methodical assessments of wireless environments to optimize coverage and security.
        \item Physical site surveys identify potential signal obstructions, interference sources, and optimal access point locations.
        \item Predictive site surveys use software modeling to simulate wireless coverage before hardware deployment.
        \item Post-implementation validation surveys confirm that the deployed wireless network meets design requirements.
    \end{itemize}
    
    \begin{block}{Site Survey Components}
        Signal strength measurements, interference detection, channel utilization analysis, and coverage overlap assessment.
    \end{block}
\end{frame}

\begin{frame}
    \frametitle{Heat Maps: Visual Analysis of Wireless Coverage}
    
    \begin{itemize}
        \item \textbf{Heat maps} provide color-coded visualizations of wireless signal strength throughout a physical space.
        \item They help identify coverage gaps, areas of signal overlap, and potential interference zones.
        \item Heat maps support capacity planning by showing high-density usage areas that may require additional access points.
        \item Regular heat map analysis helps detect unauthorized access points and potential security threats.
    \end{itemize}
    
    \begin{figure}
        \centering
        \framebox{\parbox{0.8\textwidth}{
            [This is where a sample heat map image would be displayed, showing different signal strengths represented by color variations]
        }}
        \caption{Example Wireless Heat Map Showing Signal Coverage}
    \end{figure}
\end{frame}

\begin{frame}
    \frametitle{Mobile Device Management (MDM): Controlling the Mobile Ecosystem}
    
    \begin{itemize}
        \item \textbf{Mobile Device Management (MDM)} provides centralized control over mobile devices in an organization.
        \item MDM enables remote configuration, policy enforcement, application management, and security monitoring.
        \item Core MDM capabilities include device enrollment, inventory management, and compliance verification.
        \item Advanced MDM solutions incorporate mobile application management and content management features.
    \end{itemize}
    
    \begin{alertblock}{MDM Security Capabilities}
        Remote lock and wipe, encryption enforcement, application blacklisting/whitelisting, jailbreak/root detection, and certificate management.
    \end{alertblock}
\end{frame}

\begin{frame}
    \frametitle{BYOD vs. COPE vs. CYOD: Selecting the Right Deployment Model}
    
    \begin{itemize}
        \item \textbf{Bring Your Own Device (BYOD)} allows employees to use personal devices for work, reducing hardware costs but increasing security challenges.
        \item \textbf{Corporate-Owned, Personally Enabled (COPE)} provides organization-owned devices with limited personal use allowed.
        \item \textbf{Choose Your Own Device (CYOD)} lets employees select from approved device options that are then purchased by the organization.
        \item Each model represents different balances between user satisfaction, cost, and security control.
    \end{itemize}
    
    \begin{table}
        \begin{tabular}{|l|c|c|c|}
            \hline
            \textbf{Factor} & \textbf{BYOD} & \textbf{COPE} & \textbf{CYOD} \\
            \hline
            Cost to Organization & Low & High & Medium \\
            User Satisfaction & High & Medium & High \\
            Security Control & Low & High & High \\
            Support Complexity & High & Low & Medium \\
            \hline
        \end{tabular}
    \end{table}
\end{frame}

\begin{frame}
    \frametitle{Mobile Connection Methods: Cellular, Wi-Fi \& Bluetooth Security}
    
    \begin{itemize}
        \item Mobile devices typically support multiple connection methods, each with distinct security considerations.
        \item \textbf{Cellular connections} use carrier-managed infrastructure with built-in encryption but vary in security by generation (3G/4G/5G).
        \item \textbf{Wi-Fi connections} offer higher bandwidth but require careful security configuration to prevent eavesdropping.
        \item \textbf{Bluetooth} enables short-range device communications but has historically suffered from numerous security vulnerabilities.
    \end{itemize}
    
    \begin{block}{Securing Mobile Connections}
        Use VPNs for sensitive communications, disable auto-connect features for Wi-Fi and Bluetooth, and implement connection policies through MDM.
    \end{block}
\end{frame}

\begin{frame}
    \frametitle{Wi-Fi Protected Access 3 (WPA3): Features and Implementation}
    
    \begin{itemize}
        \item \textbf{WPA3} is the latest Wi-Fi security protocol that addresses vulnerabilities in previous standards like WPA2.
        \item WPA3 implements \textbf{Simultaneous Authentication of Equals (SAE)} to replace the vulnerable Pre-Shared Key (PSK) method.
        \item Forward secrecy in WPA3 ensures that captured data cannot be decrypted even if the password is compromised later.
        \item WPA3-Enterprise adds 192-bit security mode for organizations requiring higher security levels.
    \end{itemize}
    
    \begin{alertblock}{WPA2 vs. WPA3 Key Improvements}
        Protection against offline dictionary attacks, improved protection for weak passwords, and simplified secure configuration for devices without displays.
    \end{alertblock}
\end{frame}

\begin{frame}
    \frametitle{AAA/Remote Authentication Dial-In User Service (RADIUS): Authentication, Authorization, and Accounting}
    
    \begin{itemize}
        \item \textbf{AAA} framework provides Authentication (identity verification), Authorization (access control), and Accounting (usage tracking).
        \item \textbf{RADIUS} is a network protocol that implements the AAA framework, particularly for network access authentication.
        \item RADIUS centralizes authentication for network devices, wireless networks, and VPN connections.
        \item Integration with directory services like Active Directory enables consistent identity management.
    \end{itemize}
    
    \begin{figure}
        \centering
        \scriptsize
        \begin{tabular}{c}
            User → Access Device (Authenticator) → RADIUS Server → Directory Service \\
            ↑ \\
            Authentication Result Flow \\
        \end{tabular}
        \caption{Simplified RADIUS Authentication Flow}
    \end{figure}
\end{frame}

\begin{frame}
    \frametitle{Cryptographic Protocols: Ensuring Data Privacy in Transit}
    
    \begin{itemize}
        \item \textbf{Cryptographic protocols} establish secure communications channels over potentially insecure networks.
        \item \textbf{Transport Layer Security (TLS)} secures web traffic, email, and many other application protocols.
        \item \textbf{Internet Protocol Security (IPsec)} provides security at the network layer for VPN implementations.
        \item Wireless networks use specific protocols like CCMP (Counter Mode CBC-MAC Protocol) for WPA2/WPA3 encryption.
    \end{itemize}
    
    \begin{exampleblock}{Key Protocol Selection Factors}
        Consider performance requirements, compliance standards, supported algorithms, key management capabilities, and compatibility with existing systems.
    \end{exampleblock}
\end{frame}

\begin{frame}
    \frametitle{Authentication Protocols: Verifying Identity Securely}
    
    \begin{itemize}
        \item \textbf{Authentication protocols} standardize the process of verifying claimed identities across systems.
        \item \textbf{LDAP} (Lightweight Directory Access Protocol) provides directory services for storing and retrieving identity information.
        \item \textbf{Kerberos} uses ticket-based authentication to verify identities without transmitting passwords.
        \item \textbf{SAML} and \textbf{OAuth} enable single sign-on and delegated authorization across different systems.
    \end{itemize}
    
    \begin{block}{Multi-Factor Authentication (MFA)}
        Combines two or more authentication factors: something you know (password), something you have (token), and something you are (biometric).
    \end{block}
\end{frame}

\begin{frame}
    \frametitle{Application Security: The First Line of Defense}
    
    \begin{itemize}
        \item Applications are primary attack targets because they often have direct access to sensitive data.
        \item \textbf{Application security} involves building security into applications from the design phase through deployment and maintenance.
        \item Secure software development lifecycle (SSDLC) integrates security activities throughout development.
        \item Defense-in-depth for applications includes input validation, authentication, authorization, and error handling.
    \end{itemize}
    
    \begin{alertblock}{Common Application Vulnerabilities}
        Injection attacks, broken authentication, sensitive data exposure, XML external entities, broken access control, security misconfigurations, and cross-site scripting (XSS).
    \end{alertblock}
\end{frame}

\begin{frame}
    \frametitle{Input Validation: Preventing Injection Attacks}
    
    \begin{itemize}
        \item \textbf{Input validation} ensures that application data meets expected formats and constraints before processing.
        \item Server-side validation is critical—client-side validation can be easily bypassed by attackers.
        \item Implement both "allow list" (explicitly approve valid input) and "deny list" (reject known bad input) approaches.
        \item Proper input validation helps prevent SQL injection, command injection, and cross-site scripting attacks.
    \end{itemize}
    
    \begin{exampleblock}{Input Validation Techniques}
        Data type checking, range validation, format validation, length restrictions, and parameterized queries for database operations.
    \end{exampleblock}
\end{frame}

\begin{frame}
    \frametitle{Secure Cookies: Protecting Session Data}
    
    \begin{itemize}
        \item \textbf{Cookies} store session information and user preferences in web applications.
        \item \textbf{Secure cookies} are transmitted only over encrypted HTTPS connections to prevent interception.
        \item The HttpOnly flag prevents JavaScript access to cookies, protecting against cross-site scripting attacks.
        \item SameSite attribute restricts cookie transmission to same-site requests, preventing cross-site request forgery.
    \end{itemize}
    
    \begin{table}
        \begin{tabular}{|l|p{5cm}|}
            \hline
            \textbf{Cookie Attribute} & \textbf{Security Function} \\
            \hline
            Secure & Transmission over HTTPS only \\
            HttpOnly & Blocks JavaScript access \\
            SameSite & Controls cross-origin behavior \\
            Expires/Max-Age & Limits cookie lifetime \\
            \hline
        \end{tabular}
    \end{table}
\end{frame}

\begin{frame}
    \frametitle{Static Code Analysis: Finding Vulnerabilities Before Deployment}
    
    \begin{itemize}
        \item \textbf{Static code analysis} examines source code without executing it to identify potential security vulnerabilities.
        \item Static analysis tools can detect issues like buffer overflows, SQL injection, and insecure cryptographic implementations.
        \item Analysis should be integrated into the development pipeline for early detection and remediation.
        \item False positives require human review to determine actual security impact and appropriate response.
    \end{itemize}
    
    \begin{block}{Static Analysis Benefits}
        Scales to large codebases, finds vulnerabilities early in development, enforces coding standards, and provides consistent application of security rules.
    \end{block}
\end{frame}

\begin{frame}
    \frametitle{Code Signing: Ensuring Software Integrity}
    
    \begin{itemize}
        \item \textbf{Code signing} uses digital signatures to verify the authenticity and integrity of software.
        \item Signed applications provide assurance that code has not been modified since it was signed by the developer.
        \item Code signing relies on public key infrastructure (PKI) and trusted certificate authorities.
        \item Operating systems and browsers use code signatures to verify application trustworthiness.
    \end{itemize}
    
    \begin{alertblock}{Code Signing Process}
        \begin{enumerate}
            \item Developer obtains a code signing certificate from a trusted authority
            \item Developer creates a cryptographic hash of the software
            \item Hash is encrypted with the developer's private key
            \item Signature is verified using the developer's public key
        \end{enumerate}
    \end{alertblock}
\end{frame}

\begin{frame}
    \frametitle{Sandboxing: Isolating Applications for Security}
    
    \begin{itemize}
        \item \textbf{Sandboxing} creates isolated environments where applications can run with limited access to system resources.
        \item Sandboxes contain potential damage from malicious or vulnerable applications by restricting their capabilities.
        \item Modern web browsers sandbox each tab to prevent malicious websites from affecting the system or other tabs.
        \item Containerization technologies like Docker implement sandbox-like isolation for applications.
    \end{itemize}
    
    \begin{exampleblock}{Sandbox Implementation Methods}
        Virtual machines, containers, application virtualization, browser sandboxes, and operating system security features like seccomp on Linux.
    \end{exampleblock}
\end{frame}

\begin{frame}
    \frametitle{Conclusion: Building a Comprehensive Security Strategy}
    
    \begin{itemize}
        \item Security must be approached holistically, considering all computing resources and their interconnections.
        \item Defense in depth implements multiple security layers to protect against various threats and attack vectors.
        \item Balance security requirements with operational needs, user experience, and resource constraints.
        \item Effective security requires ongoing vigilance: regularly review, test, and update security controls as threats evolve.
    \end{itemize}
    
    \begin{alertblock}{Key Takeaway}
        Security is not a product but a process—it requires continuous improvement, adaptation to new threats, and integration into all aspects of computing resource management.
    \end{alertblock}
\end{frame}


\end{document}