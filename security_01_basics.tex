\documentclass{beamer}

% Basic packages
\usepackage{tikz}
\usepackage{graphicx}
\usepackage{amsmath}

% Theme setup
\usetheme{Madrid}
\usecolortheme{default}

% Footer setup
\setbeamertemplate{footline}[frame number]
\setbeamertemplate{navigation symbols}{}

% Title information
\title{Introduction to Security Controls}
\subtitle{Understanding the Foundations of Cybersecurity}
\author{Brendan Shea, PhD}
\date{\today}

\begin{document}

\begin{frame}
    \titlepage
\end{frame}

\begin{frame}
    \frametitle{Lecture Overview: Security Controls}
    
    \begin{itemize}
        \item In this lecture, we will explore the fundamental concepts of \textbf{security controls} in cybersecurity.
        
        \item We will examine four major \textbf{categories} of controls: Technical, Managerial, Operational, and Physical.
        
        \item You will learn about six different \textbf{types} of controls: Preventive, Deterrent, Detective, Corrective, Compensating, and Directive.
        
        \item We will analyze real-world examples of how these controls work together to create effective security systems.
    \end{itemize}
    
    \begin{block}{Learning Objectives}
        By the end of this lecture, you will understand how to identify, classify, and evaluate different security controls.
    \end{block}
\end{frame}

\begin{frame}
    \frametitle{What is Information Security?}
    
    \begin{itemize}
        \item \textbf{Information Security} is the practice of protecting information by mitigating risks and threats.
        
        \item The goal is to protect the \textbf{CIA triad}:
        \begin{itemize}
            \item \textbf{Confidentiality}: Keeping information private
            \item \textbf{Integrity}: Ensuring information hasn't been altered
            \item \textbf{Availability}: Making information accessible when needed
        \end{itemize}
        
        \item Information can be digital, physical, or knowledge-based.
    \end{itemize}
    
    \begin{alertblock}{Core Concept}
        Information security protects all forms of information throughout their entire lifecycle.
    \end{alertblock}
\end{frame}

\begin{frame}
    \frametitle{Understanding Security Terms}
    
    \begin{itemize}
        \item A \textbf{threat} is any potential danger that could harm an asset or organization.
        
        \item A \textbf{vulnerability} is a weakness that could be exploited by a threat.
        
        \item An \textbf{exploit} is a specific way to take advantage of a vulnerability.
        
        \item A \textbf{risk} is the potential for loss or damage when a threat exploits a vulnerability.
    \end{itemize}
    
    \begin{exampleblock}{Example Scenario}
        \begin{itemize}
            \item Threat: A hacker
            \item Vulnerability: Weak password
            \item Exploit: Password guessing program
            \item Risk: Unauthorized account access
        \end{itemize}
    \end{exampleblock}
\end{frame}

\begin{frame}
    \frametitle{Types of Security Risks}
    
    \begin{table}
        \begin{tabular}{|l|l|}
            \hline
            \textbf{Risk Category} & \textbf{Examples} \\
            \hline
            Physical & Fire, theft, natural disasters \\
            Technical & Malware, hacking, system failures \\
            Human & User errors, social engineering \\
            Organizational & Process failures, policy gaps \\
            \hline
        \end{tabular}
    \end{table}
    
    \begin{block}{Understanding Impact}
        Risks can affect:
        \begin{itemize}
            \item Data confidentiality
            \item System integrity
            \item Service availability
            \item Organizational reputation
        \end{itemize}
    \end{block}
\end{frame}


\begin{frame}
    \frametitle{What is a Security Control?}
    
    \begin{itemize}
        \item A \textbf{security control} is any measure designed to protect our systems, data, and resources from threats.
        
        \item Security controls work like the different security features in your home, such as locks, alarms, and security cameras.
        
        \item These measures can be physical objects, technical solutions, or rules and procedures that people follow.
        
        \item The goal of security controls is to \textbf{reduce risk} by protecting against threats, detecting problems, and helping us respond to security incidents.
    \end{itemize}
    
    \begin{alertblock}{Key Point}
        Security controls are the building blocks of a strong security program!
    \end{alertblock}
\end{frame}

\begin{frame}
    \frametitle{Why Do We Need Security Controls?}
    
    \begin{itemize}
        \item Organizations need to protect valuable \textbf{assets}, which include data, systems, and physical resources from unauthorized access or damage.
        
        \item Security controls help manage and reduce \textbf{risks}, which are potential threats that could harm our systems or expose sensitive information.
        
        \item Many organizations must follow specific \textbf{compliance requirements} that mandate certain security measures to protect user privacy and data.
        
        \item Without proper security controls, organizations are vulnerable to various attacks that could result in financial losses, reputation damage, or legal consequences.
    \end{itemize}
\end{frame}

\begin{frame}
    \frametitle{Overview of Security Control Categories}
    
    \begin{itemize}
        \item Security controls can be divided into four main \textbf{categories}, each serving different aspects of an organization's security needs.
        
        \item \textbf{Technical controls} use technology to protect systems and data, such as firewalls and encryption.
        
        \item \textbf{Managerial controls} focus on security decisions and oversight, including policies and risk assessments.
        
        \item \textbf{Operational controls} and \textbf{physical controls} involve day-to-day procedures and tangible security measures.
    \end{itemize}
    
    \begin{exampleblock}{Real World Example}
        A school's security system uses all four categories working together:
        \begin{itemize}
            \item Technical: Computer passwords
            \item Managerial: Security policies
            \item Operational: Security training
            \item Physical: Door locks
        \end{itemize}
    \end{exampleblock}
\end{frame}

\begin{frame}
    \frametitle{Technical Controls: Introduction}
    
    \begin{itemize}
        \item \textbf{Technical controls} are security measures that are implemented and executed by computer systems and software.
        
        \item These controls form the technological foundation of modern cybersecurity practices.
        
        \item Technical controls operate with minimal human intervention once properly configured.
        
        \item They provide consistent and automated protection against many common security threats.
    \end{itemize}
    
    \begin{block}{Key Characteristics}
        Automated, technology-based, and system-enforced protections
    \end{block}
\end{frame}

\begin{frame}
    \frametitle{Technical Controls: Common Examples}
    
    \begin{itemize}
        \item \textbf{Authentication systems} verify user identities through passwords, biometrics, or security tokens.
        
        \item \textbf{Encryption} protects data by converting it into a format that can only be read with the correct key.
        
        \item \textbf{Firewalls} monitor and control incoming and outgoing network traffic based on security rules.
        
        \item \textbf{Antivirus software} detects, prevents, and removes malicious software from computer systems.
    \end{itemize}
    
    \begin{alertblock}{Critical Point}
        Technical controls must be regularly updated to remain effective against new threats!
    \end{alertblock}
\end{frame}

\begin{frame}
    \frametitle{Technical Controls: Implementation}
    
    \begin{itemize}
        \item Technical controls should be implemented in \textbf{layers} to provide defense in depth against various threats.
        
        \item Each technical control must be properly \textbf{configured} to match the organization's security requirements.
        
        \item Regular \textbf{maintenance} and updates are essential to ensure controls remain effective over time.
        
        \item Organizations should maintain \textbf{documentation} of all technical controls and their configurations.
    \end{itemize}
    
    \begin{exampleblock}{Implementation Example}
        A laptop's security might include:
        \begin{itemize}
            \item Login password
            \item Disk encryption
            \item Firewall
            \item Antivirus
        \end{itemize}
    \end{exampleblock}
\end{frame}

\begin{frame}
    \frametitle{Technical Controls: Advantages and Limitations}
    
    \begin{columns}[t]
        \column{0.5\textwidth}
        \textbf{Advantages}
        \begin{itemize}
            \item Consistent operation
            \item Automated responses
            \item Scalable protection
            \item Measurable effectiveness
        \end{itemize}
        
        \column{0.5\textwidth}
        \textbf{Limitations}
        \begin{itemize}
            \item Initial setup costs
            \item Regular updates needed
            \item Technical expertise required
            \item Can be circumvented
        \end{itemize}
    \end{columns}
    
    \begin{block}{Balance}
        Technical controls must be balanced with other control types for effective security.
    \end{block}
\end{frame}

\begin{frame}
    \frametitle{Managerial Controls: Introduction}
    
    \begin{itemize}
        \item \textbf{Managerial controls} are administrative measures that guide the implementation of security practices.
        
        \item These controls focus on managing risks and making decisions about security strategies.
        
        \item Managerial controls establish the framework for all other security controls within an organization.
        
        \item They require active involvement from leadership to ensure effective implementation.
    \end{itemize}
    
    \begin{block}{Essential Role}
        Managerial controls provide direction and oversight for the entire security program.
    \end{block}
\end{frame}

\begin{frame}
    \frametitle{Managerial Controls: Key Components}
    
    \begin{itemize}
        \item \textbf{Security policies} establish the rules and guidelines that govern how an organization protects its assets.
        
        \item \textbf{Risk assessments} help identify potential threats and vulnerabilities to organizational resources.
        
        \item \textbf{Compliance programs} ensure the organization meets all relevant legal and regulatory requirements.
        
        \item \textbf{Resource allocation} determines how to distribute security resources effectively.
    \end{itemize}
    
    \begin{tikzpicture}[remember picture,overlay]
        \node[anchor=south east] at (current page.south east) {
            \begin{tabular}{|l|}
                \hline
                Policy Development \\
                Risk Management \\
                Resource Planning \\
                \hline
            \end{tabular}
        };
    \end{tikzpicture}
\end{frame}

\begin{frame}
    \frametitle{Managerial Controls: Documentation}
    
    \begin{itemize}
        \item Every organization should maintain comprehensive \textbf{security documentation} that outlines all policies and procedures.
        
        \item Documentation must be regularly \textbf{reviewed and updated} to reflect changes in the threat landscape.
        
        \item Security policies should clearly define \textbf{roles and responsibilities} for all members of the organization.
        
        \item Written procedures must provide clear guidance for implementing security measures.
    \end{itemize}
    
    \begin{alertblock}{Documentation Requirements}
        All security policies must be:
        \begin{itemize}
            \item Clear and understandable
            \item Regularly updated
            \item Easily accessible
            \item Formally approved
        \end{itemize}
    \end{alertblock}
\end{frame}

\begin{frame}
    \frametitle{Managerial Controls: Decision Making}
    
    \begin{itemize}
        \item Effective security management requires balancing \textbf{security needs} with business operations and resources.
        
        \item Managers must evaluate the \textbf{cost-effectiveness} of different security measures before implementation.
        
        \item Security decisions should be based on thorough \textbf{risk analysis} rather than reactive responses to incidents.
        
        \item Organizations need clear procedures for \textbf{escalating} security issues to appropriate decision-makers.
    \end{itemize}
    
    \begin{exampleblock}{Decision Framework}
        When evaluating new security measures, consider:
        \begin{itemize}
            \item Risk level
            \item Implementation cost
            \item Operational impact
            \item Resource requirements
        \end{itemize}
    \end{exampleblock}
\end{frame}

\begin{frame}
    \frametitle{Operational Controls: Introduction}
    
    \begin{itemize}
        \item \textbf{Operational controls} are the security procedures and tasks performed by people rather than automated systems.
        
        \item These controls focus on day-to-day activities that maintain and protect organizational security.
        
        \item Operational controls require consistent human execution and oversight to be effective.
        
        \item They bridge the gap between managerial policies and technical implementations.
    \end{itemize}
    
    \begin{block}{Key Characteristic}
        Operational controls depend on people following established procedures correctly and consistently.
    \end{block}
\end{frame}

\begin{frame}
    \frametitle{Operational Controls: Security Awareness}
    
    \begin{itemize}
        \item \textbf{Security awareness training} educates employees about their role in maintaining organizational security.
        
        \item Regular training sessions ensure staff understand current threats and appropriate security responses.
        
        \item Employees must learn to recognize and report potential \textbf{security incidents} promptly.
        
        \item Training programs should include practical exercises and real-world examples.
    \end{itemize}
    
    \begin{exampleblock}{Training Topics}
        Essential security awareness areas include:
        \begin{itemize}
            \item Password management
            \item Email security
            \item Social engineering
            \item Incident reporting
        \end{itemize}
    \end{exampleblock}
\end{frame}

\begin{frame}
    \frametitle{Operational Controls: Daily Procedures}
    
    \begin{columns}[t]
        \column{0.5\textwidth}
        \textbf{Regular Tasks}
        \begin{itemize}
            \item System monitoring
            \item Backup verification
            \item Log reviews
            \item Security updates
        \end{itemize}
        
        \column{0.5\textwidth}
        \textbf{Periodic Tasks}
        \begin{itemize}
            \item Security audits
            \item Access reviews
            \item Policy updates
            \item Disaster drills
        \end{itemize}
    \end{columns}
    
    \begin{alertblock}{Critical Point}
        Consistent execution of operational procedures is essential for maintaining security.
    \end{alertblock}
\end{frame}

\begin{frame}
    \frametitle{Operational Controls: Incident Response}
    
    \begin{itemize}
        \item Organizations must maintain detailed \textbf{incident response procedures} for handling security breaches.
        
        \item Staff should be trained to recognize and properly report potential security incidents.
        
        \item \textbf{Response teams} need clear procedures for investigating and containing security threats.
        
        \item Regular practice drills help ensure effective response during actual security incidents.
    \end{itemize}
    
    \begin{block}{Incident Response Steps}
        \begin{enumerate}
            \item Identification
            \item Containment
            \item Eradication
            \item Recovery
        \end{enumerate}
    \end{block}
\end{frame}

\begin{frame}
    \frametitle{Physical Controls: Introduction}
    
    \begin{itemize}
        \item \textbf{Physical controls} are tangible security measures that protect facilities, equipment, and resources.
        
        \item These controls create barriers between protected assets and potential threats.
        
        \item Physical controls often work in conjunction with technical and operational controls.
        
        \item They form the first line of defense against unauthorized physical access.
    \end{itemize}
    
    \begin{alertblock}{Remember}
        The strongest technical controls can be defeated by weak physical security!
    \end{alertblock}
\end{frame}

\begin{frame}
    \frametitle{Physical Controls: Access Control}
    
    \begin{itemize}
        \item \textbf{Access control systems} manage and monitor entry to protected areas within a facility.
        
        \item Organizations must establish clear procedures for issuing and revoking physical access credentials.
        
        \item Different areas may require different levels of \textbf{access restriction} based on security needs.
        
        \item Physical access controls should maintain detailed logs of all entry and exit activities.
    \end{itemize}
    
    \begin{exampleblock}{Common Access Controls}
        \begin{itemize}
            \item ID badges
            \item Key cards
            \item Biometric scanners
            \item Security guards
        \end{itemize}
    \end{exampleblock}
\end{frame}

\begin{frame}
    \frametitle{Physical Controls: Environmental Protection}
    
    \begin{itemize}
        \item Physical controls must protect against both human threats and \textbf{environmental hazards}.
        
        \item Critical systems require protection from fire, water damage, and power fluctuations.
        
        \item Environmental monitoring systems should track temperature, humidity, and other relevant conditions.
        
        \item Backup power systems must maintain essential security controls during power outages.
    \end{itemize}
    
    \begin{block}{Environmental Systems}
        \begin{tabular}{ll}
            Fire Suppression & Temperature Control \\
            Water Detection & Humidity Monitoring \\
            Power Backup & Emergency Lighting \\
        \end{tabular}
    \end{block}
\end{frame}

\begin{frame}
    \frametitle{Physical Controls: Security Zones}
    
    \begin{itemize}
        \item Organizations should implement \textbf{layered security zones} with increasing protection levels.
        
        \item Each security zone must have clearly defined boundaries and access requirements.
        
        \item Sensitive areas require multiple physical controls working together for adequate protection.
        
        \item Regular security assessments should verify the effectiveness of zone protections.
    \end{itemize}
    
    \begin{tikzpicture}[remember picture,overlay]
        \node[anchor=south] at (current page.south) {
            \begin{tikzpicture}
                \draw[fill=red!10] (0,0) circle (1cm);
                \draw[fill=yellow!10] (0,0) circle (.75cm);
                \draw[fill=green!10] (0,0) circle (.5cm);
                \node at (0,0) {Core};
                \node at (0,.5) {Restricted};
                \node at (0,1) {Controlled};
            \end{tikzpicture}
        };
    \end{tikzpicture}
\end{frame}

\begin{frame}
    \frametitle{Types of Security Controls}
    
    \begin{itemize}
        \item Security controls can be classified into six \textbf{functional types} based on their purpose.
        
        \item Each type of control serves a specific role in the overall security strategy.
        
        \item Organizations typically need multiple types of controls working together.
        
        \item The effectiveness of controls depends on choosing the right type for each security need.
    \end{itemize}
    
    \begin{block}{Control Types}
        \begin{enumerate}
            \item Preventive
            \item Deterrent
            \item Detective
            \item Corrective
            \item Compensating
            \item Directive
        \end{enumerate}
    \end{block}
\end{frame}

\begin{frame}
    \frametitle{Preventive Controls: Introduction}
    
    \begin{itemize}
        \item \textbf{Preventive controls} are designed to stop security incidents before they occur.
        
        \item These controls create barriers that block unauthorized actions and access attempts.
        
        \item Preventive controls are the first line of defense in a security strategy.
        
        \item They work by eliminating vulnerabilities or blocking threat vectors.
    \end{itemize}
    
    \begin{alertblock}{Key Principle}
        It is generally more effective to prevent security incidents than to detect and respond to them after they occur.
    \end{alertblock}
\end{frame}

\begin{frame}
    \frametitle{Preventive Controls: Implementation}
    
    \begin{columns}[t]
        \column{0.5\textwidth}
        \textbf{Technical Prevention}
        \begin{itemize}
            \item Access controls
            \item Input validation
            \item Encryption
            \item Firewalls
        \end{itemize}
        
        \column{0.5\textwidth}
        \textbf{Physical Prevention}
        \begin{itemize}
            \item Door locks
            \item Security gates
            \item Cable locks
            \item Security cameras
        \end{itemize}
    \end{columns}
    
    \begin{exampleblock}{Administrative Prevention}
        \begin{itemize}
            \item Security policies
            \item Access procedures
            \item Security training
        \end{itemize}
    \end{exampleblock}
\end{frame}

\begin{frame}
    \frametitle{Preventive Controls: Effectiveness}
    
    \begin{itemize}
        \item The effectiveness of preventive controls depends on proper \textbf{configuration} and maintenance.
        
        \item Organizations must regularly test and validate their preventive controls.
        
        \item Even strong preventive controls can be circumvented if not properly supported by other control types.
        
        \item Cost-benefit analysis should guide investments in preventive controls.
    \end{itemize}
    
    \begin{block}{Measuring Effectiveness}
        \begin{itemize}
            \item Number of blocked attempts
            \item Reduction in incidents
            \item System uptime
            \item Compliance rates
        \end{itemize}
    \end{block}
\end{frame}

\begin{frame}
    \frametitle{Deterrent Controls: Introduction}
    
    \begin{itemize}
        \item \textbf{Deterrent controls} are designed to discourage potential attackers from attempting security violations.
        
        \item These controls work by making the target appear more difficult or risky to attack.
        
        \item Deterrent controls often have both psychological and practical effects on potential threats.
        
        \item They complement preventive controls by reducing the likelihood of attack attempts.
    \end{itemize}
    
    \begin{alertblock}{Key Principle}
        Effective deterrents make potential attackers decide that the risk isn't worth the potential reward.
    \end{alertblock}
\end{frame}

\begin{frame}
    \frametitle{Deterrent Controls: Implementation}
    
    \begin{itemize}
        \item Deterrent controls must be visible enough to influence potential attacker behavior.
        
        \item Organizations should implement deterrents across physical, technical, and administrative domains.
        
        \item The strength of deterrent controls often depends on the perceived consequences of violation.
        
        \item Regular assessment ensures deterrents remain credible and effective.
    \end{itemize}
    
    \begin{table}
        \begin{tabular}{|l|l|l|}
            \hline
            \textbf{Technical} & \textbf{Physical} & \textbf{Administrative} \\
            \hline
            Warning banners & Warning signs & Security policies \\
            Login monitors & Visible cameras & Legal notices \\
            Access logs & Security lighting & Ethics training \\
            Failed attempt limits & Guard patrols & Disciplinary procedures \\
            \hline
        \end{tabular}
    \end{table}
\end{frame}

\begin{frame}
    \frametitle{Deterrent Controls: Psychological Aspects}
    
    \begin{itemize}
        \item \textbf{Psychological deterrence} works by affecting the risk-reward calculations of potential attackers.
        
        \item Clear communication of security measures increases their deterrent value.
        
        \item The visibility of security controls often matters more than their actual strength.
        
        \item Organizations must maintain the credibility of their deterrent measures.
    \end{itemize}
    
    \begin{exampleblock}{Effective Deterrence Examples}
        \begin{itemize}
            \item Prominent security cameras
            \item Published security policies
            \item Visible security personnel
            \item Clear consequence statements
        \end{itemize}
    \end{exampleblock}
\end{frame}

\begin{frame}
    \frametitle{Deterrent Controls: Limitations}
    
    \begin{itemize}
        \item Deterrent controls do not physically prevent security violations from occurring.
        
        \item The effectiveness of deterrents varies based on the attacker's motivation and risk tolerance.
        
        \item Over-reliance on deterrents can create a false sense of security.
        
        \item Organizations need to balance deterrence with actual protective measures.
    \end{itemize}
    
    \begin{block}{Common Limitations}
        \begin{itemize}
            \item May not affect determined attackers
            \item Requires perception of credible consequences
            \item Effectiveness hard to measure
            \item Cannot stand alone
        \end{itemize}
    \end{block}
\end{frame}

\begin{frame}
    \frametitle{Detective Controls: Introduction}
    
    \begin{itemize}
        \item \textbf{Detective controls} are designed to identify and record security violations or attempts.
        
        \item These controls work by monitoring systems, networks, and physical spaces for suspicious activity.
        
        \item Detective controls are essential for identifying when preventive controls have failed.
        
        \item They provide valuable information for incident response and security improvement.
    \end{itemize}
    
    \begin{alertblock}{Key Principle}
        You can't respond to security incidents if you don't know they're happening!
    \end{alertblock}
\end{frame}

\begin{frame}
    \frametitle{Detective Controls: Types and Examples}
    
    \begin{itemize}
        \item Detective controls must operate continuously to maintain security awareness.
        
        \item Different types of detective controls monitor different aspects of security.
        
        \item Most detective controls generate logs or alerts for security analysis.
        
        \item Real-time detection allows for faster incident response.
    \end{itemize}
    
    \begin{table}
        \begin{tabular}{|l|l|}
            \hline
            \textbf{Control Type} & \textbf{Examples} \\
            \hline
            System Monitoring & Log files, Audit trails \\
            Network Monitoring & IDS, Traffic analysis \\
            Physical Monitoring & Motion sensors, Cameras \\
            Access Monitoring & Login tracking, Badge readers \\
            \hline
        \end{tabular}
    \end{table}
\end{frame}

\begin{frame}
    \frametitle{Detective Controls: Monitoring Process}
    
    \begin{itemize}
        \item \textbf{Security monitoring} requires a systematic approach to data collection and analysis.
        
        \item Organizations must establish baselines to identify abnormal activity effectively.
        
        \item Detective controls should generate appropriate alerts for different security events.
        
        \item Regular review of detection data helps identify security trends and patterns.
    \end{itemize}
    
    \begin{block}{Monitoring Steps}
        \begin{enumerate}
            \item Data Collection
            \item Analysis
            \item Alert Generation
            \item Investigation
            \item Response
        \end{enumerate}
    \end{block}
\end{frame}

\begin{frame}
    \frametitle{Detective Controls: Effective Implementation}
    
    \begin{itemize}
        \item Organizations must carefully configure detective controls to minimize false alarms.
        
        \item \textbf{Alert thresholds} should balance security awareness with operational efficiency.
        
        \item Detective controls require regular maintenance and tuning to remain effective.
        
        \item Staff must be trained to properly interpret and respond to detection alerts.
    \end{itemize}
    
    \begin{exampleblock}{Configuration Considerations}
        \begin{itemize}
            \item Logging level settings
            \item Alert sensitivity
            \item Storage requirements
            \item Response procedures
        \end{itemize}
    \end{exampleblock}
\end{frame}

\begin{frame}
    \frametitle{Corrective Controls: Introduction}
    
    \begin{itemize}
        \item \textbf{Corrective controls} are measures designed to fix problems after a security incident occurs.
        
        \item These controls help restore systems and data to their normal operational state.
        
        \item Corrective controls often work in conjunction with detective controls.
        
        \item They are essential for maintaining business continuity after security breaches.
    \end{itemize}
    
    \begin{alertblock}{Key Principle}
        Even with strong prevention, organizations must be prepared to correct security incidents quickly.
    \end{alertblock}
\end{frame}

\begin{frame}
    \frametitle{Corrective Controls: Common Types}
    
    \begin{itemize}
        \item Organizations need different types of corrective controls for various security scenarios.
        
        \item Each type of corrective control addresses specific aspects of incident recovery.
        
        \item The effectiveness of correction often depends on how quickly controls can be activated.
        
        \item Regular testing helps ensure corrective controls will work when needed.
    \end{itemize}
    
    \begin{table}
        \begin{tabular}{|l|l|}
            \hline
            \textbf{Control Type} & \textbf{Purpose} \\
            \hline
            Backup Systems & Restore lost or corrupted data \\
            Antivirus Tools & Remove malware infections \\
            Patch Management & Fix security vulnerabilities \\
            System Recovery & Restore system functionality \\
            \hline
        \end{tabular}
    \end{table}
\end{frame}

\begin{frame}
    \frametitle{Corrective Controls: Implementation Steps}
    
    \begin{itemize}
        \item Organizations must develop clear procedures for implementing corrective controls.
        
        \item \textbf{Recovery priorities} should be established before incidents occur.
        
        \item Staff need proper training to execute corrective measures effectively.
        
        \item Regular testing and updates ensure corrective controls remain viable.
    \end{itemize}
    
    \begin{block}{Recovery Process}
        \begin{enumerate}
            \item Incident Assessment
            \item Control Selection
            \item Implementation
            \item Verification
            \item Documentation
        \end{enumerate}
    \end{block}
\end{frame}

\begin{frame}
    \frametitle{Corrective Controls: Success Factors}
    
    \begin{itemize}
        \item The success of corrective controls depends on proper preparation and quick response.
        
        \item Organizations must maintain updated documentation of all corrective procedures.
        
        \item Regular testing helps identify and address potential recovery issues.
        
        \item Staff must understand their roles in the correction process.
    \end{itemize}
    
    \begin{exampleblock}{Critical Success Factors}
        \begin{itemize}
            \item Current recovery plans
            \item Tested procedures
            \item Available resources
            \item Trained personnel
            \item Clear responsibilities
        \end{itemize}
    \end{exampleblock}
\end{frame}

\begin{frame}
    \frametitle{Compensating Controls: Introduction}
    
    \begin{itemize}
        \item \textbf{Compensating controls} are alternative security measures used when primary controls are not feasible.
        
        \item These controls provide similar levels of protection through different means.
        
        \item Organizations implement compensating controls due to technical, operational, or cost limitations.
        
        \item The effectiveness must be equivalent to or greater than the original control.
    \end{itemize}
    
    \begin{alertblock}{Key Principle}
        Compensating controls must provide protection that is as strong as the original control they replace.
    \end{alertblock}
\end{frame}

\begin{frame}
    \frametitle{Compensating Controls: Use Cases}
    
    \begin{itemize}
        \item Organizations need compensating controls when primary controls cannot be implemented.
        
        \item Each use case requires careful evaluation of security equivalence.
        
        \item The choice of compensating controls must be justified and documented.
        
        \item Regular assessment ensures continued effectiveness of alternative measures.
    \end{itemize}
    
    \begin{table}
        \begin{tabular}{|l|l|}
            \hline
            \textbf{Primary Control} & \textbf{Compensating Control} \\
            \hline
            Biometric access & Multi-factor authentication \\
            Full disk encryption & File-level encryption \\
            Network segmentation & Enhanced monitoring \\
            Physical security & Video surveillance \\
            \hline
        \end{tabular}
    \end{table}
\end{frame}

\begin{frame}
    \frametitle{Compensating Controls: Implementation}
    
    \begin{itemize}
        \item Implementation of compensating controls requires careful planning and documentation.
        
        \item Organizations must demonstrate that alternative controls meet security requirements.
        
        \item \textbf{Risk assessment} is essential when evaluating compensating controls.
        
        \item Regular reviews ensure compensating controls remain appropriate and effective.
    \end{itemize}
    
    \begin{block}{Implementation Steps}
        \begin{enumerate}
            \item Identify limitations
            \item Assess alternatives
            \item Document justification
            \item Implement control
            \item Verify effectiveness
        \end{enumerate}
    \end{block}
\end{frame}

\begin{frame}
    \frametitle{Compensating Controls: Evaluation Criteria}
    
    \begin{itemize}
        \item Organizations must evaluate compensating controls against specific criteria.
        
        \item The evaluation process should consider both security effectiveness and operational impact.
        
        \item Documentation must demonstrate how compensating controls meet security objectives.
        
        \item Regular assessment helps identify when compensating controls need adjustment.
    \end{itemize}
    
    \begin{exampleblock}{Evaluation Factors}
        \begin{itemize}
            \item Security strength
            \item Implementation cost
            \item Operational impact
            \item Maintenance requirements
            \item Compliance alignment
        \end{itemize}
    \end{exampleblock}
\end{frame}
\begin{frame}
    \frametitle{Directive Controls: Introduction}
    
    \begin{itemize}
        \item \textbf{Directive controls} are measures that guide and direct people's behavior regarding security.
        
        \item These controls establish the requirements for proper security practices and procedures.
        
        \item Directive controls form the foundation for security awareness and compliance.
        
        \item They help create a strong security culture within the organization.
    \end{itemize}
    
    \begin{alertblock}{Key Principle}
        Directive controls establish what people should do, rather than technically enforcing or preventing actions.
    \end{alertblock}
\end{frame}

\begin{frame}
    \frametitle{Directive Controls: Components}
    
    \begin{itemize}
        \item Organizations need various types of directive controls to guide different aspects of security.
        
        \item Each component addresses specific security behaviors and requirements.
        
        \item Directive controls must be clear, accessible, and regularly updated.
        
        \item Staff need proper training to understand and follow directive controls.
    \end{itemize}
    
    \begin{table}
        \begin{tabular}{|l|l|}
            \hline
            \textbf{Component} & \textbf{Purpose} \\
            \hline
            Security Policies & Define requirements \\
            Usage Guidelines & Direct daily activities \\
            Security Procedures & Guide specific tasks \\
            Standards & Set minimum expectations \\
            \hline
        \end{tabular}
    \end{table}
\end{frame}

\begin{frame}
    \frametitle{Directive Controls: Implementation}
    
    \begin{itemize}
        \item Effective implementation of directive controls requires clear communication and training.
        
        \item Organizations must ensure all staff understand their security responsibilities.
        
        \item \textbf{Regular updates} keep directive controls aligned with current security needs.
        
        \item Compliance monitoring helps ensure directive controls are being followed.
    \end{itemize}
    
    \begin{block}{Key Elements}
        \begin{enumerate}
            \item Clear documentation
            \item Staff training
            \item Regular updates
            \item Compliance tracking
            \item Performance feedback
        \end{enumerate}
    \end{block}
\end{frame}

\begin{frame}
    \frametitle{Review: Control Types Working Together}
    
    \begin{itemize}
        \item Different types of controls must work together to create effective security.
        
        \item Each control type serves a specific purpose in the overall security strategy.
        
        \item Organizations need a balanced mix of all control types.
        
        \item Regular assessment helps optimize the combination of controls.
    \end{itemize}
    
    \begin{exampleblock}{Control Type Integration}
        \begin{itemize}
            \item Preventive blocks threats
            \item Deterrent discourages attempts
            \item Detective identifies incidents
            \item Corrective fixes problems
            \item Compensating provides alternatives
            \item Directive guides behavior
        \end{itemize}
    \end{exampleblock}
\end{frame}

\begin{frame}
    \frametitle{Summary: Security Control Categories}
    
    \begin{itemize}
        \item \textbf{Technical Controls} provide automated protection through technology solutions.
        
        \item \textbf{Managerial Controls} guide security through policies and risk management.
        
        \item \textbf{Operational Controls} involve day-to-day procedures performed by people.
        
        \item \textbf{Physical Controls} protect tangible assets and facilities.
    \end{itemize}
    
    \begin{alertblock}{Key Takeaway}
        Effective security requires all categories working together in a coordinated approach.
    \end{alertblock}
\end{frame}

\begin{frame}
    \frametitle{Summary: Security Control Types}
    
    \begin{table}
        \begin{tabular}{|l|l|}
            \hline
            \textbf{Control Type} & \textbf{Primary Purpose} \\
            \hline
            Preventive & Stop incidents before they occur \\
            Deterrent & Discourage potential attacks \\
            Detective & Identify security violations \\
            Corrective & Fix problems after detection \\
            Compensating & Provide alternative protection \\
            Directive & Guide security behavior \\
            \hline
        \end{tabular}
    \end{table}
\end{frame}

\begin{frame}
    \frametitle{Review Questions}
    
    \begin{block}{Class Discussion}
        Consider these questions about security controls:
        
        \begin{enumerate}
            \item How do technical and physical controls work together to protect assets?
            
            \item Why might an organization need to implement compensating controls?
            
            \item What role do directive controls play in maintaining security?
            
            \item How can detective controls support corrective controls?
        \end{enumerate}
    \end{block}
\end{frame}

\begin{frame}
    \frametitle{Practical Exercise}
    
    \begin{exampleblock}{School Security Analysis}
        Analyze our school's security controls:
        \begin{itemize}
            \item Identify examples of each control category
            \item Determine which control types are being used
            \item Suggest potential improvements
            \item Consider how controls work together
        \end{itemize}
    \end{exampleblock}
    
    \begin{block}{Group Activity}
        Form small groups and create a security control plan for protecting:
        \begin{itemize}
            \item A computer lab
            \item Student data
            \item Physical access to the building
        \end{itemize}
    \end{block}
\end{frame}

\end{document}