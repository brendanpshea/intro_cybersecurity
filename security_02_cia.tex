\documentclass{beamer}
\usetheme{Madrid}
\usecolortheme{whale}

\usepackage{tikz}
\usepackage{listings}
\usepackage{xcolor}

\title{Information Security Fundamentals}
\subtitle{Understanding How We Protect Digital Information}
\author{Prepared by Security Team}
\date{\today}

\begin{document}

\begin{frame}
    \titlepage
\end{frame}

\begin{frame}
    \frametitle{Introduction to CIA: The Foundation of Security}
    \begin{block}{What is Information Security?}
        Just like we protect valuable things in the physical world with locks and safes, we need ways to protect our digital information.
    \end{block}
    \begin{itemize}
        \item Information security protects everything from your personal photos to your banking information from people who shouldn't have access to it.
        \item The \textbf{CIA triad} is like a three-part checklist that helps us make sure our information is properly protected.
        \item Every time you use a password, encryption, or verify a website's security, you're using CIA principles.
        \item We'll explore how these principles work together to keep our digital world safe.
    \end{itemize}
\end{frame}

\begin{frame}
    \frametitle{Confidentiality: Keeping Secrets Safe}
    \begin{itemize}
        \item \textbf{Confidentiality} is about keeping secrets secret - making sure only the right people can see sensitive information.
        \item Think about how we protect private information:
            \begin{itemize}
                \item \textbf{Passwords}: Like having a key to your digital house
                \item \textbf{Encryption}: Like putting a message in a special code
                \item \textbf{Access controls}: Like having an ID card to enter restricted areas
                \item \textbf{Private mode browsing}: Like leaving no footprints behind
            \end{itemize}
        \item When you send a private message, confidentiality ensures only the intended recipient can read it.
        \item Banks use confidentiality to protect your account information from unauthorized viewers.
    \end{itemize}
\end{frame}

\begin{frame}
    \frametitle{Integrity: Keeping Information Trustworthy}
    \begin{block}{Why Integrity Matters}
        Imagine if someone could change your grades or bank balance without permission - integrity prevents this!
    \end{block}
    \begin{itemize}
        \item \textbf{Integrity} means making sure information hasn't been tampered with or accidentally changed.
        \item When you download a file, your computer checks if it downloaded correctly and completely.
        \item Digital signatures are like a wax seal on a letter - they show if something has been changed.
        \item Social media platforms use integrity checks to ensure your posts aren't altered by others.
    \end{itemize}
\end{frame}

\begin{frame}
    \frametitle{Availability: Making Sure Information is There When You Need It}
    \begin{alertblock}{Availability Impact}
        Even brief system outages can have severe consequences - from lost sales to life-threatening situations
    \end{alertblock}
    
    \textbf{Availability} ensures systems work when legitimate users need them:
    \begin{itemize}
        \item Systems must respond quickly and reliably
        \item Backup systems provide redundancy when primary systems fail
        \item Load balancing prevents system overload
        \item Disaster recovery plans ensure business continuity
    \end{itemize}
\end{frame}

\begin{frame}
    \frametitle{Putting CIA Together: Real World Examples}
    Here's how CIA principles protect common services:
    \begin{itemize}
        \item \textbf{Mobile Banking}:
          \begin{itemize}
            \item Confidentiality: Encryption of transactions
            \item Integrity: Transaction verification codes
            \item Availability: Multiple server locations
          \end{itemize}
        \item \textbf{Email Services}:
          \begin{itemize}
            \item Confidentiality: Message encryption
            \item Integrity: Digital signatures
            \item Availability: Redundant storage
          \end{itemize}
    \end{itemize}
\end{frame}

\begin{frame}
    \frametitle{Non-repudiation: Taking Responsibility}
    \textbf{Non-repudiation} prevents users from denying their actions on a system.
    \begin{itemize}
        \item Key components of non-repudiation:
          \begin{itemize}
            \item Digital signatures on documents
            \item Secure timestamp services
            \item Audit log maintenance
            \item Access tracking systems
          \end{itemize}
        \item Common applications:
          \begin{itemize}
            \item Email communication records
            \item Financial transaction logs
            \item Document modification history
            \item System access records
          \end{itemize}
    \end{itemize}
\end{frame}

\begin{frame}
    \frametitle{Introduction to AAA: Authentication, Authorization, and Accounting}
    \begin{block}{The Three Steps of Access Control}
        Think of AAA like a secure building: checking ID (Authentication), determining where you can go (Authorization), and keeping records (Accounting)
    \end{block}
    
    \textbf{AAA} provides a framework for controlling system access:
    \begin{itemize}
        \item Authentication verifies identity claims
        \item Authorization determines access rights
        \item Accounting tracks user actions
        \item Together they create:
          \begin{itemize}
            \item Complete access control
            \item Audit capabilities
            \item Security compliance
            \item Incident investigation tools
          \end{itemize}
    \end{itemize}
\end{frame}

\begin{frame}
    \frametitle{Authentication: Proving Who You Are}
    \begin{block}{Authentication Factors}
        Something you know, something you have, something you are
    \end{block}
    \begin{itemize}
        \item \textbf{Authentication} is how you prove you are who you say you are in the digital world.
        \item Passwords and PINs are like digital keys that only you should know.
        \item Two-factor authentication is like needing both a key and an ID to enter.
        \item Using multiple factors makes it much harder for someone to pretend to be you.
    \end{itemize}
\end{frame}

\begin{frame}
    \frametitle{Authenticating People: Methods We Use}
    \begin{itemize}
        \item Common ways people prove their identity:
            \begin{itemize}
                \item \textbf{Knowledge-based}: 
                    \begin{itemize}
                        \item Passwords you remember
                        \item Security questions
                        \item PIN numbers
                    \end{itemize}
                \item \textbf{Possession-based}:
                    \begin{itemize}
                        \item Phone for SMS codes
                        \item Security keys
                        \item ID cards
                    \end{itemize}
                \item \textbf{Biometric}:
                    \begin{itemize}
                        \item Fingerprint scans
                        \item Face recognition
                        \item Voice patterns
                    \end{itemize}
            \end{itemize}
    \end{itemize}
\end{frame}

\begin{frame}
    \frametitle{Authenticating Systems: Machine Identity}
    \begin{itemize}
        \item \textbf{Machine authentication} ensures computers and devices can trust each other.
        \item When you visit a website, your browser checks its digital certificate - like checking a store's business license.
        \item Secure websites use HTTPS to prove they are legitimate, showing a padlock icon in your browser.
        \item Digital certificates are like ID cards for websites and servers, issued by trusted authorities.
    \end{itemize}
\end{frame}

\begin{frame}
    \frametitle{Authentication in Action: Real-World Examples}
    \begin{block}{Common Authentication Scenarios}
        Let's look at how authentication protects you every day
    \end{block}
    \begin{itemize}
        \item When you unlock your phone with a fingerprint, you're using biometric authentication.
        \item School computers might require both your student ID and password to log in.
        \item Online banking often uses multiple steps: password, security questions, and text message codes.
        \item Gaming consoles authenticate both you and your games to prevent unauthorized access.
    \end{itemize}
\end{frame}

\begin{frame}
    \frametitle{Authorization: Determining What You Can Do}
    \begin{block}{Authentication vs. Authorization}
        Authentication proves who you are, authorization determines what you're allowed to do
    \end{block}
    \begin{itemize}
        \item \textbf{Authorization} is like having different access cards for different areas of a building.
        \item Just because you can log into a system doesn't mean you can access everything in it.
        \item Your student ID might let you into the library but not the teacher's lounge.
        \item Different permission levels help protect sensitive information and resources.
    \end{itemize}
\end{frame}

\begin{frame}
    \frametitle{Authorization Models: Different Ways to Control Access}
    \begin{itemize}
        \item Common authorization models include:
            \begin{itemize}
                \item \textbf{Role-Based Access Control (RBAC)}:
                    \begin{itemize}
                        \item Like how students, teachers, and administrators have different permissions
                        \item Access based on your role, not who you are
                        \item Easier to manage for large organizations
                    \end{itemize}
                \item \textbf{Discretionary Access Control (DAC)}:
                    \begin{itemize}
                        \item Like when you choose who can see your social media posts
                        \item Owner decides who gets access
                        \item Common in personal computing
                    \end{itemize}
            \end{itemize}
    \end{itemize}
\end{frame}

\begin{frame}
    \frametitle{Access Control Lists (ACLs)}
    \begin{itemize}
        \item \textbf{Access Control Lists} are like guest lists that specify exactly who can do what.
        \item They work similarly to file permissions on your computer, controlling who can read, write, or modify.
        \item In social media, your friends list acts like an ACL for your private posts.
        \item ACLs can be very specific - like allowing someone to view a document but not edit it.
    \end{itemize}
\end{frame}

\begin{frame}
    \frametitle{Principle of Least Privilege}
    \begin{block}{A Fundamental Security Rule}
        Give users only the access they need to do their job - nothing more!
    \end{block}
    \begin{itemize}
        \item \textbf{Least privilege} is like giving a housesitter only the front door key, not keys to everything.
        \item This principle helps prevent accidental changes and limits what attackers can do if they break in.
        \item Apps on your phone ask permission only for what they need - they shouldn't get more access than necessary.
        \item Even administrators should use regular accounts for daily tasks, using admin access only when needed.
    \end{itemize}
\end{frame}

\begin{frame}
    \frametitle{Accounting: Keeping Track of What Happens}
    \begin{itemize}
        \item \textbf{Accounting} in security means creating detailed records of who did what and when.
        \item System logs track important events like failed login attempts, file changes, and permission changes.
        \item This information helps investigate security incidents and prove what happened - like a security camera's footage.
        \item Logs must be protected from tampering and backed up regularly to maintain their integrity.
        \item Good accounting practices help organizations comply with legal requirements and industry standards.
    \end{itemize}
\end{frame}

\begin{frame}
    \frametitle{Security Logs: What We Track}
    \begin{itemize}
        \item Important events we need to monitor:
            \begin{itemize}
                \item \textbf{Authentication Events}:
                    \begin{itemize}
                        \item Successful and failed login attempts
                        \item Password changes
                        \item Account lockouts
                    \end{itemize}
                \item \textbf{System Events}:
                    \begin{itemize}
                        \item File access and modifications
                        \item Software installations
                        \item System reboots
                    \end{itemize}
                \item \textbf{Security Events}:
                    \begin{itemize}
                        \item Firewall alerts
                        \item Antivirus detections
                        \item Permission changes
                    \end{itemize}
            \end{itemize}
    \end{itemize}
\end{frame}

\begin{frame}
    \frametitle{Gap Analysis: Finding Security Weaknesses}
    \begin{block}{What is a Security Gap?}
        A security gap is the difference between where your security is and where it needs to be
    \end{block}
    \begin{itemize}
        \item \textbf{Gap analysis} helps identify weaknesses in security systems, like finding holes in a fence.
        \item Organizations compare their current security measures against industry best practices and requirements.
        \item Regular assessments help catch problems before they can be exploited by attackers.
        \item Gap analysis leads to concrete recommendations for improving security.
        \item Think of it like a security health check-up that shows what needs improvement.
    \end{itemize}
\end{frame}

\begin{frame}
    \frametitle{Conducting a Basic Gap Analysis}
    \begin{itemize}
        \item Steps in performing a basic gap analysis:
            \begin{itemize}
                \item \textbf{Assessment}:
                    \begin{itemize}
                        \item Document current security measures
                        \item Review existing policies
                        \item Test security controls
                    \end{itemize}
                \item \textbf{Comparison}:
                    \begin{itemize}
                        \item Check against security standards
                        \item Review industry best practices
                        \item Consider legal requirements
                    \end{itemize}
                \item \textbf{Planning}:
                    \begin{itemize}
                        \item Prioritize identified gaps
                        \item Develop improvement plans
                        \item Set realistic timelines
                    \end{itemize}
            \end{itemize}
    \end{itemize}
\end{frame}

\begin{frame}
    \frametitle{Zero Trust: A Modern Security Approach}
    \begin{block}{Trust Nothing, Verify Everything}
        Traditional security trusted everything inside the network - Zero Trust trusts nothing by default
    \end{block}
    \begin{itemize}
        \item \textbf{Zero Trust} is like a security guard who checks everyone's ID, even if they work there.
        \item Traditional security was like a castle with strong walls but trust once inside.
        \item Modern networks need security everywhere because there is no clear "inside" anymore.
        \item Every access request is treated as potentially dangerous and must be verified.
        \item Working from home and cloud computing make Zero Trust especially important.
    \end{itemize}
\end{frame}

\begin{frame}
    \frametitle{Traditional vs. Zero Trust Security}
    \begin{table}
        \begin{tabular}{|p{0.45\textwidth}|p{0.45\textwidth}|}
            \hline
            \textbf{Traditional Security} & \textbf{Zero Trust} \\
            \hline
            Trust inside network & Trust nothing by default \\
            \hline
            Verify once at entry & Verify every request \\
            \hline
            Like castle walls & Like security checkpoints everywhere \\
            \hline
            Focus on perimeter & Security throughout system \\
            \hline
            Location-based trust & Identity-based trust \\
            \hline
            Static access rules & Dynamic access decisions \\
            \hline
        \end{tabular}
    \end{table}
\end{frame}


\begin{frame}
    \frametitle{Zero Trust in Action}
    Even with the right password, you might be denied access if:
    \begin{itemize}
        \item Your location suddenly changes (like logging in from another country).
        \item You're trying to access resources at unusual times.
        \item Your behavior patterns don't match your normal activity.
        \item The device you're using isn't recognized or secure.
        \item The system detects potential security risks in real-time.
        \item This dynamic approach helps catch potential security breaches early.
    \end{itemize}
\end{frame}

\begin{frame}
    \frametitle{Components of Zero Trust: Control Plane}

    \begin{block}{Understanding the Control Plane}
        The \textbf{Control Plane} is defined as the part of the Zero Trust system that makes decisions about who gets access to what.
    \end{block}

    \begin{itemize}
        \item Key elements that manage Zero Trust security:
            \begin{itemize}
                \item \textbf{Adaptive Identity}:
                    \begin{itemize}
                        \item Continuously evaluates user behavior
                        \item Adjusts access based on risk
                        \item Considers context like location and device
                    \end{itemize}
                \item \textbf{Policy Engine}:
                    \begin{itemize}
                        \item Makes real-time access decisions
                        \item Applies security rules consistently
                        \item Updates policies automatically
                    \end{itemize}
            \end{itemize}
        \item The control plane acts like a smart security system that's always watching and adjusting.
    \end{itemize}
\end{frame}

\begin{frame}
    \frametitle{Control Plane: Threat Scope Reduction}
    \begin{block}{Making the Target Smaller}
        The less attackers can see or access, the harder it is for them to cause harm
    \end{block}
    \begin{itemize}
        \item \textbf{Threat scope reduction} is like keeping valuables in separate safes rather than one big vault.
        \item Systems are divided into smaller, isolated segments to limit potential damage.
        \item Users can only see and access what they absolutely need for their work.
        \item Even if attackers break in somewhere, they can't easily reach other parts of the system.
        \item Regular access reviews help remove unnecessary permissions that could be exploited.
    \end{itemize}
\end{frame}



\begin{frame}
    \frametitle{Control Plane: Policy-Driven Access Control}
    \begin{block}{Automated Security Decisions}
        Policies are like a rulebook that automatically determines who gets access to what
    \end{block}
    \begin{itemize}
        \item \textbf{Policy-driven access} means using clear rules to make security decisions automatically.
        \item These policies consider multiple factors like user role, device security, and risk level.
        \item Rules can change automatically based on security threats or unusual activity.
        \item Think of it like a smart doorman who knows all the building's rules and applies them consistently.
        \item Policies must be detailed enough to be secure but flexible enough to allow legitimate work.
    \end{itemize}
\end{frame}

\begin{frame}
    \frametitle{Control Plane: The Policy Administrator}
    \begin{itemize}
        \item Components of policy administration:
            \begin{itemize}
                \item \textbf{Policy Creation}:
                    \begin{itemize}
                        \item Writing clear security rules
                        \item Defining access conditions
                        \item Setting up authentication requirements
                    \end{itemize}
                \item \textbf{Policy Management}:
                    \begin{itemize}
                        \item Updating rules as needed
                        \item Monitoring policy effectiveness
                        \item Responding to security incidents
                    \end{itemize}
                \item \textbf{Policy Enforcement}:
                    \begin{itemize}
                        \item Ensuring rules are followed
                        \item Logging policy violations
                        \item Taking action on violations
                    \end{itemize}
            \end{itemize}
    \end{itemize}
\end{frame}

\begin{frame}
    \frametitle{Understanding the Data Plane}

    \begin{block}{The Data Plane}
        The \textbf{Data Plane} is defined as the part of the Zero Trust system that actually enforces security policies and controls access.
    \end{block}

    \begin{itemize}
        \item Every time you try to access something, the Data Plane:
            \begin{itemize}
                \item Checks your identity and permissions
                \item Verifies your device's security status
                \item Ensures the connection is secure
                \item Monitors for suspicious behavior
            \end{itemize}
        \item This happens continuously, not just when you first connect.
        \item Even a brief security issue can cause access to be revoked immediately.
        \item The Data Plane works with the Control Plane to keep systems secure.
    \end{itemize}
\end{frame}

\begin{frame}
    \frametitle{Data Plane: Implicit Trust Zones}
    \begin{block}{What is an Implicit Trust Zone?}
        Areas where traditional security assumes everything is safe - a dangerous assumption!
    \end{block}
    \begin{itemize}
        \item An \textbf{implicit trust zone} is like assuming everyone in a school building is supposed to be there.
        \item Traditional networks trusted everything inside the company network.
        \item This old approach is risky because:
            \begin{itemize}
                \item One breach gives access to everything
                \item Insider threats go unnoticed
                \item Compromised devices spread problems
            \end{itemize}
        \item Zero Trust eliminates these assumed-safe zones entirely.
    \end{itemize}
\end{frame}

\begin{frame}
    \frametitle{Data Plane: Subject and System Interactions}
    \begin{itemize}
        \item In Zero Trust, every interaction between users (\textbf{subjects}) and resources (\textbf{systems}) must be verified.
        \item Examples of subject/system interactions:
        \item Opening a document requires checking:
            \begin{itemize}
                \item User identity and permissions
                \item Device security status
                \item File sensitivity level
                \item Location and time of access
            \end{itemize}
        \item These checks happen automatically and continuously.
        \item Even small changes in any factor can trigger a security response.
        \item The system maintains detailed logs of all interactions.
    \end{itemize}
\end{frame}

\begin{frame}
    \frametitle{Data Plane: Policy Enforcement Point (PEP)}
    \begin{block}{The Security Checkpoint}
        Like a guard checking IDs, the PEP verifies every request before allowing access
    \end{block}
    \begin{itemize}
        \item The \textbf{Policy Enforcement Point} acts as the security guard of the Zero Trust system.
        \item Every request must pass through the PEP, with no exceptions.
        \item The PEP communicates with the Policy Engine to make access decisions.
        \item It can immediately block access if security requirements aren't met.
        \item Modern PEPs are smart enough to consider context and adapt to changing conditions.
        \item They maintain detailed records of all access attempts, approved or denied.
    \end{itemize}
\end{frame}

\begin{frame}
    \frametitle{Introduction to Physical Security}
    \begin{alertblock}{Critical Reminder}
        Physical security failures can completely bypass even the strongest digital protections
    \end{alertblock}
    \begin{itemize}
        \item \textbf{Physical security} protects tangible assets and critical infrastructure
        \item Protection requires multiple elements:
          \begin{itemize}
            \item Deterrence measures
            \item Access control systems
            \item Detection mechanisms
            \item Response procedures
          \end{itemize}
        \item Every measure needs:
          \begin{itemize}
            \item Regular testing
            \item Backup systems
            \item Maintenance plans
          \end{itemize}
    \end{itemize}
\end{frame}

\begin{frame}
    \frametitle{Layers of Physical Security}
    \begin{itemize}
        \item Security works in distinct layers:
          \begin{itemize}
            \item \textbf{Perimeter Security}:
              \begin{itemize}
                \item Fences and walls
                \item Bollards and barriers
                \item Security lighting
                \item Surveillance systems
              \end{itemize}
            \item \textbf{Building Security}:
              \begin{itemize}
                \item Access control systems
                \item Security personnel
                \item Hardened entrances
                \item Emergency systems
              \end{itemize}
          \end{itemize}
    \end{itemize}
\end{frame}

\begin{frame}
    \frametitle{Perimeter Protection: Bollards and Barriers}
    \begin{itemize}
        \item \textbf{Bollards} protect against vehicle-based threats
        \item Types of bollards include:
          \begin{itemize}
            \item Fixed permanent posts
            \item Retractable systems
            \item Removable barriers
            \item Decorative options
          \end{itemize}
        \item Implementation considerations:
          \begin{itemize}
            \item Proper spacing requirements
            \item Impact resistance ratings
            \item Emergency access needs
            \item Aesthetic integration
          \end{itemize}
    \end{itemize}
\end{frame}

\begin{frame}
    \frametitle{Access Control Vestibules}
    \begin{block}{Security Vestibule Purpose}
        Creates a secure buffer zone where credentials can be verified before granting entry
    \end{block}
    \begin{itemize}
        \item \textbf{Access control vestibules} prevent unauthorized entry
        \item Required components:
          \begin{itemize}
            \item Two interlocked doors
            \item Authentication systems
            \item Surveillance cameras
            \item Emergency overrides
          \end{itemize}
        \item Security features:
          \begin{itemize}
            \item Anti-tailgating measures
            \item Contraband detection
            \item Physical isolation
          \end{itemize}
    \end{itemize}
\end{frame}

\begin{frame}
    \frametitle{Fencing and Physical Barriers}
    \begin{itemize}
        \item Types of security fencing:
            \begin{itemize}
                \item \textbf{Chain-link}:
                    \begin{itemize}
                        \item Basic perimeter marking
                        \item Can add barbed wire
                        \item Cost-effective solution
                    \end{itemize}
                \item \textbf{Anti-climb}:
                    \begin{itemize}
                        \item Mesh design prevents footholds
                        \item Higher security rating
                        \item Often used for sensitive areas
                    \end{itemize}
                \item \textbf{Crash-rated}:
                    \begin{itemize}
                        \item Stops vehicle attacks
                        \item Reinforced construction
                        \item Used at critical facilities
                    \end{itemize}
            \end{itemize}
    \end{itemize}
\end{frame}

\begin{frame}
    \frametitle{Video Surveillance Systems}
    \begin{block}{Modern CCTV: More Than Just Cameras}
        Today's systems use AI to detect suspicious behavior automatically
    \end{block}
    \begin{itemize}
        \item \textbf{Video surveillance} combines cameras, storage, and intelligent monitoring.
        \item Modern systems can detect unusual activities like:
        \item People in restricted areas or at unusual times.
        \item Objects left behind or removed.
        \item Suspicious behavior patterns.
        \item Facial recognition can track known threats.
        \item Systems maintain searchable archives for investigations.
        \item Integration with access control provides better security.
    \end{itemize}
\end{frame}

\begin{frame}
    \frametitle{Security Guards and Human Elements}
    \begin{itemize}
        \item \textbf{Security personnel} provide crucial functions that technology cannot:
        \item Make complex decisions in unusual situations.
        \item Respond to emergencies with appropriate judgment.
        \item Interact with visitors and employees professionally.
        \item Notice subtle behavioral cues that machines might miss.
        \item Key responsibilities include:
            \begin{itemize}
                \item Access control enforcement
                \item Patrol and monitoring
                \item Emergency response
                \item Visitor management
                \item Incident reporting
            \end{itemize}
    \end{itemize}
\end{frame}

\begin{frame}
    \frametitle{Access Badges and Credentials}
    \begin{table}
        \begin{tabular}{|p{0.3\textwidth}|p{0.6\textwidth}|}
            \hline
            \textbf{Badge Type} & \textbf{Security Features} \\
            \hline
            Basic ID & Photo, name, expiration date \\
            \hline
            Magnetic Stripe & Encoded data, swipe access \\
            \hline
            Proximity & Contactless, encrypted, harder to clone \\
            \hline
            Smart Card & Multiple credentials, high encryption \\
            \hline
            Multi-factor & Combined with PIN or biometrics \\
            \hline
        \end{tabular}
    \end{table}
    \begin{itemize}
        \item Badges should be visibly worn at all times.
        \item Lost badges must be reported immediately.
        \item Regular audits ensure only active badges work.
    \end{itemize}
\end{frame}

\begin{frame}
    \frametitle{Security Lighting Fundamentals}
    \begin{alertblock}{Essential Consideration}
        Security lighting must be on emergency power - darkness creates vulnerability
    \end{alertblock}
    
    \textbf{Security lighting} serves multiple critical purposes:
    \begin{itemize}
        \item Deters criminal activity by increasing visibility
        \item Enables effective camera surveillance at night
        \item Supports security personnel in monitoring
        \item Creates safe paths for emergency evacuation
    \end{itemize}
\end{frame}

\begin{frame}
    \frametitle{Types of Security Lighting}
    Common security lighting approaches:
    \begin{itemize}
        \item \textbf{Continuous Lighting}:
          \begin{itemize}
            \item Constant illumination
            \item Most common method
            \item Higher energy usage
            \item Best for high-security areas
          \end{itemize}
        \item \textbf{Standby Lighting}:
          \begin{itemize}
            \item Motion-activated operation
            \item Energy efficient design
            \item Psychological deterrent
            \item Good for low-traffic areas
          \end{itemize}
    \end{itemize}
\end{frame}

\begin{frame}
    \frametitle{Introduction to Security Sensors}
    \textbf{Security sensors} act as the nervous system of physical security, detecting various types of threats.
    \begin{itemize}
        \item Key deployment factors:
          \begin{itemize}
            \item Environmental conditions
            \item Coverage requirements
            \item False alarm rates
            \item Integration capabilities
          \end{itemize}
        \item Performance considerations:
          \begin{itemize}
            \item Detection accuracy
            \item Response time
            \item Maintenance needs
            \item Failure modes
          \end{itemize}
    \end{itemize}
\end{frame}

\begin{frame}
    \frametitle{Types of Security Sensors}
    \begin{table}
        \begin{tabular}{|p{0.25\textwidth}|p{0.3\textwidth}|p{0.35\textwidth}|}
            \hline
            \textbf{Sensor Type} & \textbf{Detection Method} & \textbf{Best Use Case} \\
            \hline
            Infrared & Heat detection & Indoor motion detection \\
            \hline
            Pressure & Weight/force changes & Secure entry points \\
            \hline
            Microwave & Movement detection & Large open areas \\
            \hline
            Ultrasonic & High-frequency sound & Small enclosed spaces \\
            \hline
        \end{tabular}
    \end{table}
    
    Implementation guidelines:
    \begin{itemize}
        \item Combine multiple sensor types for reliability
        \item Test regularly under various conditions
        \item Maintain proper calibration schedules
    \end{itemize}
\end{frame}

\begin{frame}
    \frametitle{Infrared Sensor Technology}
    \begin{block}{How Infrared Sensors Work}
        These sensors detect heat signatures from people, animals, and objects
    \end{block}
    \begin{itemize}
        \item \textbf{Passive Infrared (PIR)} sensors detect changes in heat patterns:
            \begin{itemize}
                \item Monitor temperature differences
                \item Identify movement through detection zones
                \item Work well in complete darkness
                \item Can be fooled by rapid temperature changes
            \end{itemize}
        \item Modern PIR sensors include:
        \item Advanced signal processing to reduce false alarms.
        \item Pet-immune variations for home security.
        \item Integration with video systems for verification.
    \end{itemize}
\end{frame}

\begin{frame}
    \frametitle{Pressure and Contact Sensors}
    \begin{itemize}
        \item \textbf{Pressure sensors} detect physical force or weight changes:
        \item Common applications include:
            \begin{itemize}
                \item Floor mats near secure entries
                \item Fence and wall monitoring
                \item Underground intrusion detection
                \item Vehicle detection systems
            \end{itemize}
        \item Advanced features now include:
        \item Weight range discrimination for different threats.
        \item Pattern recognition for normal versus suspicious activity.
        \item Integration with access control systems.
        \item Weatherproof designs for outdoor use.
    \end{itemize}
\end{frame}

\begin{frame}
    \frametitle{Wave-Based Detection Systems}
    \begin{itemize}
        \item Two main types of wave-based sensors:
            \begin{itemize}
                \item \textbf{Microwave}:
                    \begin{itemize}
                        \item Uses radio waves
                        \item Covers large areas
                        \item Penetrates thin walls
                        \item Good for outdoor use
                    \end{itemize}
                \item \textbf{Ultrasonic}:
                    \begin{itemize}
                        \item Uses high-frequency sound
                        \item Best for enclosed spaces
                        \item Doesn't penetrate walls
                        \item Very sensitive to movement
                    \end{itemize}
            \end{itemize}
        \item Both types can work through darkness, smoke, or fog.
    \end{itemize}
\end{frame}

\begin{frame}
    \frametitle{Introduction to Deception Technology}
    \begin{block}{A New Approach to Security}
        Instead of just defending, deception technology tricks attackers into revealing themselves
    \end{block}
    \begin{itemize}
        \item \textbf{Deception technology} creates traps and decoys to catch attackers:
        \item Looks like legitimate systems but monitors for unauthorized access.
        \item Provides early warning of potential attacks.
        \item Wastes attacker time and resources.
        \item Helps gather information about attack methods.
        \item Can be both physical and digital deceptions.
    \end{itemize}
\end{frame}

\begin{frame}
    \frametitle{Understanding Honeypots}
    \begin{alertblock}{Security Note}
        While honeypots are powerful tools, they must be carefully isolated from production systems to prevent them from becoming a security risk.
    \end{alertblock}
    \begin{itemize}
        \item \textbf{Honeypots} are decoy systems designed to attract potential attackers
        \item Types of honeypots include:
          \begin{itemize}
            \item High-interaction: Full system emulation
            \item Medium-interaction: Service emulation
            \item Low-interaction: Port monitoring only
          \end{itemize}
        \item Common implementation targets:
          \begin{itemize}
            \item Web servers and applications
            \item Database systems
            \item IoT device simulations
          \end{itemize}
    \end{itemize}
\end{frame}

\begin{frame}
    \frametitle{Honeynets: Networks of Deception}
    \begin{itemize}
        \item A \textbf{honeynet} combines multiple honeypots in a network
        \item Standard components include:
          \begin{itemize}
            \item Fake web servers and services
            \item Simulated databases
            \item Decoy file shares
            \item Mock user accounts
          \end{itemize}
        \item Key benefits:
          \begin{itemize}
            \item Early attack detection
            \item Threat pattern analysis
            \item Attacker technique study
            \item Automated response testing
          \end{itemize}
    \end{itemize}
\end{frame}

\begin{frame}
    \frametitle{Honeyfiles and Document Tracking}
    \begin{table}
        \begin{tabular}{|p{0.3\textwidth}|p{0.6\textwidth}|}
            \hline
            \textbf{Honeyfile Type} & \textbf{Purpose} \\
            \hline
            Password Lists & Detect credential theft attempts \\
            \hline
            Fake Documents & Track unauthorized access \\
            \hline
            Decoy Spreadsheets & Monitor data exfiltration \\
            \hline
            Configuration Files & Identify system probing \\
            \hline
        \end{tabular}
    \end{table}
    \begin{itemize}
        \item \textbf{Honeyfiles} are decoy documents that alert when accessed
        \item Deployment strategies include:
          \begin{itemize}
            \item Strategic placement in shared drives
            \item Integration with DLP systems
            \item Automated alert mechanisms
          \end{itemize}
    \end{itemize}
\end{frame}

\begin{frame}
    \frametitle{Honeytokens: Digital Breadcrumbs}
    \begin{itemize}
        \item \textbf{Honeytokens} are pieces of fake data designed to detect theft
        \item Common implementations:
          \begin{itemize}
            \item Fake login credentials
            \item Invalid credit card numbers
            \item Decoy API keys
            \item Bogus email addresses
          \end{itemize}
        \item Detection capabilities:
          \begin{itemize}
            \item Data breach tracking
            \item Insider threat identification
            \item Exfiltration monitoring
            \item Attack attribution
          \end{itemize}
    \end{itemize}
\end{frame}

\end{document}