\documentclass{beamer}
\usetheme{Copenhagen}
\usecolortheme{dolphin}
\useinnertheme{circles}

% Packages
\usepackage{graphicx}
\usepackage{booktabs}
\usepackage{xcolor}
\usepackage{listings}

% Custom colors
\definecolor{mygreen}{RGB}{0,150,0}
\definecolor{myred}{RGB}{150,0,0}
\definecolor{myblue}{RGB}{0,0,150}

% Title information
\title{Understanding and Analyzing Malicious Activity}
\subtitle{A High School Introduction to Cybersecurity Threats}
\author{Cybersecurity Fundamentals}
\date{\today}

\begin{document}

% Title slide
\begin{frame}
    \titlepage
\end{frame}

% Slide 1
\begin{frame}
    \frametitle{Understanding Cybersecurity Threats: An Introduction}
    
    \begin{itemize}
        \item Cybersecurity threats are intentional actions designed to compromise digital systems and information.
        \item \textbf{Malicious activity} refers to any action intended to harm, disrupt, or gain unauthorized access to computer systems.
        \item Almost 80\% of organizations experienced at least one successful cyber attack in the past year.
        \item Understanding different types of threats is the first step in effective defense.
        \item This presentation covers the major categories of cybersecurity threats and how to identify them.
    \end{itemize}
\end{frame}

% Slide 2
\begin{frame}
    \frametitle{The Importance of Recognizing Malicious Activity}
    
    \begin{alertblock}{Why Detection Matters}
        Early detection of threats can significantly reduce damage and recovery costs.
    \end{alertblock}
    
    \begin{itemize}
        \item The average data breach costs organizations over \$4 million in damages and recovery.
        \item \textbf{Threat indicators} are observable evidence that an attack may be in progress.
        \item Most successful attacks show warning signs before major damage occurs:
            \begin{itemize}
                \item Unusual network traffic patterns
                \item Unexpected system behavior
                \item Anomalous login activities
                \item Suspicious file modifications
            \end{itemize}
        \item Building a "security mindset" helps protect both organizations and individuals.
    \end{itemize}
\end{frame}

% Slide 3
\begin{frame}
    \frametitle{Key Threat Categories in the Digital World}
    
    \begin{table}
        \centering
        \begin{tabular}{l|l}
            \textbf{Category} & \textbf{Examples} \\
            \hline
            Malware & Ransomware, Trojans, Viruses \\
            Physical & Brute force, RFID cloning \\
            Network & DDoS, DNS attacks, Wireless \\
            Application & Injection, Buffer overflow \\
            Cryptographic & Downgrade, Collision attacks \\
        \end{tabular}
    \end{table}
    
    \begin{itemize}
        \item Each category exploits different vulnerabilities within digital systems.
        \item Attackers often combine multiple attack vectors for maximum impact.
        \item Defenses must address all categories to be effective.
    \end{itemize}
    
    \begin{example}
        In 2017, the WannaCry ransomware attack affected over 200,000 computers across 150 countries, showing how quickly digital threats can spread globally.
    \end{example}
\end{frame}

% Slide 4
\begin{frame}
    \frametitle{Basic Concepts in Threat Identification}
    
    \begin{block}{The Security Triad}
        \textbf{Confidentiality}, \textbf{Integrity}, and \textbf{Availability} form the core principles of information security.
    \end{block}
    
    \begin{itemize}
        \item \textbf{Threat actors} are individuals or groups who initiate attacks:
            \begin{itemize}
                \item Nation-states
                \item Cybercriminals
                \item Hacktivists
                \item Insiders
            \end{itemize}
        \item \textbf{Attack vectors} are pathways used to gain unauthorized access.
        \item \textbf{Vulnerabilities} are weaknesses that can be exploited.
        \item \textbf{Indicators of compromise (IoCs)} are evidence that an attack has occurred.
    \end{itemize}
\end{frame}
% Slide 5
\begin{frame}
    \frametitle{Ransomware: When Your Data Is Held Hostage}
    
    \begin{itemize}
        \item \textbf{Ransomware} is malicious software that encrypts victim's files and demands payment for the decryption key.
        \item Common infection vectors include:
            \begin{itemize}
                \item Phishing emails with malicious attachments
                \item Drive-by downloads from compromised websites
                \item Exploiting unpatched security vulnerabilities
            \end{itemize}
        \item Modern ransomware employs strong encryption that makes recovery without the key virtually impossible.
        \item Attackers typically demand payment in cryptocurrency to maintain anonymity.
    \end{itemize}
    
    \begin{alertblock}{Prevention Focus}
        Regular backups stored offline are one of the best defenses against ransomware attacks.
    \end{alertblock}
\end{frame}

% Slide 6
\begin{frame}
    \frametitle{Trojans: Deceptive Packages with Hidden Dangers}
    
    \begin{table}
        \centering
        \begin{tabular}{l|l}
            \textbf{Trojan Type} & \textbf{Primary Function} \\
            \hline
            Banking & Steal financial credentials \\
            RAT & Remote Access/Control \\
            Downloader & Install additional malware \\
            Backdoor & Create persistent access \\
            Spyware & Monitor user activity \\
        \end{tabular}
    \end{table}
    
    \begin{itemize}
        \item \textbf{Trojan horses} are malware disguised as legitimate software to trick users into installation.
        \item Unlike viruses, Trojans do not self-replicate but rely on user deception.
        \item They're often distributed through social engineering tactics that exploit trust.
    \end{itemize}
    
    \begin{example}
        Michael Scott downloaded what he thought was a free screen saver, but it was actually a Trojan that gave hackers access to Dunder Mifflin's customer database.
    \end{example}
\end{frame}

% Slide 7
\begin{frame}
    \frametitle{Worms: Self-Replicating Digital Threats}
    
    \begin{block}{Key Characteristic}
        The defining feature of worms is their ability to spread automatically without user interaction.
    \end{block}
    
    \begin{itemize}
        \item \textbf{Computer worms} are standalone malware that replicate and spread across networks without requiring host files.
        \item Worms exploit network vulnerabilities to propagate rather than attaching to programs.
        \item Impact of worm infections:
            \begin{itemize}
                \item Network bandwidth consumption
                \item System resource depletion
                \item Service disruption
                \item Delivery of secondary payloads
            \end{itemize}
        \item The rapid spread makes worms particularly dangerous for large organizations.
    \end{itemize}
\end{frame}

% Slide 8
\begin{frame}
    \frametitle{Spyware: When Someone Is Watching}
    
    \begin{columns}
        \begin{column}{0.5\textwidth}
            \begin{itemize}
                \item \textbf{Spyware} is malicious software designed to gather information without the user's knowledge.
                \item Common information targets:
                    \begin{itemize}
                        \item Browsing history
                        \item Login credentials
                        \item Financial details
                        \item Personal messages
                    \end{itemize}
                \item Often remains hidden while monitoring activity.
            \end{itemize}
        \end{column}
        \begin{column}{0.5\textwidth}
            \begin{alertblock}{Warning Signs}
                \begin{itemize}
                    \item System slowdowns
                    \item Unexpected pop-ups
                    \item Browser changes
                    \item Strange network activity
                    \item Battery drain (mobile)
                \end{itemize}
            \end{alertblock}
        \end{column}
    \end{columns}
\end{frame}

% Slide 9
\begin{frame}
    \frametitle{Bloatware: Resource Drain and Hidden Risks}
    
    \begin{itemize}
        \item \textbf{Bloatware} refers to unwanted software that consumes excessive system resources while providing limited value.
        \item While not always malicious, bloatware can contain hidden components that compromise security.
        \item Common sources of bloatware:
            \begin{itemize}
                \item Pre-installed on new devices
                \item Bundled with other software downloads
                \item Adware installations
                \item Trial software that remains after expiration
            \end{itemize}
        \item Some bloatware collects user data for advertising purposes without clear disclosure.
    \end{itemize}
    
    \begin{block}{Gray Area Threat}
        Bloatware exists in a gray area between legitimate software and malware, making it challenging to classify and address.
    \end{block}
\end{frame}

% Slide 10
\begin{frame}
    \frametitle{Viruses: How They Infect and Spread}
    
    \begin{columns}
        \begin{column}{0.5\textwidth}
            \begin{itemize}
                \item \textbf{Computer viruses} are malicious programs that replicate by inserting copies into other programs.
                \item Require a host file and user action to spread between systems.
                \item Unlike worms, viruses need a "host" application and cannot propagate independently.
            \end{itemize}
        \end{column}
        \begin{column}{0.5\textwidth}
            \begin{table}
                \centering
                \small
                \begin{tabular}{l|l}
                    \textbf{Virus Type} & \textbf{Behavior} \\
                    \hline
                    Boot sector & Infects startup code \\
                    File infector & Attaches to executables \\
                    Macro & Uses document macros \\
                    Polymorphic & Changes its code \\
                \end{tabular}
            \end{table}
        \end{column}
    \end{columns}
    
    \begin{example}
        Leslie Knope opened an email attachment about a "Parks Award," accidentally releasing a virus that replaced all documents on the Parks Department server with pictures of raccoons.
    \end{example}
\end{frame}

% Slide 11
\begin{frame}
    \frametitle{Keyloggers: Tracking Every Keystroke}
    
    \begin{alertblock}{Privacy Invasion}
        Keyloggers can capture sensitive information including passwords, credit card numbers, and private communications.
    \end{alertblock}
    
    \begin{itemize}
        \item \textbf{Keyloggers} are surveillance tools that record keystrokes made by a computer user.
        \item Implementation methods:
            \begin{itemize}
                \item Software applications
                \item Browser extensions
                \item Hardware devices (USB or keyboard adapters)
                \item Firmware modifications
            \end{itemize}
        \item Malicious keyloggers are often installed through phishing, social engineering, or bundled with other malware.
        \item Modern keyloggers may also capture screenshots, monitor clipboard content, and track browsing activity.
    \end{itemize}
\end{frame}

% Slide 12
\begin{frame}
    \frametitle{Logic Bombs: Time-Triggered Attacks}
    
    \begin{itemize}
        \item \textbf{Logic bombs} are malicious code segments programmed to execute when specific conditions are met.
        \item Common trigger conditions:
        
        \begin{tabular}{l|l}
            \textbf{Trigger Type} & \textbf{Example} \\
            \hline
            Temporal & Specific date/time \\
            Logical & File presence/absence \\
            Quantitative & Login count threshold \\
            Environmental & Network connection status \\
        \end{tabular}
        
        \item Often planted by insiders with programming access, making them difficult to detect.
        \item Logic bombs can lie dormant for extended periods before activation.
        \item Common payloads include data deletion, encryption, or creating backdoor access.
    \end{itemize}
    
    \begin{block}{Detection Challenge}
        Because logic bombs remain inactive until their trigger condition, they often evade traditional security scanning.
    \end{block}
\end{frame}

% Slide 13
\begin{frame}
    \frametitle{Rootkits: Hidden Deep in Your System}
    
    \begin{columns}
        \begin{column}{0.6\textwidth}
            \begin{itemize}
                \item \textbf{Rootkits} are collections of tools designed to gain and maintain unauthorized administrator-level access.
                \item They operate at the lowest levels of the operating system, making detection extremely difficult.
                \item Rootkits can modify system files and processes to hide their presence from security software.
                \item They often create persistent backdoors that survive system reboots and software updates.
            \end{itemize}
        \end{column}
        \begin{column}{0.4\textwidth}
            \begin{table}
                \centering
                \small
                \begin{tabular}{l|l}
                    \textbf{Type} & \textbf{Target} \\
                    \hline
                    User-mode & Applications \\
                    Kernel-mode & OS core \\
                    Bootkit & Boot process \\
                    Firmware & Hardware \\
                \end{tabular}
            \end{table}
        \end{column}
    \end{columns}
    
    \begin{alertblock}{Stealth Threat}
        The primary danger of rootkits is their ability to remain undetected while enabling ongoing system compromise.
    \end{alertblock}
\end{frame}

% Slide 14
\begin{frame}
    \frametitle{Brute Force Attacks: Breaking Down the Door}
    
    \begin{block}{What Is a Brute Force Attack?}
        A systematic attempt to discover passwords or keys by trying all possible combinations until the correct one is found.
    \end{block}
    
    \begin{itemize}
        \item \textbf{Brute force attacks} attempt to gain unauthorized access by systematically trying all possible combinations.
        \item Common targets:
            \begin{itemize}
                \item Login credentials
                \item Encryption keys
                \item PIN codes
                \item API keys
            \end{itemize}
        \item Success depends on computing power and the complexity of the target password or key.
        \item Modern systems typically implement countermeasures like account lockouts and rate limiting.
    \end{itemize}
    
    \begin{alertblock}{Time to Crack}
        \small
        A password with complexity of:
        \begin{itemize}
            \item 6 lowercase letters: Minutes to hours
            \item 8 mixed case + numbers: Days to weeks
            \item 12+ mixed case + symbols: Years to centuries
        \end{itemize}
    \end{alertblock}
\end{frame}

% Slide 15
\begin{frame}
    \frametitle{RFID Cloning: Digital Identity Theft}
    
    \begin{itemize}
        \item \textbf{Radio-frequency identification (RFID)} technology uses electromagnetic fields to identify and track tags attached to objects.
        \item \textbf{RFID cloning} involves copying data from a legitimate RFID tag to create an unauthorized duplicate.
        \item Common targets:
        
        \begin{tabular}{l|l}
            \textbf{Target} & \textbf{Use Case} \\
            \hline
            Access cards & Building entry \\
            Credit cards & Contactless payment \\
            Passport chips & Border control \\
            Product tags & Inventory tracking \\
        \end{tabular}
        
        \item Attackers can use specialized readers to capture RFID data from several feet away without physical contact.
        \item Cloned tags can be used to gain unauthorized physical access or make fraudulent transactions.
    \end{itemize}
    
    \begin{example}
        Jim Halpert noticed his office access card worked even after he reported it lost. Security footage revealed Dwight had cloned Jim's RFID card to monitor his arrival and departure times.
    \end{example}
\end{frame}

% Slide 16
\begin{frame}
    \frametitle{Environmental Attacks: Exploiting Physical Vulnerabilities}
    
    \begin{alertblock}{Physical Security Matters}
        Even the strongest digital security can be compromised if physical access controls are inadequate.
    \end{alertblock}
    
    \begin{itemize}
        \item \textbf{Environmental attacks} exploit physical conditions and surroundings of computing systems.
        \item Categories of environmental attacks:
            \begin{itemize}
                \item \textbf{Temperature manipulation}: Overheating or cooling systems to cause failures
                \item \textbf{Power attacks}: Deliberate power surges, cutting power during critical operations
                \item \textbf{Electromagnetic attacks}: Using EM radiation to interfere with or monitor devices
                \item \textbf{Acoustic attacks}: Using sound to extract information from certain devices
            \end{itemize}
        \item Social engineering often accompanies these attacks to gain initial physical access.
    \end{itemize}
\end{frame}

% Slide 17
\begin{frame}
    \frametitle{DDoS Attacks: Overwhelming the Target}
    
    \begin{columns}
        \begin{column}{0.55\textwidth}
            \begin{itemize}
                \item \textbf{Distributed Denial of Service (DDoS)} attacks aim to make online services unavailable by overwhelming them with traffic.
                \item Unlike regular DoS attacks, DDoS utilizes multiple compromised computer systems as attack sources.
                \item These attacks can generate hundreds of gigabits per second of malicious traffic.
                \item Common targets include websites, online services, and DNS providers.
            \end{itemize}
        \end{column}
        \begin{column}{0.45\textwidth}
            \begin{table}
                \centering
                \small
                \begin{tabular}{l|l}
                    \textbf{Layer} & \textbf{Attack Type} \\
                    \hline
                    Network & ICMP Flood \\
                    Transport & SYN Flood \\
                    Session & SSL Abuse \\
                    Application & HTTP Flood \\
                \end{tabular}
            \end{table}
        \end{column}
    \end{columns}
    
\end{frame}

% Slide 18
\begin{frame}
    \frametitle{Amplified \& Reflected DDoS: Multiplying the Impact}
    
    \begin{alertblock}{Amplification Factor}
        Some reflection techniques can multiply the attacker's traffic by factors of 50x or more.
    \end{alertblock}
    
    \begin{itemize}
        \item \textbf{Amplified DDoS attacks} exploit protocols that return larger responses than the initial request.
        \item \textbf{Reflected attacks} bounce traffic off third-party servers to hide the attacker's identity.
        \item Common protocols exploited:
        
        \begin{tabular}{l|c}
            \textbf{Protocol} & \textbf{Amplification Factor} \\
            \hline
            DNS & 28-54x \\
            NTP & 556-1,337x \\
            SSDP & 30x \\
            Memcached & 10,000-51,000x \\
        \end{tabular}
        
        \item These attacks are particularly dangerous because they require minimal resources from the attacker.
    \end{itemize}
\end{frame}

% Slide 19
\begin{frame}
    \frametitle{DNS Attacks: Compromising the Internet's Directory}
    
    \begin{itemize}
        \item The \textbf{Domain Name System (DNS)} translates human-readable domain names into IP addresses.
        \item Common DNS attack types:
            \begin{itemize}
                \item \textbf{DNS cache poisoning}: Inserts fraudulent records into DNS resolvers
                \item \textbf{DNS tunneling}: Abuses DNS protocols to exfiltrate data
                \item \textbf{DNS hijacking}: Modifies DNS settings to redirect users
                \item \textbf{DNS amplification}: Uses DNS servers for DDoS attacks
            \end{itemize}
        \item These attacks can be difficult to detect because DNS traffic is typically trusted by network security systems.
    \end{itemize}
    
    \begin{example}
        Ron Swanson typed "www.parksandrec.gov" into his browser but was redirected to a fake government website that asked for his credentials. The IT department later discovered the office DNS settings had been hijacked.
    \end{example}
\end{frame}

% Slide 20
\begin{frame}
    \frametitle{Wireless Network Vulnerabilities}
    
    \begin{block}{Convenience vs. Security}
        The ease of access that makes wireless networks convenient also creates unique security challenges.
    \end{block}
    
    \begin{itemize}
        \item Common wireless attack vectors:
    \end{itemize}
    
    \begin{table}
        \centering
        \begin{tabular}{l|l}
            \textbf{Attack Type} & \textbf{Description} \\
            \hline
            Evil Twin & Rogue access points mimicking legitimate networks \\
            WPA/WPA2 Cracking & Breaking wireless encryption using captured handshakes \\
            Jamming & Using radio interference to disrupt communications \\
            Wardriving & Scanning for vulnerable networks from a vehicle \\
            Packet Sniffing & Capturing and analyzing unencrypted wireless traffic \\
        \end{tabular}
    \end{table}
  
\end{frame}

% Slide 21
\begin{frame}
    \frametitle{On-Path Attacks: The Digital Eavesdropper}
    
    \begin{columns}
        \begin{column}{0.6\textwidth}
            \begin{itemize}
                \item \textbf{On-path attacks} (or \textbf{man in the middle}) occur when an attacker secretly relays and possibly alters communications between two parties.
                \item The victims believe they are communicating directly with each other, unaware of the attacker in between.
                \item These attacks can be used to intercept data, steal credentials, or manipulate transactions.
            \end{itemize}
        \end{column}
        \begin{column}{0.4\textwidth}
            \begin{itemize}
                \item Common vectors:
                    \begin{itemize}
                        \item ARP spoofing
                        \item DNS spoofing
                        \item Evil twin Wi-Fi
                        \item SSL stripping
                        \item BGP hijacking
                    \end{itemize}
            \end{itemize}
        \end{column}
    \end{columns}
    
    \begin{center}
        \begin{tabular}{c|c|c}
            Sender & Attacker & Receiver \\
            \hline
            Thinks they're & Intercepts & Thinks they're \\
            sending to receiver & \& modifies & receiving from sender \\
        \end{tabular}
    \end{center}
\end{frame}

% Slide 22
\begin{frame}
    \frametitle{Credential Replay: Using Stolen Authentication}
    
    \begin{itemize}
        \item \textbf{Credential replay attacks} involve capturing authentication data and reusing it to gain unauthorized access.
        \item These attacks don't require knowing the actual password, only the authentication tokens or hashes.
        \item Common targets:
            \begin{itemize}
                \item Authentication cookies
                \item Session tokens
                \item Kerberos tickets
                \item OAuth tokens
                \item NTLM hashes
            \end{itemize}
        \item Attackers typically capture credentials using network sniffers, on-path attacks, or malware.
    \end{itemize}
    
    \begin{block}{Modern Challenge}
        Even with strong passwords, replay attacks can succeed if the authentication protocol itself doesn't prevent reuse.
    \end{block}
\end{frame}

% Slide 23
\begin{frame}
    \frametitle{Malicious Code in Network Traffic}
    
    \begin{itemize}
        \item Network traffic can be manipulated to deliver \textbf{malicious code} directly to target systems.
        \item Delivery mechanisms:
        
        \begin{tabular}{l|l}
            \textbf{Method} & \textbf{Description} \\
            \hline
            Code injection & Inserting code into legitimate web traffic \\
            Malvertising & Malicious code embedded in online ads \\
            Drive-by downloads & Silent downloads from compromised sites \\
            Traffic manipulation & Altering packets in transit to add malicious code \\
        \end{tabular}
        
        \item Network-level exploits may target vulnerabilities in how systems process network packets.
        \item Content filtering and deep packet inspection can help identify and block malicious code in transit.
    \end{itemize}
    
    \begin{example}
        While using the office Wi-Fi, Pam Beesly noticed strange redirects when browsing. IT discovered malicious JavaScript was being injected into web pages as they passed through a compromised router.
    \end{example}
\end{frame}

% Slide 24
\begin{frame}
    \frametitle{Injection Attacks: Exploiting Input Vulnerabilities}
    
    \begin{alertblock}{OWASP Top Threat}
        Injection attacks consistently rank among the most dangerous web application security risks in the OWASP Top Ten.
    \end{alertblock}
    
    \begin{columns}
        \begin{column}{0.5\textwidth}
            \begin{itemize}
                \item \textbf{Injection attacks} occur when untrusted data is sent to an interpreter as part of a command or query.
                \item Common injection types:
                    \begin{itemize}
                        \item SQL injection
                        \item Command injection
                        \item LDAP injection
                        \item XSS (Cross-Site Scripting)
                    \end{itemize}
            \end{itemize}
        \end{column}
        \begin{column}{0.6\textwidth}
            \begin{block}{SQL Injection Example}
                Intended: \texttt{SELECT * FROM users WHERE username = 'input1' AND password = 'input2'} \newline
                
                Malicious input1:
                \texttt{admin' --} \newline
                
                Resulting query:
                \texttt{SELECT * FROM users WHERE username = 'admin'} \newline
            \end{block}
        \end{column}
    \end{columns}
\end{frame}

% Slide 25
\begin{frame}
    \frametitle{Buffer Overflows: When Memory Fails}
    
    \begin{columns}
        \begin{column}{0.6\textwidth}
            \begin{itemize}
                \item \textbf{Buffer overflow} attacks occur when a program writes data beyond the allocated memory buffer.
                \item These vulnerabilities arise from poor programming practices and inadequate input validation.
                \item Successful exploitation can lead to:
                    \begin{itemize}
                        \item System crashes
                        \item Data corruption
                        \item Arbitrary code execution
                        \item Privilege escalation
                    \end{itemize}
            \end{itemize}
        \end{column}
        \begin{column}{0.4\textwidth}
            \begin{alertblock}{Protection Methods}
                \begin{itemize}
                    \item ASLR
                    \item DEP/NX
                    \item Stack canaries
                    \item Bounds checking
                    \item Safe libraries
                \end{itemize}
            \end{alertblock}
        \end{column}
    \end{columns}
    
    \begin{block}{Low-Level Vulnerability}
        Buffer overflows exploit how computers manage memory, making them particularly dangerous but also more difficult to execute.
    \end{block}
\end{frame}

% Slide 26
\begin{frame}
    \frametitle{Replay Attacks on Applications}
    
    \begin{alertblock}{Not Just Network Traffic}
        Application-level replay differs from network replay by targeting specific application functions rather than authentication alone.
    \end{alertblock}
    
    \begin{itemize}
        \item \textbf{Application replay attacks} involve capturing valid data transmissions and retransmitting them to trick an application.
        \item Attack scenarios and targets:
        
        \begin{tabular}{l|l}
            \textbf{Target} & \textbf{Attack Scenario} \\
            \hline
            Financial transactions & Duplicating money transfers \\
            API requests & Replaying authorized API calls \\
            Authentication sequences & Reusing login credentials \\
            Session management & Hijacking user sessions \\
        \end{tabular}
        
        \item Successful attacks can lead to transaction duplication, session hijacking, or privilege escalation.
 
    \end{itemize}
\end{frame}

% Slide 27
\begin{frame}
    \frametitle{Privilege Escalation: Gaining Unauthorized Access}
    
    \begin{columns}
        \begin{column}{0.5\textwidth}
            \begin{itemize}
                \item \textbf{Privilege escalation} involves gaining elevated access to resources that should be protected.
                \item Two primary types:
                    \begin{itemize}
                        \item \textbf{Vertical}: Gaining higher permission levels
                        \item \textbf{Horizontal}: Accessing resources of peers
                    \end{itemize}
                \item Attackers often chain multiple vulnerabilities to achieve full system compromise.
            \end{itemize}
        \end{column}
        \begin{column}{0.5\textwidth}
            \begin{block}{Common Vectors}
                \begin{itemize}
                    \item Misconfigured permissions
                    \item Unpatched vulnerabilities
                    \item Default credentials
                    \item Application flaws
                    \item Token manipulation
                    \item Path traversal
                \end{itemize}
            \end{block}
        \end{column}
    \end{columns}

\end{frame}

% Slide 28
\begin{frame}
    \frametitle{Forgery Attacks: Digital Counterfeiting}
    
    \begin{block}{Types of Forgery}
        Forgery attacks can target requests, responses, or the underlying identities used in digital systems.
    \end{block}
    
    \begin{tabular}{|p{3cm}|p{8cm}|}
        \textbf{Attack Type} & \textbf{Description} \\
        \hline
        Cross-site request forgery (CSRF) & Tricks users into submitting unwanted requests \\
        Server-side request forgery (SSRF) & Forces server to make unauthorized internal requests \\
        Cookie forgery & Creates/modifies cookies to impersonate users \\
        Email forgery & Manipulates email headers to fake message origins \\
    \end{tabular}
    
\end{frame}

% Slide 29
\begin{frame}
    \frametitle{Directory Traversal: Escaping the Sandbox}
    
    \begin{alertblock}{Also Known As}
        Directory traversal is sometimes called the "dot-dot-slash" attack, referring to the "../" sequence used to navigate directories.
    \end{alertblock}
    
    \begin{columns}
        \begin{column}{0.6\textwidth}
            \begin{itemize}
                \item \textbf{Directory traversal} attacks exploit insufficient security validation to access files outside intended directories.
                \item Attackers use path manipulation techniques like "../" sequences to navigate the file system.
                \item Common targets include web servers, file upload functions, and content management systems.
            \end{itemize}
        \end{column}
        \begin{column}{0.4\textwidth}
            \begin{block}{Example Attack String}
                \texttt{../../etc/passwd}\\
                \texttt{..\\..\\windows\\system32\\}
            \end{block}
        \end{column}
    \end{columns}
    
\end{frame}

% Slide 30
\begin{frame}
    \frametitle{Downgrade \& Collision Attacks: Weakening Encryption}
    
    \begin{columns}
        \begin{column}{0.5\textwidth}
            \begin{itemize}
                \item \textbf{Downgrade attacks} force systems to use weaker encryption protocols than they normally would.
                \item These attacks exploit backward compatibility features to use deprecated, less secure algorithms.
                \item Notable examples:
                    \begin{itemize}
                        \item POODLE (SSL 3.0)
                        \item FREAK (Export-grade RSA)
                        \item Logjam (Weak DH parameters)
                    \end{itemize}
            \end{itemize}
        \end{column}
        \begin{column}{0.5\textwidth}
            \begin{itemize}
                \item \textbf{Collision attacks} identify different inputs that produce the same cryptographic hash value.
                \item When successful, collision attacks can be used to:
                    \begin{itemize}
                        \item Forge digital signatures
                        \item Bypass integrity checks
                        \item Create malicious duplicates
                        \item Break certificate validation
                    \end{itemize}
            \end{itemize}
        \end{column}
    \end{columns}
    
    \begin{block}{Cryptographic Aging}
        Encryption algorithms that were once considered secure can become vulnerable over time as computing power increases.
    \end{block}
\end{frame}

% Slide 31
\begin{frame}
    \frametitle{Birthday Attacks: Probability in Cryptography}
    
    \begin{columns}
        \begin{column}{0.5\textwidth}
            \begin{itemize}
                \item \textbf{Birthday attacks} are named after the birthday paradox in probability theory.
                \item The paradox shows that in a group of just 23 people, there's a 50\% chance two share a birthday.
                \item Similarly, finding a collision in an n-bit hash function requires approximately $2^{n/2}$ attempts, not $2^n$.
                \item This makes finding collisions much easier than would be intuitively expected.
            \end{itemize}
        \end{column}
        \begin{column}{0.5\textwidth}
            \begin{table}
                \centering
                \small
                \begin{tabular}{l|c}
                    \textbf{Hash Size} & \textbf{Security Level} \\
                    \hline
                    128-bit & $2^{64}$ operations \\
                    160-bit & $2^{80}$ operations \\
                    256-bit & $2^{128}$ operations \\
                    512-bit & $2^{256}$ operations \\
                \end{tabular}
            \end{table}
        \end{column}
    \end{columns}
    
    \begin{block}{Practical Implication}
        Hash functions used for security applications must be significantly larger than might seem necessary to resist birthday attacks.
    \end{block}
\end{frame}

% Slide 32
\begin{frame}
    \frametitle{Password Spraying: Casting a Wide Net}
    
    \begin{alertblock}{Why It Works}
        Password spraying evades account lockout mechanisms by trying just a few common passwords against many accounts rather than many passwords against one account.
    \end{alertblock}
    
    \begin{itemize}
        \item \textbf{Password spraying} is a brute force technique that attempts a small number of commonly used passwords against many accounts.
        \item Attack methodology:
            \begin{enumerate}
                \item Gather a list of valid usernames or email addresses
                \item Select a small set of commonly used passwords
                \item Try each password against all accounts before moving to the next password
                \item Space attempts to avoid triggering lockout policies
            \end{enumerate}
        \item Surprisingly effective due to password reuse and predictable password patterns.
        \item Mitigation requires strong password policies, MFA, and monitoring for multiple failed login attempts across different accounts.
    \end{itemize}
\end{frame}

\end{document}