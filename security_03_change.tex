\documentclass{beamer}
\usetheme{Madrid}
\usecolortheme{dolphin}

\usepackage{xcolor}
\usepackage{listings}
\usepackage{booktabs}

\title{Change Management in Cybersecurity}
\author{Brendan Shea, PhD}
\date{\today}

\begin{document}

\begin{frame}
\titlepage
\end{frame}

\begin{frame}
\frametitle{Why Change Management Matters in Cybersecurity}
\begin{itemize}
    \item \textbf{Change management} is the systematic approach to handling all changes made to a system or IT infrastructure. It ensures changes are implemented securely and efficiently.
    
    \item Security breaches often occur during system changes when proper protocols aren't followed. Over 80\% of security incidents can be traced back to poorly managed changes.
    
    \item Uncontrolled changes can create vulnerabilities, disrupt business operations, and compromise data integrity.
    
    \item Effective change management helps maintain system stability while implementing necessary security updates and improvements.
\end{itemize}
\end{frame}

\begin{frame}
\frametitle{Protecting Your Systems Through Smart Changes}
\begin{alertblock}{Key Principle}
Changes should enhance security without compromising system stability
\end{alertblock}
\begin{itemize}
    \item Every change to your system creates a potential security risk. Smart changes minimize these risks through careful planning and execution.
    
    \item \textbf{Risk assessment} must be performed before implementing any change to identify potential security vulnerabilities.
    
    \item Changes should follow the principle of \textbf{least privilege} - implementing only what is necessary to achieve the desired outcome.
    
    \item Successful change management requires balancing security requirements with operational needs.
\end{itemize}
\end{frame}

\begin{frame}
\frametitle{The Business Side of Security Changes}
\begin{itemize}
    \item Security changes affect multiple aspects of business operations. Understanding these impacts is crucial for successful implementation.
    
    \item \textbf{Business continuity} must be maintained throughout the change process. This includes planning for potential disruptions and having contingency plans.
    
    \item Changes need to align with business objectives while maintaining required security standards and compliance requirements.
    
    \item The \textbf{total cost of ownership (TCO)} for security changes includes implementation, training, maintenance, and potential business impact costs.
\end{itemize}
\end{frame}

\begin{frame}
    \frametitle{Who Needs to Approve Security Changes?}
    \begin{columns}[t]
    \column{0.5\textwidth}
    \textbf{Required Approvers:}
    \begin{itemize}
        \item IT Security Team
        \item System Owners
        \item Department Managers
        \item Executive Sponsor
    \end{itemize}
    \column{0.5\textwidth}
    \textbf{Approval Steps:}
    \begin{itemize}
        \item Initial Review
        \item Technical Assessment
        \item Risk Evaluation
        \item Final Authorization
    \end{itemize}
    \end{columns}
    \end{frame}
    
    \begin{frame}
    \frametitle{Key Players: Security Stakeholders}
    \begin{itemize}
        \item \textbf{Primary stakeholders} include system users, IT staff, and security teams who interact with the system daily.
        
        \item \textbf{Secondary stakeholders} encompass compliance officers, auditors, and business unit managers who oversee system usage.
        
        \item Each stakeholder group has unique security concerns and requirements that must be addressed during changes.
        
        \item Effective communication between stakeholders is essential for successful security implementation.
    \end{itemize}
    \end{frame}
    
    \begin{frame}
    \frametitle{Who Owns What? Understanding Ownership}
    \begin{block}{Definition}
    \textbf{System ownership} refers to the responsibility and accountability for a system's security, maintenance, and operation.
    \end{block}
    \begin{itemize}
        \item \textbf{Technical owners} are responsible for implementing and maintaining security controls.
        
        \item \textbf{Business owners} make decisions about system access, functionality, and risk acceptance.
        
        \item Ownership includes responsibility for approving changes and ensuring security compliance.
        
        \item Clear ownership definitions prevent confusion during security incidents and change implementation.
    \end{itemize}
    \end{frame}
    
    \begin{frame}
    \frametitle{Measuring the Impact of Security Changes}
    \begin{itemize}
        \item \textbf{Impact analysis} evaluates how security changes affect system functionality, user access, and business processes.
        
        \item Changes must be categorized by their potential impact level: Low, Medium, or High risk to operations.
        
        \item Consider both immediate effects and long-term consequences of security modifications.
    \end{itemize}
    \begin{table}
    \begin{tabular}{lll}
    \toprule
    \textbf{Impact Level} & \textbf{Response Time} & \textbf{Approval Needed} \\
    \midrule
    Low & 24-48 hours & Team Lead \\
    Medium & 1 week & Department Head \\
    High & 2+ weeks & Executive Team \\
    \bottomrule
    \end{tabular}
    \end{table}
    \end{frame}

    \begin{frame}
        \frametitle{Testing Before Deploying: Best Practices}
        \begin{itemize}
            \item \textbf{Test results} provide concrete evidence that security changes work as intended and don't introduce new vulnerabilities.
            
            \item Every security change must undergo multiple testing phases: unit testing, integration testing, and user acceptance testing.
            
            \item Document all test outcomes, including unexpected behaviors or failures, to improve future change implementations.
            
            \item \textbf{Testing environments} should mirror production settings as closely as possible to ensure accurate results.
        \end{itemize}
        \end{frame}
        
        \begin{frame}
        \frametitle{When Things Go Wrong: Backout Plans}
        \begin{exampleblock}{Example Backout Scenario}
        During a firewall update, new rules accidentally block legitimate traffic. The backout plan allows immediate restoration of previous rules while maintaining security.
        \end{exampleblock}
        \begin{itemize}
            \item A \textbf{backout plan} is your emergency exit strategy when security changes cause unexpected problems.
            
            \item Every change must include detailed steps for reversing modifications if necessary.
            
            \item Backout procedures should be tested before implementing major changes.
            
            \item Time requirements for reverting changes must be clearly documented and communicated.
        \end{itemize}
        \end{frame}
        
        \begin{frame}
        \frametitle{Planning Your Maintenance Window}
        \begin{itemize}
            \item A \textbf{maintenance window} is a scheduled period when systems can be safely modified with minimal impact on operations.
        \end{itemize}
        \begin{table}
        \begin{tabular}{lll}
        \toprule
        \textbf{Change Type} & \textbf{Typical Window} & \textbf{Notice Required} \\
        \midrule
        Minor Patches & 2-4 hours & 48 hours \\
        Major Updates & 4-8 hours & 1 week \\
        System Upgrades & 8+ hours & 2 weeks \\
        \bottomrule
        \end{tabular}
        \end{table}
        \begin{itemize}
            \item Schedule windows during periods of lowest system usage to minimize disruption.
            
            \item Always include buffer time for unexpected complications and thorough testing.
        \end{itemize}
        \end{frame}
        
        \begin{frame}
        \frametitle{Standard Operating Procedure: Your Security Playbook}
        \begin{block}{Definition}
        A \textbf{Standard Operating Procedure (SOP)} is a documented process that describes the steps required to complete a security task consistently and securely.
        \end{block}
        \begin{enumerate}
            \item SOPs must include detailed steps for implementing common security changes.
            
            \item Procedures should be written clearly enough for any qualified team member to follow.
            
            \item Regular reviews and updates of SOPs ensure they remain relevant and effective.
            
            \item Documentation should include both technical steps and required approvals.
        \end{enumerate}
        \end{frame}

        \begin{frame}
            \frametitle{Technical Changes: The Building Blocks}
            \begin{alertblock}{Critical Reminder}
            All technical changes must be documented, tested, and approved before implementation.
            \end{alertblock}
            \begin{itemize}
                \item \textbf{Technical changes} form the foundation of security improvements but require careful management to prevent system disruption.
                
                \item Each technical modification must be evaluated for potential security implications and system dependencies.
                
                \item Changes should follow the principle of \textbf{least privilege} and be implemented incrementally when possible.
                
                \item Document all technical specifications, including configuration changes and script modifications.
            \end{itemize}
            \end{frame}
            
            \begin{frame}
            \frametitle{Allow Lists vs. Deny Lists}
            \begin{columns}[t]
            \column{0.5\textwidth}
            \textbf{Allow Lists:}
            \begin{itemize}
                \item Explicitly permit specific actions
                \item Default stance: deny all
                \item More restrictive approach
                \item Better security control
            \end{itemize}
            \column{0.5\textwidth}
            \textbf{Deny Lists:}
            \begin{itemize}
                \item Explicitly block specific actions
                \item Default stance: allow all
                \item More permissive approach
                \item Easier to maintain
            \end{itemize}
            \end{columns}
            \end{frame}
            
            \begin{frame}
            \frametitle{Setting Boundaries: Restricted Activities}
            \begin{itemize}
                \item \textbf{Restricted activities} are actions that require special authorization or monitoring due to their potential security impact.
                
                \item Security policies must clearly define which activities are restricted and who can authorize them.
                
                \item Implementation of restrictions should be automated where possible to ensure consistent enforcement.
                
                \item Regular audits of restricted activities help identify potential security policy violations or needed policy updates.
            \end{itemize}
            \begin{block}{Common Restricted Activities}
            \begin{itemize}
                \item Administrative access to critical systems
                \item Database schema modifications
                \item Firewall rule changes
                \item Remote access to secure networks
            \end{itemize}
            \end{block}
            \end{frame}
            
            \begin{frame}
            \frametitle{Managing System Downtime}
            \begin{table}
            \begin{tabular}{lll}
            \toprule
            \textbf{System Type} & \textbf{Max Downtime} & \textbf{Recovery Time} \\
            \midrule
            Critical & 15 minutes & \textless 5 minutes \\
            Important & 1 hour & \textless 30 minutes \\
            Non-critical & 4 hours & \textless 2 hours \\
            \bottomrule
            \end{tabular}
            \end{table}
            \begin{itemize}
                \item \textbf{Downtime} must be carefully planned and communicated to minimize impact on business operations.
                
                \item Different systems have different tolerance levels for downtime based on their criticality.
                
                \item Always include buffer time in downtime estimates to account for unexpected complications.
                
                \item Maintain clear communication channels during downtime periods for status updates and emergency escalation.
            \end{itemize}
            \end{frame}

            \begin{frame}
                \frametitle{Service and Application Restarts}
                \begin{block}{Key Concept}
                A \textbf{restart procedure} is a documented set of steps for safely stopping and starting services while maintaining security controls.
                \end{block}
                \begin{itemize}
                    \item Services and applications must be restarted in a specific order to maintain security dependencies.
                    
                    \item Always verify security controls are active after restarts before allowing user access.
                    
                    \item Maintain logs of all restart activities for security auditing and troubleshooting.
                    
                    \item Include verification steps to ensure all security features are functioning properly post-restart.
                \end{itemize}
                \end{frame}
                
                \begin{frame}
                \frametitle{Legacy Systems: Managing Old Technology}
                \begin{itemize}
                    \item \textbf{Legacy applications} are older systems that may lack modern security features but remain critical to operations.
                    
                    \item Security changes must be carefully tested on legacy systems to prevent unexpected failures.
                    
                    \item Implementation of compensating controls may be necessary to protect legacy systems.
                \end{itemize}
                \begin{alertblock}{Warning}
                Legacy systems often require additional security measures to compensate for outdated security features. Never assume legacy systems meet current security standards without verification.
                \end{alertblock}
                \end{frame}
                
                \begin{frame}
                \frametitle{Understanding System Dependencies}
                \begin{itemize}
                    \item A \textbf{system dependency} is a relationship between two systems where one relies on the other for functionality or security.
                    \item \textbf{Direct dependencies} are immediate relationships, while \textbf{indirect dependencies} are secondary relationships that may not be immediately apparent.
                \end{itemize}
                \begin{columns}[t]
                \column{0.5\textwidth}
                
                \textbf{Direct Dependencies:}
                \begin{itemize}
                    \item Authentication services
                    \item Database connections
                    \item Network services
                    \item Security controls
                \end{itemize}
                \column{0.5\textwidth}
                \textbf{Indirect Dependencies:}
                \begin{itemize}
                    \item Backup systems
                    \item Monitoring tools
                    \item Logging services
                    \item Audit systems
                \end{itemize}
                \end{columns}
                
                \end{frame}
                
                \begin{frame}
                \frametitle{Documentation: Why It Matters}
                \begin{itemize}
                    \item \textbf{Documentation} provides a clear record of all security changes, configurations, and decisions.
                    
                    \item Well-maintained documentation helps track security modifications and troubleshoot issues.
                    
                    \item Documentation serves as evidence for security audits and compliance requirements.
                    
                    \item Proper documentation enables knowledge transfer and consistent security implementation across teams.
                \end{itemize}
                \begin{exampleblock}{Documentation Best Practice}
                Create and maintain a centralized repository for all security-related documentation, including change logs, configuration details, and approval records.
                \end{exampleblock}
                \end{frame}

                \begin{frame}
                    \frametitle{Keeping Your Network Diagrams Current}
                    \begin{block}{Definition}
                    \textbf{Network diagrams} are visual representations of system infrastructure that must be updated with every security change.
                    \end{block}
                    \begin{itemize}
                        \item Network diagrams should include all security controls, access points, and system boundaries.
                        
                        \item Updates must reflect both physical and logical security changes to the infrastructure.
                        
                        \item Maintain separate versions for different security classification levels when necessary.
                        
                        \item Regular review of network diagrams helps identify potential security gaps or unauthorized changes.
                    \end{itemize}
                    \end{frame}
                    
                    \begin{frame}
                    \frametitle{Updating Security Policies and Procedures}
                    \begin{itemize}
                        \item \textbf{Security policies} must evolve to address new threats and changes in system infrastructure.
                        
                        \item Policy updates require careful review and approval from security stakeholders.
                        
                        \item Changes to procedures must align with overall security policy guidelines.
                    \end{itemize}
                    \begin{table}
                    \begin{tabular}{lll}
                    \toprule
                    \textbf{Document Type} & \textbf{Review Frequency} & \textbf{Approver} \\
                    \midrule
                    Policies & Annual & Executive Team \\
                    Procedures & Quarterly & Security Lead \\
                    Guidelines & Semi-annual & Department Head \\
                    \bottomrule
                    \end{tabular}
                    \end{table}
                    \end{frame}
                    
                    \begin{frame}
                    \frametitle{Version Control Basics}
                    \textbf{Version control} ensures all team members work with current, approved security configurations. Track all changes systematically to maintain security compliance.
                    \begin{columns}[t]
                    \column{0.5\textwidth}
                    \textbf{What to Track:}
                    \begin{itemize}
                        \item Configuration files
                        \item Security scripts
                        \item System documentation
                        \item Policy documents
                    \end{itemize}
                    \column{0.5\textwidth}
                    \textbf{How to Track:}
                    \begin{itemize}
                        \item Version numbers
                        \item Change descriptions
                        \item Author information
                        \item Timestamps
                    \end{itemize}
                    \end{columns}
                    
                    \end{frame}
                    
                    \begin{frame}
                    \frametitle{Putting It All Together: Change Management Success}
                    \begin{alertblock}{Critical Success Factors}
                    Effective security change management requires planning, documentation, testing, and consistent communication.
                    \end{alertblock}
                    \begin{itemize}
                        \item Remember that change management is a continuous process of improvement and adaptation.
                        
                        \item Security changes must balance risk mitigation with operational needs.
                        
                        \item Documentation and version control provide the foundation for sustainable security practices.
                        
                        \item Regular reviews and updates keep your security posture strong and current.
                    \end{itemize}
                    \end{frame}

                    \begin{frame}
                        \frametitle{Key Takeaways: The Pillars of Change Management}
                        \begin{columns}[t]
                        \column{0.5\textwidth}
                        \textbf{Process Elements:}
                        \begin{itemize}
                            \item Approval workflows
                            \item Documentation
                            \item Testing procedures
                            \item Backout plans
                        \end{itemize}
                        \column{0.5\textwidth}
                        \textbf{Technical Elements:}
                        \begin{itemize}
                            \item System dependencies
                            \item Security controls
                            \item Version control
                            \item Monitoring
                        \end{itemize}
                        \end{columns}
                        \end{frame}
                        
                        \begin{frame}
                        \frametitle{Real-World Applications}
                        \begin{block}{Case Study Scenario}
                        Your school is implementing a new student information system. Consider all the change management elements that should be addressed.
                        \end{block}
                        \begin{itemize}
                            \item What stakeholders need to be involved in the approval process?
                            
                            \item How would you handle the transition from the old system?
                            
                            \item What security considerations are most important?
                            
                            \item How would you document and test the changes?
                        \end{itemize}
                        \end{frame}
                        
                        \begin{frame}
                        \frametitle{Discussion Questions}
                        \begin{itemize}
                            \item Why is it important to have a backout plan even for small changes? Provide an example of when you might need one.
                            
                            \item How does proper change management help prevent security incidents? Can you think of a real-world example?
                            
                            \item What challenges might arise when implementing changes in systems with legacy applications?
                            
                            \item How would you balance the need for quick security updates with proper change management procedures?
                        \end{itemize}
                        \end{frame}
                        
                        \begin{frame}
                        \frametitle{Group Activity: Change Management Simulation}
                        \begin{alertblock}{Scenario}
                        Your team needs to implement a critical security patch across multiple systems.
                        \end{alertblock}
                        \begin{enumerate}
                            \item Break into groups of 3-4 students
                            
                            \item Each group develops a change management plan including:
                            \begin{itemize}
                                \item Timeline and maintenance window
                                \item Required approvals
                                \item Testing procedures
                                \item Backout strategy
                            \end{itemize}
                            
                            \item Groups present their plans and discuss different approaches
                            
                            \item Class evaluates each plan's strengths and potential risks
                        \end{enumerate}
                        \end{frame}
                        
                        \end{document}

\end{document}