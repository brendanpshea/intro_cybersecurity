\documentclass{beamer}
\usetheme{Madrid}
\usecolortheme{whale}
\usepackage{graphicx}
\usepackage{amsmath}
\usepackage{hyperref}
\usepackage{listings}

\title{Cybersecurity Threat Actors: Compare and Contrast}
\subtitle{A Comprehensive Overview}
\author{Brendan Shea, PhD}
\date{\today}

\begin{document}

\begin{frame}
    \titlepage
\end{frame}

%%%% SLIDE 1 %%%%
\begin{frame}
    \frametitle{Understanding the Cyber Threat Landscape: An Overview}
    
    \begin{itemize}
        \item Cybersecurity threats have evolved significantly in complexity and impact over the past decades.
        \item \textbf{Threat actors} are individuals or groups who have the potential to cause harm to information systems and networks.
        \item The global cost of cybercrime is projected to reach \$10.5 trillion annually by 2025, highlighting the importance of understanding threats.
        \item Modern cybersecurity requires identifying not just attack methods, but the actors behind them and their motivations.
        \item This knowledge enables organizations to build more effective, targeted defense strategies.
    \end{itemize}
\end{frame}

%%%% SLIDE 2 %%%%
\begin{frame}
    \frametitle{Why Study Threat Actors? The Importance of Know Your Enemy}
    
    \begin{alertblock}{Security Principle}
        Understanding who might attack you and why is fundamental to effective defense.
    \end{alertblock}
    
    \begin{itemize}
        \item Different threat actors employ different tactics, techniques, and procedures (TTPs).
        \item Knowing potential attackers helps prioritize defenses against the most likely threats.
        \item \textbf{Threat intelligence} involves gathering and analyzing information about threat actors to improve security posture.
        \item Defenses against nation-state actors differ significantly from those targeting opportunistic criminals.
        \item Early identification of threat actor signatures can dramatically reduce incident response time.
    \end{itemize}
\end{frame}

%%%% SLIDE 3 %%%%
\begin{frame}
    \frametitle{The Evolution of Cyber Threats: Past, Present, and Future}
    
    \begin{itemize}
        \item The 1980s-1990s: Early hackers were primarily motivated by curiosity and technical challenge.
        \item The 2000s: Rise of financially motivated cybercrime and the formation of underground economies.
        \item The 2010s: Emergence of state-sponsored cyber operations and sophisticated persistent threats.
        \item Current landscape: Blurred lines between threat actor categories with shared tools and techniques.
        \item Future trends point toward more automated attacks, AI-powered threats, and increased targeting of emerging technologies.
    \end{itemize}
\end{frame}

%%%% SLIDE 4 %%%%
\begin{frame}
    \frametitle{Nation-State Actors: Government-Sponsored Cyber Operations}
    
    \begin{block}{Key Characteristics}
        Nation-state actors typically have extensive resources, sophisticated capabilities, and strategic objectives aligned with national interests.
    \end{block}
    
    \begin{itemize}
        \item \textbf{Nation-state actors} are government-sponsored groups that conduct cyber operations to further national interests.
        \item These actors typically maintain the most sophisticated and persistent attack capabilities.
        \item Primary motivations include espionage, critical infrastructure sabotage, and military advantage.
        \item Examples include APT28 (Russia), APT1 (China), Equation Group (attributed to NSA), and Lazarus Group (North Korea).
        \item Nation-state attacks often feature custom malware, zero-day exploits, and multi-year campaign timeframes.
    \end{itemize}
\end{frame}

\begin{frame}
    \frametitle{Case Study: Stuxnet and Nation-State Capabilities}
    
    \begin{block}{Background}
        Discovered in 2010, Stuxnet targeted Iranian nuclear centrifuges at the Natanz uranium enrichment facility.
    \end{block}
    
    \begin{itemize}
        \item Stuxnet demonstrated unprecedented sophistication, using four zero-day vulnerabilities and stolen digital certificates.
        \item The malware was specifically designed to target Siemens industrial control systems used in uranium enrichment.
        \item It represented the first known case of malware designed to cause physical damage to critical infrastructure.
        \item Attribution points to a joint US-Israeli operation codenamed "Olympic Games."
        \item This case illustrates how nation-state actors can combine intelligence resources, technical expertise, and strategic patience.
    \end{itemize}
\end{frame}

%%%% SLIDE 5 %%%%
\begin{frame}
    \frametitle{Script Kiddies and Unskilled Attackers: Low Sophistication, High Impact}
    
    \begin{itemize}
        \item \textbf{Script kiddies} are inexperienced attackers who use existing tools and exploits without understanding the underlying technology.
        \item Despite low sophistication, these actors can cause significant damage due to the availability of automated attack tools.
        \item Motivations typically include curiosity, desire for notoriety, or simple mischief rather than financial gain.
        \item These attackers often target vulnerable systems indiscriminately rather than focusing on specific organizations.
        \item The democratization of hacking tools has significantly increased the number of unskilled threat actors.
    \end{itemize}
\end{frame}

%%%% SLIDE 6 %%%%
\begin{frame}
    \frametitle{Hacktivists: When Digital Activism Meets Cyber Capabilities}
    
    \begin{alertblock}{Notable Example}
        The Anonymous collective has conducted operations against organizations they perceive as corrupt, including governments, corporations, and religious institutions.
    \end{alertblock}
    
    \begin{itemize}
        \item \textbf{Hacktivism} refers to hacking for politically or socially motivated purposes rather than financial gain.
        \item Hacktivist operations typically seek to bring attention to causes through website defacement, DDoS attacks, or data leaks.
        \item These actors often operate in loose collectives rather than rigid hierarchical structures.
        \item Their technical sophistication varies widely, from basic DDoS attacks to complex data exfiltration.
        \item Hacktivists frequently announce their campaigns publicly to maximize awareness of their cause.
    \end{itemize}
\end{frame}

%%%% SLIDE 7 %%%%
\begin{frame}
    \frametitle{The Insider Threat: Dangers from Within}
    
    \begin{itemize}
        \item \textbf{Insider threats} come from individuals with legitimate access to an organization's systems and data.
        \item These actors can be current or former employees, contractors, or business partners with authorized access.
        \item Insider attacks are particularly dangerous because they bypass many perimeter security controls.
        \item Motivations include financial gain, revenge for perceived wrongs, ideological disagreements, or coercion by outside actors.
        \item Studies suggest insider threats are responsible for approximately 22\% of security incidents but tend to be the most costly.
    \end{itemize}
\end{frame}

%%%% SLIDE 8 %%%%
\begin{frame}
    \frametitle{Organized Crime in Cyberspace: Digital Profit Centers}
    
    \begin{block}{Business Model}
        Cybercriminal groups have evolved sophisticated business models including Ransomware-as-a-Service (RaaS), which allows affiliates to deploy attacks while sharing profits with the malware developers.
    \end{block}
    
    \begin{itemize}
        \item \textbf{Cyber criminal organizations} operate like businesses, with hierarchical structures and specialized roles.
        \item Primary motivation is financial gain through ransomware, banking trojans, credential theft, and fraud.
        \item These groups maintain advanced technical capabilities and often recruit skilled developers and security experts.
        \item Modern cybercrime groups have developed sophisticated supply chains and partnerships in the criminal underground.
        \item Examples include groups like FIN7, Carbanak, and various ransomware gangs like REvil and DarkSide.
    \end{itemize}
\end{frame}

%%%% CASE STUDY 2: CYBERCRIMINAL GROUPS %%%%
\begin{frame}
    \frametitle{Case Study: Conti Ransomware Group Operations}
    
    \begin{alertblock}{Impact}
        The Conti ransomware group extorted over \$180 million from victims in 2021 alone, targeting healthcare, government, and critical infrastructure.
    \end{alertblock}
    
    \begin{itemize}
        \item Conti operated a sophisticated business model with specialized teams for initial access, ransomware deployment, and negotiations.
        \item The group maintained a detailed wiki, help desk, and salary structure mimicking legitimate software companies.
        \item In 2022, an insider leaked Conti's internal communications and source code following geopolitical disagreements.
        \item The group leveraged "double extortion" tactics, both encrypting data and threatening to publish stolen information.
        \item This case demonstrates the professionalization and business-like operation of modern cybercriminal enterprises.
    \end{itemize}
\end{frame}


%%%% SLIDE 9 %%%%
\begin{frame}
    \frametitle{Shadow IT: The Accidental Threat Actor}
    
    \begin{itemize}
        \item \textbf{Shadow IT} refers to information technology systems deployed by departments without explicit organizational approval.
        \item These unofficial systems often lack proper security oversight, creating vulnerabilities in the organization's security posture.
        \item Unlike malicious threat actors, shadow IT practitioners typically have legitimate business objectives but create risk inadvertently.
        \item Common examples include cloud services, productivity apps, and communication tools deployed without IT department knowledge.
        \item Studies suggest that 40\% of IT spending occurs outside the IT department in large enterprises.
    \end{itemize}
\end{frame}

%%%% SLIDE 10 %%%%
\begin{frame}
    \frametitle{Internal vs. External Threats: Comparing Access and Impact}
    
    \begin{alertblock}{Security Challenge}
        Organizations must balance protection against external attackers while maintaining appropriate monitoring for insider threats without creating a culture of distrust.
    \end{alertblock}
    
    \begin{itemize}
        \item \textbf{Internal threats} originate from within the organization's security perimeter and exploit legitimate access.
        \item \textbf{External threats} come from outside the organization and must first breach perimeter defenses.
        \item Internal actors often have deeper knowledge of organizational systems and where valuable data resides.
        \item External actors typically have more resources and can target multiple organizations simultaneously.
        \item Detection methods differ significantly: external threats often leave evidence of intrusion while internal threats may appear as normal activity.
    \end{itemize}
\end{frame}

%%%% SLIDE 11 %%%%
\begin{frame}
    \frametitle{Following the Money: How Resource Levels Shape Attack Capabilities}
    
    \begin{itemize}
        \item Threat actor resources directly correlate with their attack sophistication, persistence, and scale.
        \item \textbf{Low-resource actors} typically rely on publicly available tools and target vulnerable systems opportunistically.
        \item \textbf{Medium-resource actors} can develop custom tools and sustain operations over weeks or months.
        \item \textbf{High-resource actors} (like nation-states) can develop zero-day exploits, maintain persistent access for years, and target hardened systems.
        \item Resource considerations include not just financial capital but human expertise, infrastructure, and time availability.
    \end{itemize}
\end{frame}

%%%% SLIDE 12 %%%%
\begin{frame}
    \frametitle{Sophistication Spectrum: From Basic Scripts to Advanced Persistent Threats}
    
    \begin{block}{APT Definition}
        An \textbf{Advanced Persistent Threat (APT)} is a sophisticated, multi-phase attack campaign conducted by well-resourced actors who maintain long-term, stealthy presence in targeted systems.
    \end{block}
    
    \begin{itemize}
        \item Technical sophistication exists on a spectrum from basic automated tools to complex custom frameworks.
        \item Low sophistication: pre-packaged exploits, phishing kits, and DDoS-for-hire services.
        \item Medium sophistication: customized malware, social engineering, and lateral movement techniques.
        \item High sophistication: zero-day exploitation, advanced evasion, supply chain compromises, and hardware implants.
        \item Sophistication level influences detection difficulty, attack attribution, and required defensive measures.
    \end{itemize}
\end{frame}

%%%% SLIDE 13 %%%%
\begin{frame}
    \frametitle{Tools of the Trade: Comparing Threat Actor Arsenals}
    
    \begin{itemize}
        \item Different threat actors employ distinctive toolsets that reflect their resources, objectives, and technical capabilities.
        \item Script kiddies primarily use publicly available exploits, automated scanners, and pre-packaged malware kits.
        \item Cybercriminal groups leverage commodity malware, phishing frameworks, and increasingly, legitimate system administration tools.
        \item Hacktivists favor DDoS tools, web defacement scripts, and data exfiltration utilities to maximize public impact.
        \item APT groups utilize custom implants, fileless malware, specialized backdoors, and sophisticated command-and-control infrastructure.
    \end{itemize}
\end{frame}

%%%% SLIDE 14 %%%%
\begin{frame}
    \frametitle{Data Theft: Who Wants Your Information and Why}
    
    \begin{alertblock}{Data Classification}
        Understanding what data different actors target helps organizations implement appropriate protections based on data classification and value.
    \end{alertblock}
    
    \begin{itemize}
        \item \textbf{Data exfiltration} involves the unauthorized transfer of data from an organization to an external location.
        \item Nation-states target intellectual property, defense information, and strategic intelligence to gain competitive advantages.
        \item Criminal groups focus on personally identifiable information (PII), payment data, and healthcare records that can be monetized.
        \item Hacktivists seek sensitive communications, controversial internal documents, and evidence of perceived wrongdoing.
        \item Insider threats often target specific high-value data based on their knowledge of where it resides and its market value.
    \end{itemize}
\end{frame}

%%%% SLIDE 15 %%%%
\begin{frame}
    \frametitle{Cyber Espionage: The Digital Spy Game}
    
    \begin{itemize}
        \item \textbf{Cyber espionage} is the act of obtaining secrets and confidential information without permission using cyber capabilities.
        \item Primary practitioners include nation-states and their proxies, though corporate espionage by competitors also occurs.
        \item Espionage operations prioritize stealth and long-term persistence over immediate impact or monetization.
        \item Key targets include government agencies, defense contractors, critical infrastructure, and companies with valuable intellectual property.
        \item Sophisticated espionage campaigns may persist for years before discovery, with attackers adapting tactics to avoid detection.
    \end{itemize}
\end{frame}

%%%% SLIDE 16 %%%%
\begin{frame}
    \frametitle{Breaking Things: Actors Focused on Service Disruption}
    
    \begin{block}{Impact Assessment}
        Service disruptions can cost organizations between \$300,000 and \$1 million per hour depending on the industry and systems affected.
    \end{block}
    
    \begin{itemize}
        \item \textbf{Service disruption} attacks aim to prevent legitimate users from accessing systems, applications, or data.
        \item Common techniques include Distributed Denial of Service (DDoS), ransomware deployment, and critical system sabotage.
        \item Hacktivists use disruption to bring attention to causes, while nation-states may target critical infrastructure for strategic advantage.
        \item Criminal groups increasingly use disruption as leverage for extortion rather than as a goal itself.
        \item The rise of Internet of Things (IoT) has created new opportunities for massive disruption attacks.
    \end{itemize}
\end{frame}

%%%% CASE STUDY 5: SUPPLY CHAIN ATTACKS %%%%
\begin{frame}
    \frametitle{Case Study: SolarWinds and Supply Chain Vulnerabilities}
    
    \begin{alertblock}{Scope}
        The attack affected approximately 18,000 organizations, including multiple US government agencies and Fortune 500 companies.
    \end{alertblock}
    
    \begin{itemize}
        \item In 2020, threat actors compromised SolarWinds' build system to inject malicious code into the Orion network monitoring product.
        \item The operation demonstrated extraordinary patience, with attackers maintaining access for months before activating backdoors.
        \item The US government attributed the attack to Russia's Foreign Intelligence Service (SVR).
        \item Victims included the US Treasury, Justice Department, and numerous technology companies.
        \item This case demonstrates how attacking trusted vendors can provide access to thousands of organizations simultaneously.
    \end{itemize}
\end{frame}

%%%% SLIDE 17 %%%%
\begin{frame}
    \frametitle{Extortion Economics: Ransomware and Digital Blackmail}
    
    \begin{itemize}
        \item \textbf{Digital extortion} involves threatening to harm or expose victims unless financial demands are met.
        \item Ransomware encrypts critical data and demands payment for decryption keys, with average demands exceeding \$200,000 in 2023.
        \item Double extortion attacks both encrypt data and threaten to publish stolen information if ransom isn't paid.
        \item Primarily conducted by criminal organizations, though some nation-states use similar tactics for financial gain.
        \item Organizations with time-sensitive operations (healthcare, manufacturing) or regulatory obligations are particularly vulnerable to extortion.
    \end{itemize}
\end{frame}

%%%% SLIDE 18 %%%%
\begin{frame}
    \frametitle{Show Me the Money: Financial Motivations in Cyberattacks}
    
    \begin{alertblock}{Evolution of Monetization}
        Cybercriminal business models have evolved from direct theft to sophisticated schemes including ransomware-as-a-service, cryptojacking, and business email compromise.
    \end{alertblock}
    
    \begin{itemize}
        \item \textbf{Financial gain} remains the primary motivation for most cybercriminal activity globally.
        \item Methods include direct theft (banking trojans), ransomware, cryptocurrency mining, payment fraud, and business email compromise.
        \item Criminal groups operate increasingly specialized marketplaces selling access, tools, and stolen data.
        \item The average cost of a data breach reached \$4.35 million in 2022, creating strong financial incentives for attackers.
        \item Financially motivated actors typically follow return-on-investment principles, targeting the easiest victims with adequate payouts.
    \end{itemize}
\end{frame}

%%%% SLIDE 19 %%%%
\begin{frame}
    \frametitle{Hacktivism and Ideology: When Beliefs Drive Cyber Operations}
    
    \begin{itemize}
        \item \textbf{Philosophical and political beliefs} motivate hacktivists and ideologically-driven threat actors.
        \item These actors view their activities as activism or civil disobedience rather than criminal behavior.
        \item Common targets include government agencies, corporations perceived as unethical, and organizations with opposing ideological views.
        \item Operations typically aim to expose perceived wrongdoing, embarrass targets, or disrupt operations to draw attention to causes.
        \item Unlike financial actors, ideological attackers may persist despite minimal practical success, driven by conviction rather than profit.
    \end{itemize}
\end{frame}

%%%% SLIDE 20 %%%%
\begin{frame}
    \frametitle{Ethics and "White Hat" Operations: Beneficial Breaches?}
    
    \begin{block}{Ethical Hacking Definition}
        \textbf{Ethical hacking} involves authorized attempts to gain unauthorized access to systems, applications, or data by simulating the actions of malicious attackers.
    \end{block}
    
    \begin{itemize}
        \item \textbf{Ethical motivations} in hacking include improving security, identifying vulnerabilities before malicious actors, and protecting users.
        \item Security researchers discover and responsibly disclose vulnerabilities through coordinated vulnerability disclosure programs.
        \item Penetration testers and red teams conduct authorized attacks to identify weaknesses in organizational defenses.
        \item Bug bounty programs provide financial incentives for ethical hackers to identify and report security issues.
        \item The line between ethical and unethical behavior can blur when disclosures are made without coordination or authorization.
    \end{itemize}
\end{frame}

%%%% SLIDE 21 %%%%
\begin{frame}
    \frametitle{Digital Revenge: When Personal Grudges Go Online}
    
    \begin{itemize}
        \item \textbf{Revenge} motivates attacks by individuals with personal grievances against organizations or individuals.
        \item Disgruntled former employees represent a significant threat due to their insider knowledge and potentially retained access.
        \item Revenge-motivated attackers often focus on causing embarrassment, reputational damage, or operational disruption.
        \item These actors may accept greater personal risk than financially motivated attackers due to emotional investment.
        \item Attacks frequently include public disclosure of sensitive information, sabotage of systems, or defacement of public-facing resources.
    \end{itemize}
\end{frame}

%%%% SLIDE 22 %%%%
\begin{frame}
    \frametitle{Chaos Agents: Disruption for Disruption's Sake}
    
    \begin{alertblock}{Detection Challenge}
        Actors motivated purely by chaos often exhibit unpredictable patterns that make their behavior difficult to model and detect through conventional means.
    \end{alertblock}
    
    \begin{itemize}
        \item Some threat actors are motivated primarily by \textbf{disruption and chaos} rather than financial gain or ideological goals.
        \item These individuals or groups derive satisfaction from causing disorder, confusion, and system failures.
        \item Their targets tend to be opportunistic rather than strategic, based on vulnerability and potential for visible impact.
        \item Techniques range from simple website defacements to complex attacks designed to trigger cascading failures.
        \item Historical examples include early hacker groups like Cult of the Dead Cow and certain Anonymous operations.
    \end{itemize}
\end{frame}

%%%% SLIDE 23 %%%%
\begin{frame}
    \frametitle{Cyberwarfare: When Nations Clash in Digital Space}
    
    \begin{itemize}
        \item \textbf{Cyberwarfare} involves state-sponsored offensive operations aimed at damaging another nation's capabilities or infrastructure.
        \item Unlike espionage, warfare operations prioritize impact over stealth and may target critical civilian infrastructure.
        \item Modern military conflicts now routinely include cyber operations alongside traditional kinetic warfare.
        \item Notable examples include Stuxnet (targeting Iranian nuclear facilities), attacks on Ukrainian power grid, and election interference operations.
        \item The absence of clear international norms and attribution challenges make cyberwarfare particularly destabilizing in international relations.
    \end{itemize}
\end{frame}

%%%% SLIDE 24 %%%%
\begin{frame}
    \frametitle{Case Studies: Notable Attacks and Their Perpetrators}
    
    \begin{block}{Learning from History}
        Analyzing past attacks provides valuable insights into threat actor TTPs, motivations, and the effectiveness of various defensive measures.
    \end{block}
    
    \begin{itemize}
        \item The 2020 SolarWinds supply chain attack demonstrated the sophisticated capabilities of nation-state actors (attributed to Russia).
        \item WannaCry ransomware in 2017 showed how criminal groups leverage stolen nation-state tools (attributed to North Korea).
        \item The 2014 Sony Pictures hack illustrated politically motivated destruction by state-sponsored actors (attributed to North Korea).
        \item The 2017 Equifax breach demonstrated how criminal organizations target and monetize massive personal data collections.
        \item Operation Aurora in 2009 revealed early advanced persistent threat tactics targeting intellectual property (attributed to China).
    \end{itemize}
\end{frame}

%%%% SLIDE 25 %%%%
\begin{frame}
    \frametitle{Attribution Challenges: Why Identifying Threat Actors Is Difficult}
    
    \begin{itemize}
        \item \textbf{Attribution} is the process of determining who is responsible for a cyberattack, often with limited and ambiguous evidence.
        \item Sophisticated actors use false flags and borrowed techniques to mislead investigators about their identity.
        \item Technical evidence (IP addresses, malware code, infrastructure) can be easily manipulated or obfuscated.
        \item Attribution requires combining technical forensics with intelligence about known actor behaviors, capabilities, and motivations.
        \item Even high-confidence attributions rarely meet the standard of proof that would be required in legal proceedings.
    \end{itemize}
\end{frame}

%%%% SLIDE 26 %%%%
\begin{frame}
    \frametitle{Threat Intelligence: Practical Applications}
    
    \begin{alertblock}{Intelligence Lifecycle}
        Effective threat intelligence follows a cycle of planning, collection, processing, analysis, dissemination, and feedback to continuously improve defenses.
    \end{alertblock}
    
    \begin{itemize}
        \item \textbf{Threat intelligence} transforms raw data about threats into actionable information for security decision-making.
        \item Strategic intelligence helps executives understand risks and allocate resources appropriately.
        \item Tactical intelligence enables security teams to proactively hunt for threats based on known actor behaviors.
        \item Operational intelligence provides context for incident responders during active breaches.
        \item Intelligence sharing occurs through formal organizations (ISACs), commercial services, and informal professional networks.
    \end{itemize}
\end{frame}

%%%% SLIDE 27 %%%%
\begin{frame}
    \frametitle{Defense Strategies: Tailoring Security to Specific Threat Actors}
    
    \begin{itemize}
        \item Understanding threat actors enables organizations to implement \textbf{threat-informed defense} rather than generic security controls.
        \item Defenses against nation-states require emphasis on critical data segregation, insider threat monitoring, and advanced detection capabilities.
        \item Protections against criminal groups focus on ransomware resilience, phishing defenses, and financial transaction safeguards.
        \item Mitigating insider threats requires privileged access management, behavior analytics, and data loss prevention tools.
        \item The MITRE ATT\&CK framework maps common techniques to threat actors, enabling targeted defensive measures.
    \end{itemize}
\end{frame}

%%%% SLIDE 28 %%%%
\begin{frame}
    \frametitle{The Changing Landscape: Emerging Threat Actors and Motivations}
    
    \begin{block}{Future Trends}
        The democratization of advanced attack capabilities through AI, automated exploitation tools, and attack services will continue to lower barriers to entry for sophisticated attacks.
    \end{block}
    
    \begin{itemize}
        \item The line between threat actor categories continues to blur as nation-states leverage criminal proxies and criminal groups adopt nation-state techniques.
        \item \textbf{Artificial intelligence} is emerging both as a tool for attackers and a new category of potential threat actor if improperly secured.
        \item The growth of IoT and operational technology (OT) networks creates new attack surfaces and potential threat actors.
        \item Attacks against cloud service providers, managed service providers, and supply chains demonstrate a shift toward targeting trusted intermediaries.
        \item Future motivations may include manipulating markets, influencing public opinion, and weaponizing information in addition to traditional objectives.
    \end{itemize}
\end{frame}

%%%% SLIDE 29 %%%%
\begin{frame}
    \frametitle{Putting It All Together: Comprehensive Threat Modeling}
    
    \begin{itemize}
        \item \textbf{Threat modeling} is a structured approach to identifying potential threats, likely attack vectors, and appropriate mitigations.
        \item Effective models incorporate knowledge of relevant threat actors, their capabilities, and their likely motivations.
        \item The process begins with identifying valuable assets and mapping potential exposure to different threat actors.
        \item Organizations should prioritize defenses against the threat actors most likely to target their particular industry and data types.
        \item Regular reassessment is critical as both organizational assets and threat actor landscapes evolve.
    \end{itemize}
\end{frame}

%%%% SLIDE 30 %%%%
\begin{frame}
    \frametitle{Beyond Technology: The Human Element in Cybersecurity}
    
    \begin{alertblock}{Final Thought}
        Understanding the human motivations, behaviors, and limitations of both threat actors and defenders is ultimately more important than any technological solution.
    \end{alertblock}
    
    \begin{itemize}
        \item Technology alone cannot address the full spectrum of cybersecurity challenges posed by diverse threat actors.
        \item Effective security culture and awareness are essential components of defense against all threat actor types.
        \item Human psychology drives both attacker motivations and defender behaviors, making it central to cybersecurity strategy.
        \item Building resiliency requires addressing people, processes, and technology as an integrated system.
        \item The future of cybersecurity lies in understanding the motivations and methods of threat actors while anticipating how these will evolve.
    \end{itemize}
\end{frame}

\end{document}