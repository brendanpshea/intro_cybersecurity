\documentclass{beamer}
\usetheme{Madrid}
\usecolortheme{dolphin}
\usepackage{graphicx}
\usepackage{hyperref}
\usepackage{listings}

\title{Concepts and Strategies to Protect Data}
\subtitle{A Comprehensive Overview}
\author{Your Name}
\institute{University/Institution Name}
\date{\today}

\begin{document}

\begin{frame}
\titlepage
\end{frame}

\begin{frame}
\frametitle{Introduction: The Critical World of Data Protection}
\begin{itemize}
\item Data protection is the process of safeguarding important information from corruption, compromise, or loss.
\item The volume of data created daily has increased exponentially, reaching over 2.5 quintillion bytes per day.
\item Organizations face increasing responsibility to protect sensitive information from unauthorized access.
\item Data breaches can result in significant financial losses, legal consequences, and damage to reputation.
\end{itemize}

\begin{alertblock}{Why This Matters}
According to recent studies, the average cost of a data breach is \$4.45 million, with sensitive data breaches costing significantly more.
\end{alertblock}
\end{frame}

\begin{frame}
\frametitle{Why Data Protection Matters in Today's Digital Landscape}
\begin{itemize}
\item Digital transformation has led to unprecedented amounts of data being stored, processed, and transmitted.
\item \textbf{Data breach}: Unauthorized access to data that results in data being viewed, stolen, or exposed to unauthorized parties.
\item Modern cyber threats are becoming more sophisticated, requiring robust protection strategies.
\item Regulatory requirements (such as GDPR, HIPAA, and CCPA) mandate specific data protection approaches.
\end{itemize}

\begin{exampleblock}{Example}
In 2017, the Equifax breach exposed sensitive personal information of 147 million people, demonstrating how critical proper data protection is for organizations handling sensitive data.
\end{exampleblock}
\end{frame}

\begin{frame}
\frametitle{Understanding Data Types: An Overview}
\begin{itemize}
\item Different types of data require different levels of protection based on sensitivity and regulatory requirements.
\item \textbf{Data type classification} is the categorization of data based on its content, purpose, and regulatory status.
\item Organizations must create comprehensive data inventories to identify what types of data they process.
\item Understanding data types is the foundation for implementing appropriate protection measures.
\end{itemize}

\begin{table}
\begin{tabular}{|l|l|}
\hline
\textbf{Data Type Category} & \textbf{Examples} \\
\hline
Regulated & Healthcare records, financial transactions \\
\hline
Trade Secret & Proprietary formulas, manufacturing processes \\
\hline
Intellectual Property & Patents, trademarks, copyrighted material \\
\hline
Personal Information & Names, addresses, social security numbers \\
\hline
\end{tabular}
\end{table}
\end{frame}

\begin{frame}
\frametitle{Regulated Data: Requirements and Compliance}
\begin{itemize}
\item \textbf{Regulated data} refers to information that must be protected according to laws, regulations, or industry standards.
\item Organizations handling regulated data must comply with specific requirements for storage, processing, and transmission.
\item Non-compliance with data regulations can result in severe penalties, fines, and legal consequences.
\item Examples include HIPAA for healthcare data, PCI DSS for payment card information, and FERPA for educational records.
\end{itemize}

\begin{block}{Key Compliance Requirements}
\begin{itemize}
\item Written policies and procedures
\item Regular risk assessments
\item Implementation of security controls
\item Staff training and awareness
\end{itemize}
\end{block}
\end{frame}

\begin{frame}
\frametitle{Trade Secrets: Protecting Competitive Advantage}
\begin{itemize}
\item \textbf{Trade secrets} are proprietary information that provides a company with a competitive advantage in the marketplace.
\item Unlike patents, trade secrets do not expire as long as they remain confidential and provide competitive value.
\item Trade secrets can include formulas, processes, designs, patterns, customer lists, and business strategies.
\item Protection requires both legal mechanisms and practical security measures to maintain confidentiality.
\end{itemize}

\begin{exampleblock}{Famous Trade Secret Example}
The Coca-Cola formula has been successfully protected as a trade secret for over 130 years, demonstrating how effective trade secret protection can create lasting business value.
\end{exampleblock}
\end{frame}

\begin{frame}
\frametitle{Intellectual Property: Safeguarding Innovation and Creation}
\begin{itemize}
\item \textbf{Intellectual property (IP)} refers to creations of the mind that are legally protected through patents, copyrights, and trademarks.
\item IP protection grants creators exclusive rights to use, reproduce, and profit from their creations for a specific period.
\item Digital transformation has made IP more vulnerable to theft, unauthorized copying, and distribution.
\item Organizations must implement technical and administrative controls to protect their intellectual property assets.
\end{itemize}

\begin{table}
\begin{tabular}{|l|l|l|}
\hline
\textbf{IP Type} & \textbf{Protects} & \textbf{Duration} \\
\hline
Patent & Inventions, processes & 20 years \\
\hline
Copyright & Creative works & Life + 70 years \\
\hline
Trademark & Logos, brand identifiers & Renewable indefinitely \\
\hline
\end{tabular}
\end{table}
\end{frame}

\begin{frame}
\frametitle{Legal and Financial Information: Special Protection Needs}
\begin{itemize}
\item \textbf{Legal information} includes contracts, litigation documents, and attorney-client communications that require confidentiality.
\item \textbf{Financial information} encompasses accounting records, tax documents, payroll data, and investment details.
\item Both types of information often contain sensitive details about individuals and organizations that could be exploited if exposed.
\item Specialized protection measures are necessary due to the high value and sensitivity of this information.
\end{itemize}

\begin{alertblock}{Legal and Financial Data Risks}
Improper handling of legal and financial information can lead to regulatory violations, breach of fiduciary duty, insider trading allegations, and loss of client/investor trust.
\end{alertblock}
\end{frame}

\begin{frame}
\frametitle{Human vs. Non-Human Readable Data: Key Differences}
\begin{itemize}
\item \textbf{Human-readable data} is information that can be directly understood by people without additional processing or translation.
\item \textbf{Non-human readable data} requires specialized tools, software, or algorithms to interpret and understand.
\item Human-readable data (like text documents) is more vulnerable to unauthorized access and visual exposure.
\item Non-human readable data (like compiled code or encrypted files) provides inherent security through obscurity but requires management of access tools.
\end{itemize}

\begin{columns}
\column{0.5\textwidth}
\textbf{Human-Readable Examples:}
\begin{itemize}
\item Text documents
\item Spreadsheets
\item Email messages
\item Source code
\end{itemize}

\column{0.5\textwidth}
\textbf{Non-Human Readable Examples:}
\begin{itemize}
\item Compiled applications
\item Encrypted files
\item Machine code
\item Binary data
\end{itemize}
\end{columns}
\end{frame}

\begin{frame}
\frametitle{Data Classification Systems: Purpose and Implementation}
\begin{itemize}
\item \textbf{Data classification} is the process of categorizing data based on sensitivity levels and protection requirements.
\item A formal classification system provides clear guidelines for handling different types of information.
\item Classification systems help prioritize security resources and apply appropriate controls based on data sensitivity.
\item Effective classification requires policies, procedures, training, and technological support for implementation.
\end{itemize}

\begin{block}{Benefits of Data Classification}
\begin{itemize}
\item Focuses security resources where most needed
\item Simplifies compliance with regulations
\item Reduces risk of data breaches
\item Improves data management efficiency
\end{itemize}
\end{block}
\end{frame}

\begin{frame}
    \frametitle{Example: Data Classification in Action}
    \begin{itemize}
        \item Peter's healthcare company classified patient records into four categories: Public, Internal, Confidential, and Restricted.
        \item Public data: Clinic locations, hours of operation, general health pamphlets.
        \item Confidential data: Patient diagnoses, treatment plans, medication lists.
        \item Restricted data: HIV status, mental health records, genetic testing results.
    \end{itemize}
    
    \begin{exampleblock}{Classification Impact}
    After implementing this system, Peter's company experienced 70\% fewer data incidents. When a breach occurred, its impact was limited because restricted data was isolated on systems with enhanced security controls.
    \end{exampleblock}
    \end{frame}

\begin{frame}
\frametitle{Sensitive Data: Identification and Handling Procedures}
\begin{itemize}
\item \textbf{Sensitive data} contains information that could potentially harm individuals or organizations if improperly disclosed.
\item Identification of sensitive data involves content analysis, context evaluation, and regulatory consideration.
\item Handling procedures for sensitive data typically include access controls, encryption, audit logging, and special disposal methods.
\item Organizations should document and enforce clear policies regarding who can access, modify, and share sensitive information.
\end{itemize}

\begin{exampleblock}{Common Types of Sensitive Data}
Personal identifiable information (PII), protected health information (PHI), payment card data, authentication credentials, and business strategic plans are all examples of sensitive data requiring special protection.
\end{exampleblock}
\end{frame}

\begin{frame}
\frametitle{Confidential Data: Access Control and Management}
\begin{itemize}
\item \textbf{Confidential data} is information intended for limited distribution within specific authorized groups.
\item Access to confidential data should follow the principle of least privilege, granting only necessary permissions.
\item Management of confidential information requires clear identification through labeling, watermarking, or metadata tagging.
\item Confidentiality agreements and non-disclosure provisions are commonly used to legally protect this category of data.
\end{itemize}

\begin{alertblock}{Confidentiality Breach Consequences}
Breaches of confidential information can result in competitive disadvantage, damaged relationships with clients and partners, regulatory penalties, and potential legal action for negligence in handling sensitive information.
\end{alertblock}
\end{frame}

\begin{frame}
\frametitle{Public Data: Availability Without Compromising Security}
\begin{itemize}
\item \textbf{Public data} is information that can be freely shared without restriction or negative impact on individuals or organizations.
\item Despite being openly available, public data still requires integrity protection to prevent unauthorized modification.
\item Organizations should have formal approval processes before classifying or releasing data as public.
\item Even with public data, metadata may contain sensitive information that should be removed before publication.
\end{itemize}

\begin{block}{Examples of Public Data}
\begin{itemize}
\item Product catalogs and marketing materials
\item Press releases and public statements
\item Published research findings
\item General organizational information
\end{itemize}
\end{block}
\end{frame}

\begin{frame}
\frametitle{Restricted Data: Balancing Access and Protection}
\begin{itemize}
\item \textbf{Restricted data} is highly sensitive information with access limited to specific individuals or roles.
\item This classification often applies to information that could cause significant harm if compromised.
\item Access to restricted data typically requires multiple levels of authorization and authentication.
\item Organizations should implement comprehensive audit trails for all access and modifications to restricted data.
\end{itemize}

\begin{exampleblock}{Restricted Data Protection Methods}
Restricted data often requires enterprise-grade encryption, physical access controls, secure storage systems, and regular security audits to ensure appropriate protection levels are maintained.
\end{exampleblock}
\end{frame}

\begin{frame}
\frametitle{Private Data: Individual Rights and Organizational Responsibilities}
\begin{itemize}
\item \textbf{Private data} is personal information about individuals that requires protection from unauthorized access.
\item Privacy regulations like GDPR and CCPA establish specific rights for individuals regarding their private data.
\item Organizations collecting private data have responsibilities for transparency, consent, access, correction, and deletion.
\item Protection strategies must address both technical security and compliance with privacy principles.
\end{itemize}

\begin{table}
\begin{tabular}{|l|l|}
\hline
\textbf{Individual Rights} & \textbf{Organizational Responsibilities} \\
\hline
Right to be informed & Privacy notices and transparency \\
\hline
Right to access & Providing data upon request \\
\hline
Right to rectification & Correcting inaccurate information \\
\hline
Right to erasure & Deleting data when no longer needed \\
\hline
\end{tabular}
\end{table}
\end{frame}

\begin{frame}
\frametitle{Critical Data: Highest Level Protection Strategies}
\begin{itemize}
\item \textbf{Critical data} is information essential to operations that would cause severe damage if compromised or unavailable.
\item Organizations should maintain inventories of critical data assets and their locations.
\item Protection strategies for critical data must address confidentiality, integrity, and availability requirements.
\item Business continuity and disaster recovery planning must prioritize critical data protection.
\end{itemize}

\begin{alertblock}{Critical Data Considerations}
Critical data often requires defense-in-depth protection strategies, including encryption, access controls, physical security, regular backups, and redundant systems to ensure both security and availability.
\end{alertblock}
\end{frame}
\begin{frame}
\frametitle{Understanding Data States: A Comprehensive View}
\begin{itemize}
\item \textbf{Data states} refer to the different conditions in which data exists throughout its lifecycle.
\item Each state presents unique security challenges and requires different protection approaches.
\item Protection strategies must address vulnerabilities specific to each data state.
\item A comprehensive data protection strategy must secure data in all possible states.
\end{itemize}

\begin{block}{The Three Primary Data States}
\begin{itemize}
\item Data at rest: Stored in databases, file systems, or archives
\item Data in transit: Moving between systems or networks
\item Data in use: Being processed, viewed, or modified by applications
\end{itemize}
\end{block}
\end{frame}

\begin{frame}
\frametitle{Data at Rest: Storage Protection Strategies}
\begin{itemize}
\item \textbf{Data at rest} refers to information stored in databases, file systems, storage arrays, or backup media.
\item At-rest data is vulnerable to unauthorized access, theft of storage media, and administrative privilege abuse.
\item Encryption is a primary protection method for data at rest, rendering it unreadable without proper decryption keys.
\item Access controls, secure storage locations, and media sanitization procedures are also essential protection components.
\end{itemize}

\begin{exampleblock}{Data at Rest Protection Methods}
Full-disk encryption, database encryption, file-level encryption, and secure key management systems are commonly implemented to protect data at rest from unauthorized access.
\end{exampleblock}
\end{frame}

\begin{frame}
\frametitle{Data in Transit: Securing Information on the Move}
\begin{itemize}
\item \textbf{Data in transit} is information traveling across networks between systems, applications, or users.
\item Transit data is vulnerable to interception, eavesdropping, man-in-the-middle attacks, and routing attacks.
\item Transport encryption protocols like TLS/SSL create secure tunnels for data transmission.
\item Virtual Private Networks (VPNs) provide additional protection for data moving across public networks.
\end{itemize}

\begin{alertblock}{Common Vulnerabilities}
Unencrypted communications, weak encryption protocols, improper certificate validation, and insecure wireless transmissions are leading causes of data-in-transit breaches.
\end{alertblock}
\end{frame}

\begin{frame}
\frametitle{Data in Use: Real-time Protection Challenges}
\begin{itemize}
\item \textbf{Data in use} refers to information actively being processed, viewed, or modified by applications or users.
\item This state presents unique challenges as data must be decrypted and accessible to be useful.
\item Protection methods focus on secure memory management, application security, and user authentication.
\item Memory encryption, secure enclaves, and trusted execution environments are emerging technologies for protecting data in use.
\end{itemize}

\begin{table}
\begin{tabular}{|l|l|}
\hline
\textbf{Protection Challenge} & \textbf{Potential Solution} \\
\hline
Memory scraping & Memory encryption \\
\hline
Screen capture & Privacy screens, watermarking \\
\hline
Keylogging & Secure input methods \\
\hline
Privilege escalation & Application sandboxing \\
\hline
\end{tabular}
\end{table}
\end{frame}

\begin{frame}
\frametitle{Data Sovereignty: Navigating International Regulations}
\begin{itemize}
\item \textbf{Data sovereignty} refers to the concept that data is subject to the laws and governance of the country in which it is located.
\item Different countries have varying requirements for data storage, processing, and transfer across borders.
\item Organizations operating globally must understand and comply with multiple, sometimes conflicting regulatory frameworks.
\item Non-compliance with data sovereignty laws can result in significant legal penalties and operational restrictions.
\end{itemize}

\begin{block}{Key Data Sovereignty Regulations}
\begin{itemize}
    \item EU: General Data Protection Regulation (GDPR)
    \item Russia: Data Localization Law
    \item China: Cybersecurity Law
    \item Brazil: General Data Protection Law (LGPD)
\end{itemize}
\end{block}
\end{frame}


\begin{frame}
    \frametitle{Example: Data Sovereignty Challenge}
    \begin{itemize}
        \item Scenario: Barbara's cloud software company serves customers across Europe, Asia, and North America.
        \item Challenge: Customer in Germany requires all their data to remain within EU borders to comply with GDPR.
        \item Technical solution: Barbara configured EU-specific storage buckets with geographic restriction rules.
        \item Contract solution: Service Level Agreement specifying all data storage locations and cross-border transfer restrictions.
    \end{itemize}
    
    \begin{alertblock}{Implementation Outcome}
    When audited by EU regulators, Barbara's company demonstrated complete data isolation for EU customers. The company avoided potential GDPR penalties of up to 4\% of annual global turnover or €20 million.
    \end{alertblock}
    \end{frame}

\begin{frame}
\frametitle{Geolocation Considerations in Data Protection}
\begin{itemize}
\item \textbf{Geolocation} refers to the identification of the real-world geographic location of data storage and processing.
\item Data location affects which laws and regulations apply to the information.
\item Cloud computing and distributed systems have complicated geolocation tracking and compliance.
\item Organizations must implement systems to track, document, and control where their data resides.
\end{itemize}

\begin{alertblock}{Compliance Challenges}
Without proper geolocation tracking, organizations risk unknown compliance violations, data transfer restrictions, and potential regulatory penalties from jurisdictions they didn't realize their data was subject to.
\end{alertblock}
\end{frame}

\begin{frame}
\frametitle{Geographic Restrictions: Implementation and Challenges}
\begin{itemize}
\item \textbf{Geographic restrictions} are controls that limit where data can be stored, processed, or accessed from.
\item Implementation requires both technical controls (geo-fencing, IP filtering) and contractual agreements.
\item Cloud service providers now offer region-specific data centers to help with geographic compliance.
\item Challenges include ensuring continuous compliance as systems evolve and data moves throughout its lifecycle.
\end{itemize}

\begin{exampleblock}{Implementation Example}
A multinational corporation might configure its cloud storage to ensure that EU citizen data remains within EU-based data centers, while implementing technical controls to prevent unauthorized transfers to non-compliant regions.
\end{exampleblock}
\end{frame}

\begin{frame}
\frametitle{Encryption: Principles, Types, and Applications}
\begin{itemize}
\item \textbf{Encryption} is the process of converting readable data into a coded format that can only be decoded with the proper key.
\item Symmetric encryption uses the same key for encryption and decryption, while asymmetric encryption uses different public and private keys.
\item The strength of encryption depends on the algorithm used and the key length.
\item Encryption can be applied at different levels: file, disk, database, application, or communication channel.
\end{itemize}

\scriptsize

\begin{table}
\begin{tabular}{|l|l|l|}
\hline
\textbf{Encryption Type} & \textbf{Key Management} & \textbf{Common Uses} \\
\hline
Symmetric & Single shared key & File encryption, disk encryption \\
\hline
Asymmetric & Public/private key pair & Digital signatures, secure communication \\
\hline
End-to-end & Keys only at endpoints & Messaging, email \\
\hline
Homomorphic & Computation on encrypted data & Privacy-preserving analytics \\
\hline
\end{tabular}
\end{table}
\end{frame}

\begin{frame}
\frametitle{Hashing: One-way Protection for Critical Data}
\begin{itemize}
    \item \textbf{Hashing} is a one-way function that converts data of any size into a fixed-length string of characters.
    \item Unlike encryption, hashing is not reversible – the original data cannot be retrieved from the hash value.
    \item Hashing is primarily used for data integrity verification, password storage, and digital signatures.
    \item Modern secure hashing algorithms include SHA-256, SHA-3, and specialized password hashing functions like bcrypt.
\end{itemize}

\begin{block}{Important Hashing Properties}
    \begin{itemize}
        \item Deterministic: Same input always produces same hash
        \item Fast computation for any input size
        \item Infeasible to generate original data from hash
        \item Small change in input creates dramatically different hash
    \end{itemize}
\end{block}
\end{frame}


\begin{frame}
    \frametitle{Example: Hashing for Password Protection}
    \begin{itemize}
        \item Natasha's security team implemented proper password hashing for a user authentication system.
        \item Original approach stored passwords using simple MD5 hashing (insecure and vulnerable).
        \item New implementation uses bcrypt with salting to protect against rainbow table attacks.
        \item Each user's password has a unique salt value to prevent identical passwords from having identical hashes.
    \end{itemize}
    
    \begin{exampleblock}{Concrete Example}
    Password: "Avengers2023!", Salt: "randomSaltValue"
    
    \smallskip
    MD5 hash (insecure): \\ 
    5f4dcc3b5aa765d61d8327deb882cf99. No salt stored.
    
    \smallskip
    bcrypt hash with salt (secure): \\
    \$2a\$10\$N9qo8uLOickgx2ZMRZoMyeIjZAgcfl7p92ldGxad68LJZdL17lhWy. Salt stored with hash.
    \end{exampleblock}
    \end{frame}
    


\begin{frame}
\frametitle{Data Masking: Concealing Sensitive Information}
\begin{itemize}
    \item \textbf{Data masking} is the process of hiding original data with modified content while maintaining the data's format and usability.
    \item Masking methods include character substitution, shuffling, encryption, and redaction.
    \item Common applications include development environments, training systems, and data shared with external parties.
    \item Unlike encryption, masked data remains functional for testing and analysis without exposing sensitive information.
\end{itemize}

\begin{exampleblock}{Masking Example}
A credit card number 4532-7891-2345-6789 might be masked as XXXX-XXXX-XXXX-6789, preserving the last four digits for verification purposes while protecting the majority of the sensitive information.
\end{exampleblock}
\end{frame}

\begin{frame}
    \frametitle{Example: Data Masking in Development}
    \begin{itemize}
        \item Diana's team needed to test new healthcare software features using realistic data.
        \item Using actual patient data would violate privacy regulations and expose sensitive information.
        \item Solution: Implemented data masking to transform production data while preserving its format and relationships.
        \item Developers could test with realistic data without access to actual patient information.
    \end{itemize}
    
    \begin{table}
    \begin{tabular}{|l|l|}
    \hline
    \textbf{Original Data} & \textbf{Masked Data} \\
    \hline
    Name: Clark Kent & Name: XXXX XXXX \\
    \hline
    SSN: 123-45-6789 & SSN: XXX-XX-6789 \\
    \hline
    DOB: 05/28/1985 & DOB: 05/XX/1985 \\
    \hline
    Diagnosis: Hypertension & Diagnosis: [CONDITION] \\
    \hline
    Medication: Lisinopril 10mg & Medication: [MEDICATION] [DOSE] \\
    \hline
    \end{tabular}
    \end{table}
    \end{frame}

\begin{frame}
\frametitle{Tokenization: Secure Data Representation}
\begin{itemize}
    \item \textbf{Tokenization} replaces sensitive data with non-sensitive placeholder values that reference the original data stored securely.
    \item Tokens maintain the format and sometimes partial content of the original data for usability.
    \item Unlike encryption, tokens have no mathematical relationship to the original data and cannot be reversed.
    \item Tokenization is particularly valuable for payment card processing and other highly regulated data types.
\end{itemize}

\begin{alertblock}{Tokenization vs. Encryption}
Tokenization differs from encryption in that there is no algorithm or key that can convert the token back to the original value – the relationship exists only in a separately secured lookup table.
\end{alertblock}
\end{frame}


\begin{frame}
    \frametitle{Example: Tokenization vs. Encryption}
    Both tokenization and encryption are used to protect sensitive data, but they do so in different ways. 
    Here’s a comparison of how each method would handle the same data:
    \vspace{.5cm}

    \scriptsize
    \begin{tabular}{|p{3.5cm}|p{3.5cm}|p{3.5cm}|}
    \hline
    \textbf{Original Data} & \textbf{Encrypted} & \textbf{Tokenized} \\
    \hline
    Credit Card: \\4532-7891-2345-6789 & A7F9R0... (cipher text that can be decrypted with key) & XXXX-XXXX-XXXX-6789 (token with no mathematical relation) \\
    \hline
    Customer Address: \\123 Hero Lane, Gotham & B8D2E5... (reversible with decryption key) & 74-***-** (irreversible token with lookup table) \\
    \hline
    Purchase History: \\Items, dates, amounts & Encrypted database with authorized access & Anonymized tokens for analytics while protecting identity \\
    \hline
    \end{tabular}
    
    \end{frame}


\begin{frame}
\frametitle{Obfuscation: Making Data Difficult to Understand}
\begin{itemize}
    \item \textbf{Obfuscation} is the practice of deliberately making information unclear, ambiguous, or difficult to interpret.
    \item In software protection, obfuscation transforms code to make it harder to reverse-engineer while preserving functionality.
    \item Data obfuscation techniques include scrambling, code substitution, and adding misleading elements.
    \item Obfuscation provides security through complexity but is generally considered a supplementary protection method.
\end{itemize}

\begin{table}
\begin{tabular}{|l|l|}
\hline
\textbf{Obfuscation Technique} & \textbf{Application} \\
\hline
Code obfuscation & Protect intellectual property in software \\
\hline
Format preserving & Maintain data structure while hiding content \\
\hline
Data scrambling & Reorder or randomize parts of data \\
\hline
Noise addition & Add irrelevant information to confuse analysis \\
\hline
\end{tabular}
\end{table}
\end{frame}

\begin{frame}
    \frametitle{Data Segmentation: Dividing to Protect}
    \begin{itemize}
        \item \textbf{Data segmentation} is the process of dividing information into distinct parts that can be separately protected.
        \item Segmentation limits the impact of breaches by preventing access to the complete data set.
        \item Implementation can be physical (separate systems), logical (different databases), or virtual (access controls).
        \item Effective segmentation requires clear policies on data classification and handling across segments.
    \end{itemize}
    
    \begin{exampleblock}{Segmentation in Healthcare}
    Healthcare organizations often segment patient data by department, with stricter access controls for mental health, substance abuse, and genetic information compared to general medical records.
    \end{exampleblock}
    \end{frame}
    
    \begin{frame}
    \frametitle{Permission Restrictions: Role-Based Access Control}
    \begin{itemize}
        \item \textbf{Permission restrictions} limit who can access, modify, or share specific data based on their role or identity.
        \item \textbf{Role-Based Access Control (RBAC)} assigns permissions to job functions rather than individuals.
        \item The principle of least privilege dictates that users should have only the minimum access necessary for their tasks.
        \item Regular access reviews and privilege audits are essential for maintaining appropriate permission restrictions.
    \end{itemize}
    
    \begin{block}{RBAC Implementation Steps}
        \begin{enumerate}
            \item Identify and classify data by sensitivity
            \item Define roles based on job functions
            \item Establish permissions for each role
            \item Assign individuals to appropriate roles
            \item Review and update regularly
        \end{enumerate}
    \end{block}
    \end{frame}


    
\begin{frame}
    \frametitle{Example: Role-Based Access Control Implementation}
    \begin{itemize}
        \item Tony's manufacturing company implemented RBAC to protect intellectual property and operational data.
        \item System defines roles rather than individual permissions: Engineer, Manager, HR, Finance, Contractor.
        \item Example: Process formulas are accessible to Engineers but not Contractors.
        \item Regular permission audits found and remediated 23 instances of excessive access.
    \end{itemize}
    
    \scriptsize
    \begin{table}
    \begin{tabular}{|l|c|c|c|c|}
    \hline
    \textbf{Data Type} & \textbf{Engineer} & \textbf{Manager} & \textbf{HR} & \textbf{Contractor} \\
    \hline
    Product Designs & Full & Read & None & Limited \\
    \hline
    Process Formulas & Full & Read & None & None \\
    \hline
    Employee Records & None & Limited & Full & None \\
    \hline
    Financial Data & None & Read & Limited & None \\
    \hline
    \end{tabular}
    \end{table}
    \end{frame}
    
    \begin{frame}
    \frametitle{Comparing Protection Methods: Strengths and Weaknesses}
    \begin{itemize}
        \item Each data protection method has specific strengths, weaknesses, and appropriate use cases.
        \item The most effective protection strategies combine multiple methods in a defense-in-depth approach.
        \item Method selection should consider data type, sensitivity, regulatory requirements, and operational needs.
        \item Cost, performance impact, and usability must be balanced against security requirements.
    \end{itemize}
    
    \small
    \begin{table}
    \begin{tabular}{|l|l|l|}
    \hline
    \textbf{Method} & \textbf{Strengths} & \textbf{Limitations} \\
    \hline
    Encryption & Strong mathematical protection & Key management complexity \\
    \hline
    Tokenization & No mathematical relationship & Requires secure token vault \\
    \hline
    Masking & Simplicity and performance & Not suitable for all data types \\
    \hline
    Segmentation & Limits breach impact & Operational complexity \\
    \hline
    \end{tabular}
    \end{table}
    \end{frame}
    
    \begin{frame}
    \frametitle{Conclusion: The Future of Data Protection in a Connected World}
    \begin{itemize}
        \item Data protection requirements will continue to evolve with technological advancements and regulatory changes.
        \item Emerging technologies like homomorphic encryption and quantum-resistant algorithms will reshape protection strategies.
        \item Successful data protection requires ongoing education, vigilance, and adaptability.
        \item Organizations must balance security, compliance, usability, and business objectives in their data protection frameworks.
    \end{itemize}
    
    \begin{alertblock}{Key Takeaway}
    The most effective data protection approach is comprehensive, layered, and adaptable—combining appropriate technical controls, administrative procedures, and user awareness to safeguard information throughout its lifecycle.
    \end{alertblock}
    \end{frame}
    
    \end{document}

\end{document}