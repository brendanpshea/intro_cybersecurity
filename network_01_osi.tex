\documentclass{beamer}
\usetheme{Madrid}
\usecolortheme{whale}

\usepackage{graphicx}
\usepackage{amsmath}
\usepackage{hyperref}

\title{Introduction to Computer Networks}
\subtitle{Understanding Networks \& The OSI Model}
\author{Your Name}
\institute{Institution Name}
\date{\today}

\begin{document}

\begin{frame}
    \titlepage
\end{frame}

\begin{frame}
    \frametitle{What is a Computer Network? Understanding the Basics}
    
    \begin{itemize}
        \item A \textbf{computer network} is a collection of interconnected devices that can communicate and share resources with each other.
        
        \item Networks enable the sharing of resources such as files, printers, and internet connections between connected devices.
        
        \item \begin{alertblock}{Key Concept}
            Networks operate using standardized \textbf{protocols}, which are sets of rules that govern how devices communicate.
        \end{alertblock}
        
        \item Modern networks range from simple home setups connecting a few devices to complex enterprise systems connecting thousands of computers.
    \end{itemize}
\end{frame}

\begin{frame}
    \frametitle{Networks in Our Daily Lives: From Home to Enterprise}
    
    \begin{block}{Common Network Applications}
        \begin{itemize}
            \item Home Networks:
            \begin{itemize}
                \item Wi-Fi connections for smartphones, laptops, and smart devices
                \item Shared printers and media servers
                \item Smart home automation systems
            \end{itemize}
            
            \item Enterprise Networks:
            \begin{itemize}
                \item Shared database access and file storage
                \item Email and communication systems
                \item Customer management systems
            \end{itemize}
        \end{itemize}
    \end{block}
    
    Every time you browse the internet, stream videos, or send messages, you're utilizing multiple computer networks.
\end{frame}

\begin{frame}
    \frametitle{Types of Networks: LAN, WAN, MAN, and PAN}
    
    \begin{itemize}
        \item A \textbf{Local Area Network (LAN)} connects devices within a limited area like a home, school, or office building.
        
        \item A \textbf{Wide Area Network (WAN)} spans a large geographic area, with the Internet being the largest example.
        
        \item A \textbf{Metropolitan Area Network (MAN)} covers a city or large campus, bridging the gap between LANs and WANs.
        
        \item A \textbf{Personal Area Network (PAN)} connects devices within a very short range, such as Bluetooth connections between your phone and headphones.
        
        \begin{table}
            \begin{tabular}{|l|l|}
                \hline
                Network Type & Typical Range \\
                \hline
                PAN & 1-10 meters \\
                LAN & Up to 1 kilometer \\
                MAN & Up to 50 kilometers \\
                WAN & Worldwide \\
                \hline
            \end{tabular}
        \end{table}
    \end{itemize}
\end{frame}

\begin{frame}
    \frametitle{Network Topologies: Star, Bus, Ring, and Mesh}
    
    \begin{itemize}
        \item A \textbf{network topology} defines the physical or logical arrangement of devices and connections in a network.
        
        \item The \textbf{star topology} connects all devices to a central hub or switch, making it easy to manage but vulnerable to central point failure.
        
        \item The \textbf{bus topology} connects all devices to a single central cable, creating a simple but outdated design that was common in early Ethernet networks.
        
        \begin{block}{Advanced Topologies}
            \item The \textbf{mesh topology} connects devices with multiple paths, providing redundancy and fault tolerance.
            
            \item The \textbf{ring topology} connects each device to exactly two other devices, forming a circular path for data transmission.
        \end{block}
    \end{itemize}
\end{frame}

\begin{frame}
    \frametitle{Client-Server vs Peer-to-Peer Networks}
    
    \begin{columns}[t]
        \begin{column}{0.5\textwidth}
            \textbf{Client-Server Network:}
            \begin{itemize}
                \item Dedicated servers provide resources and services to client computers.
                
                \item Examples include email servers, web servers, and file servers.
                
                \item Offers centralized control and better security but requires more maintenance.
            \end{itemize}
        \end{column}
        
        \begin{column}{0.5\textwidth}
            \textbf{Peer-to-Peer Network:}
            \begin{itemize}
                \item Each computer can act as both client and server, sharing resources directly.
                
                \item Examples include BitTorrent file sharing and early versions of Napster.
                
                \item Provides simpler setup but less security and harder to manage at scale.
            \end{itemize}
        \end{column}
    \end{columns}
\end{frame}

\begin{frame}
    \frametitle{Introduction to the OSI Model: Why It Matters}
    
    \begin{alertblock}{Definition}
        The \textbf{OSI (Open Systems Interconnection) Model} is a conceptual framework that standardizes the functions of a telecommunication or computing system into seven distinct layers.
    \end{alertblock}
    
    \begin{itemize}
        \item The OSI model helps network professionals understand, troubleshoot, and communicate about network operations.
        
        \item Each layer in the model performs specific functions and provides services to the layer above it.
        
        \item Understanding the OSI model is crucial for effective network design, maintenance, and troubleshooting.
        
        \item The model creates a common language for network professionals across different platforms and technologies.
    \end{itemize}
\end{frame}

\begin{frame}
    \frametitle{The OSI Model: A Seven-Layer Journey}
    
    \begin{block}{Layer Structure (Bottom to Top)}
        \begin{enumerate}
            \item \textbf{Physical Layer:} Handles raw bit transmission
            \item \textbf{Data Link Layer:} Provides node-to-node delivery
            \item \textbf{Network Layer:} Manages addressing and routing
            \item \textbf{Transport Layer:} Ensures end-to-end delivery
            \item \textbf{Session Layer:} Manages connections between applications
            \item \textbf{Presentation Layer:} Formats and encrypts data
            \item \textbf{Application Layer:} Provides network services to applications
        \end{enumerate}
    \end{block}
    
    \begin{alertblock}{Key Concept}
        Data flows down the layers when sending and up the layers when receiving, with each layer adding or removing its own control information.
    \end{alertblock}
\end{frame}

\begin{frame}
    \frametitle{Layer 1 - Physical Layer: The Foundation}
    
    \begin{block}{Physical Layer Responsibilities}
        \begin{itemize}
            \item The \textbf{Physical Layer} is responsible for transmitting raw bits over a physical medium like copper wire, fiber optic cable, or radio waves.
            
            \item This layer defines physical characteristics such as voltage levels, timing of voltage changes, physical data rates, and maximum transmission distances.
            
            \item It specifies the shape and layout of pins in network interfaces, as well as the functions of each pin.
            
            \item The Physical Layer converts digital bits into signals that can be transmitted over the network media.
        \end{itemize}
    \end{block}
\end{frame}

\begin{frame}
    \frametitle{Physical Layer Components: Cables, Hubs, and Signals}
    
    \begin{columns}[t]
        \begin{column}{0.6\textwidth}
            \textbf{Common Physical Media:}
            \begin{itemize}
                \item \textbf{Twisted Pair Cable:} Used in Ethernet networks, comes in shielded (STP) and unshielded (UTP) varieties.
                
                \item \textbf{Fiber Optic Cable:} Uses light for transmission, offering high speeds and immunity to electromagnetic interference.
                
                \item \textbf{Wireless Media:} Uses radio frequencies to transmit data through the air.
            \end{itemize}
        \end{column}
        
        \begin{column}{0.4\textwidth}
            \begin{alertblock}{Key Devices}
                \begin{itemize}
                    \item Hubs
                    \item Repeaters
                    \item Network adapters
                    \item Cable connectors
                \end{itemize}
            \end{alertblock}
        \end{column}
    \end{columns}
\end{frame}

\begin{frame}
    \frametitle{Layer 2 - Data Link Layer: Bridging the Gap}
    
    \begin{itemize}
        \item The \textbf{Data Link Layer} provides reliable point-to-point delivery of data frames between directly connected nodes.
        
        \item This layer detects and possibly corrects errors that may occur in the Physical Layer.
        
        \begin{block}{Primary Functions}
            \begin{itemize}
                \item \textbf{Framing:} Organizes bits from Physical Layer into manageable data units called frames.
                
                \item \textbf{Physical Addressing:} Adds MAC addresses to identify source and destination devices.
                
                \item \textbf{Error Control:} Detects and retransmits corrupted or lost frames.
                
                \item \textbf{Flow Control:} Prevents overwhelming slower receiving devices.
            \end{itemize}
        \end{block}
    \end{itemize}
\end{frame}

\begin{frame}
    \frametitle{MAC Addresses and Ethernet: Data Link Essentials}
    
    \begin{alertblock}{What is a MAC Address?}
        A \textbf{Media Access Control (MAC) address} is a unique 48-bit identifier assigned to network interfaces for communications at the Data Link Layer.
    \end{alertblock}
    
    \begin{itemize}
        \item MAC addresses are written as six pairs of hexadecimal digits (e.g., 00:1A:2B:3C:4D:5E).
        
        \item Every network interface card (NIC) has a unique MAC address burned into it during manufacturing.
        
        \item \textbf{Ethernet} is the most common Data Link Layer protocol, providing rules for:
        \begin{itemize}
            \item Cable types and connections
            \item Data packet format
            \item Protocol for sharing cable capacity
        \end{itemize}
    \end{itemize}
\end{frame}

\begin{frame}
    \frametitle{Layer 3 - Network Layer: Finding the Path}
    
    \begin{alertblock}{Network Layer Purpose}
        The \textbf{Network Layer} is responsible for packet forwarding and routing between different networks, enabling data to travel across multiple networks to reach its final destination.
    \end{alertblock}
    
    \begin{itemize}
        \item This layer handles logical addressing (IP addresses) and determines the best path for data to travel.
        
        \item \textbf{Routers} operate at the Network Layer, making decisions about how to forward packets based on logical addresses.
        
        \item The Network Layer must handle congestion and ensure quality of service (QoS) for different types of data.
        
        \item It manages the connection of heterogeneous networks, allowing different types of networks to communicate.
    \end{itemize}
\end{frame}

\begin{frame}
    \frametitle{IP Addressing and Routing: Network Layer Deep Dive}
    
    \begin{columns}[t]
        \begin{column}{0.5\textwidth}
            \textbf{IPv4 Addressing:}
            \begin{itemize}
                \item 32-bit addresses written in four octets (e.g., 192.168.1.1)
                \item Divided into network and host portions
                \item Supports about 4.3 billion unique addresses
            \end{itemize}
        \end{column}
        
        \begin{column}{0.5\textwidth}
            \textbf{IPv6 Addressing:}
            \begin{itemize}
                \item 128-bit addresses written in hexadecimal
                \item Provides vastly more unique addresses
                \item Designed to replace IPv4 as addresses run out
            \end{itemize}
        \end{column}
    \end{columns}
    
    \begin{block}{Routing Concepts}
        \begin{itemize}
            \item Routers maintain routing tables to determine the best path for packets
            \item Path selection can be static (manually configured) or dynamic (automatically updated)
        \end{itemize}
    \end{block}
\end{frame}

\begin{frame}
    \frametitle{Layer 4 - Transport Layer: End-to-End Communication}
    
    \begin{itemize}
        \item The \textbf{Transport Layer} ensures complete data transfer by providing:
        \begin{itemize}
            \item End-to-end error recovery
            \item Flow control
            \item Segmentation of data
        \end{itemize}
        
        \item This layer can establish multiple connections for different applications on the same device.
        
        \item It provides either connection-oriented (\textbf{TCP}) or connectionless (\textbf{UDP}) communication.
        
        \begin{alertblock}{Key Concept}
            The Transport Layer is the first layer to provide end-to-end communication between source and destination hosts.
        \end{alertblock}
    \end{itemize}
\end{frame}

\begin{frame}
    \frametitle{TCP vs UDP: Understanding Transport Protocols}
    
    \begin{columns}[t]
        \begin{column}{0.5\textwidth}
            \textbf{Transmission Control Protocol (TCP):}
            \begin{itemize}
                \item Provides reliable, ordered delivery of data
                \item Establishes connections before sending data
                \item Includes error checking and recovery
                \item Used for email, web browsing, file transfer
            \end{itemize}
        \end{column}
        
        \begin{column}{0.5\textwidth}
            \textbf{User Datagram Protocol (UDP):}
            \begin{itemize}
                \item Offers fast, connectionless delivery
                \item No guarantee of delivery or ordering
                \item Lower overhead than TCP
                \item Used for streaming, gaming, DNS queries
            \end{itemize}
        \end{column}
    \end{columns}
    
    \begin{block}{When to Use Each}
        Choose TCP when reliability is crucial, and UDP when speed is more important than guaranteed delivery.
    \end{block}
\end{frame}

\begin{frame}
    \frametitle{Layer 5 - Session Layer: Managing Connections}
    
    \begin{alertblock}{Primary Role}
        The \textbf{Session Layer} establishes, manages, and terminates connections (sessions) between applications on different devices.
    \end{alertblock}
    
    \begin{itemize}
        \item This layer handles the organization of communication through features like dialog control and synchronization.
        
        \item It provides three different modes of communication:
        \begin{itemize}
            \item \textbf{Simplex:} One-way communication
            \item \textbf{Half-duplex:} Two-way communication, one direction at a time
            \item \textbf{Full-duplex:} Simultaneous two-way communication
        \end{itemize}
        
        \item The Session Layer can establish checkpoints for long data transfers, enabling recovery from failures without starting over.
    \end{itemize}
\end{frame}

\begin{frame}
    \frametitle{Layer 6 - Presentation Layer: Data Translation}
    
    \begin{itemize}
        \item The \textbf{Presentation Layer} ensures that data is readable by the receiving system through:
        \begin{itemize}
            \item Character code translation (e.g., ASCII to EBCDIC)
            \item Data compression to reduce size
            \item Data encryption for security
            \item Data formatting for different systems
        \end{itemize}
        
        \begin{alertblock}{Important Note}
            This layer acts as the "translator" of the network, converting data between different formats while maintaining its meaning.
        \end{alertblock}
        
        \item Common data formats handled include JPEG, MIDI, MPEG, and ASCII.
        
        \item The Presentation Layer enables different systems to communicate regardless of their internal data representations.
    \end{itemize}
\end{frame}

\begin{frame}
    \frametitle{Encryption and Data Formats in the Presentation Layer}
    
    \begin{block}{Data Security Functions}
        \begin{itemize}
            \item \textbf{Encryption Protocols:}
            \begin{itemize}
                \item SSL/TLS for secure web browsing
                \item SSH for secure remote access
                \item PGP for secure email communication
            \end{itemize}
        \end{itemize}
    \end{block}
    
    \begin{itemize}
        \item The Presentation Layer handles data compression using various algorithms:
        \begin{itemize}
            \item Lossless compression for critical data
            \item Lossy compression for multimedia content
        \end{itemize}
        
        \item Common data format conversions include:
        \begin{itemize}
            \item Text encodings (ASCII, Unicode, UTF-8)
            \item Image formats (JPEG, PNG, GIF)
            \item Audio/Video formats (MP3, MP4, AVI)
        \end{itemize}
    \end{itemize}
\end{frame}

\begin{frame}
    \frametitle{Layer 7 - Application Layer: User Interface}
    
    \begin{alertblock}{Definition}
        The \textbf{Application Layer} is the topmost layer of the OSI model, providing network services directly to end-user applications.
    \end{alertblock}
    
    \begin{itemize}
        \item This layer enables users and applications to access network services through:
        \begin{itemize}
            \item Network resource identification and synchronization
            \item Partner identification and quality of service
            \item User authentication and privacy considerations
        \end{itemize}
        
        \item The Application Layer handles user interface and support for services like email, file transfer, and web browsing.
        
        \item It determines resource availability and synchronizes communication between applications.
    \end{itemize}
\end{frame}

\begin{frame}
    \frametitle{Common Application Layer Protocols: HTTP, FTP, SMTP}
    
    \begin{columns}[t]
        \begin{column}{0.5\textwidth}
            \textbf{Web and File Transfer:}
            \begin{itemize}
                \item \textbf{HTTP/HTTPS:} Web browsing and secure transactions
                \item \textbf{FTP:} File Transfer Protocol for uploading and downloading files 
                \item \textbf{DNS:} Domain Name System for translating domain names
            \end{itemize}
        \end{column}
        
        \begin{column}{0.5\textwidth}
            \textbf{Email and Communication:}
            \begin{itemize}
                \item \textbf{SMTP:} Sending email messages
                \item \textbf{POP3/IMAP:} Receiving email messages
                \item \textbf{DHCP:} Automatic IP address assignment
            \end{itemize}
        \end{column}
    \end{columns}
    
    \begin{block}{Protocol Functions}
        Each protocol provides specific services:
        \begin{itemize}
            \item Data formatting and encoding
            \item Session management
            \item Error reporting
        \end{itemize}
    \end{block}
\end{frame}

\begin{frame}
    \frametitle{Data Flow Through the OSI Layers: The Big Picture}
    
    \begin{block}{Data Encapsulation (Sending)}
        \begin{enumerate}
            \item Application Layer creates user data
            \item Presentation Layer formats and encrypts
            \item Session Layer adds session control
            \item Transport Layer segments data and adds port numbers
            \item Network Layer adds IP addresses
            \item Data Link Layer adds MAC addresses
            \item Physical Layer converts to bits for transmission
        \end{enumerate}
    \end{block}
    
    \begin{alertblock}{Key Concept}
        Each layer adds its own header information to the data, a process called encapsulation. The receiving device reverses this process through de-encapsulation.
    \end{alertblock}
\end{frame}

\begin{frame}
    \frametitle{Encapsulation and De-encapsulation in OSI Model}
    
    \begin{itemize}
        \item \textbf{Protocol Data Units (PDUs)} have different names at each layer:
        \begin{itemize}
            \item Application Layer: Data
            \item Transport Layer: Segments
            \item Network Layer: Packets
            \item Data Link Layer: Frames
            \item Physical Layer: Bits
        \end{itemize}
        
        \item Each layer adds control information:
        \begin{itemize}
            \item Headers added at the beginning
            \item Trailers added at the end (in some layers)
            \item Original data remains unchanged
        \end{itemize}
        
        \begin{alertblock}{Important Note}
            The receiving device removes headers in reverse order, ensuring data integrity throughout the process.
        \end{alertblock}
    \end{itemize}
\end{frame}

\begin{frame}
    \frametitle{Practical Example: Web Browsing Through OSI Layers}
    
    \begin{block}{When you type www.example.com in your browser:}
        \begin{enumerate}
            \item \textbf{Application Layer:} HTTP request is generated
            \item \textbf{Presentation Layer:} Data is formatted and possibly encrypted (HTTPS)
            \item \textbf{Session Layer:} TCP session is established
            \item \textbf{Transport Layer:} Data is segmented, TCP ports assigned
            \item \textbf{Network Layer:} IP addresses added after DNS lookup
            \item \textbf{Data Link Layer:} Frame created with MAC addresses
            \item \textbf{Physical Layer:} Converted to bits and transmitted
        \end{enumerate}
    \end{block}
    
    \begin{alertblock}{Think About}
        How would this process differ for streaming video vs. sending an email?
    \end{alertblock}
\end{frame}

\begin{frame}
    \frametitle{Troubleshooting Scenario: Network Problems}
    
    \begin{columns}[t]
        \begin{column}{0.5\textwidth}
            \textbf{Symptom:} Cannot access website
            \begin{itemize}
                \item \textbf{Physical Layer:} Check cables connected?
                \item \textbf{Data Link:} Network adapter working?
                \item \textbf{Network:} IP address valid?
                \item \textbf{Transport:} Ports blocked?
            \end{itemize}
        \end{column}
        
        \begin{column}{0.5\textwidth}
            \textbf{Common Tools:}
            \begin{itemize}
                \item ping (Network Layer)
                \item ipconfig (Network Layer)
                \item tracert (Network Layer)
                \item nslookup (Application Layer)
            \end{itemize}
        \end{column}
    \end{columns}
    
    \begin{block}{Discussion}
        What layer would you check first if a specific application fails but others work?
    \end{block}
\end{frame}

\begin{frame}
    \frametitle{Review: Key Concepts and Their Relationships}
    
    \begin{itemize}
        \item \textbf{Data Flow Understanding:}
        \begin{itemize}
            \item How does encapsulation protect data integrity?
            \item Why do we need different PDUs at different layers?
            \item How do upper layers depend on lower layers?
        \end{itemize}
        
        \item \textbf{Protocol Relationships:}
        \begin{itemize}
            \item How do TCP and IP work together?
            \item Why do we need both MAC and IP addresses?
            \item How do application protocols use transport protocols?
        \end{itemize}
    \end{itemize}
    
    \begin{alertblock}{Critical Thinking}
        Consider how changes in one layer might affect the others. For example, what happens when upgrading from IPv4 to IPv6?
    \end{alertblock}
\end{frame}

\begin{frame}
    \frametitle{Discussion Questions and Activities}
    
    \begin{block}{Group Discussion Topics}
        \begin{itemize}
            \item Compare and contrast different network topologies for a small business network.
            
            \item Explain how video conferencing applications use different layers of the OSI model.
            
            \item Analyze the security implications of using different protocols at each layer.
            
            \item Design a basic network setup for a home office, considering all OSI layers.
        \end{itemize}
    \end{block}
    
    \begin{alertblock}{Hands-On Activities}
        \begin{itemize}
            \item Use Wireshark to capture and analyze network traffic through OSI layers.
            \item Configure a basic home network and identify components at each layer.
            \item Practice using common networking tools and relate them to OSI layers.
        \end{itemize}
    \end{alertblock}
\end{frame}

\end{document}