\documentclass{beamer}
\usetheme{Madrid}
\usecolortheme{whale}
\usepackage{graphicx}
\usepackage{amsmath}
\usepackage{hyperref}
\usepackage{listings}
\usepackage{xcolor}

\title{Understanding Security Vulnerabilities}
\author{Brendan Shea, PhD}
\date{\today}

\begin{document}

\begin{frame}
    \titlepage
\end{frame}

%%%% SLIDE 1 %%%%
\begin{frame}
    \frametitle{Understanding Vulnerabilities: Types and Classification}
    
    \begin{itemize}
        \item A \textbf{security vulnerability} is a weakness that can be exploited to cause harm or gain unauthorized access to a system.
        \item Think of vulnerabilities as unlocked doors or windows in your home—they create entry points for those with malicious intent.
        \item Vulnerabilities can exist in any part of an IT system: hardware, software, networks, or even human procedures.
        \item Organizations track vulnerabilities using the \textbf{Common Vulnerabilities and Exposures (CVE)} system, which assigns unique identifiers to each known issue.
        \item Understanding the types of vulnerabilities helps prioritize which "doors" need to be locked first to keep systems secure.
    \end{itemize}
\end{frame}

%%%% SLIDE 2 %%%%
\begin{frame}
    \frametitle{The Vulnerability Lifecycle: Discovery to Remediation}
    
    \begin{alertblock}{Critical Timeline}
        The time between discovering a vulnerability and fixing it represents a period of significant risk.
    \end{alertblock}
    
    \begin{itemize}
        \item Vulnerabilities follow a predictable lifecycle from birth to resolution, similar to how a disease spreads and is eventually treated.
        \item First, someone \textbf{discovers} the vulnerability—this could be a security researcher, a hacker, or an accidental finding.
        \item Next, the vulnerability is \textbf{disclosed}, ideally to the software/hardware creator before being publicly announced.
        \item The vendor then works to develop and release a \textbf{patch or fix} for the problem.
        \item Finally, organizations must \textbf{implement the fix} and verify it works properly to complete the cycle.
    \end{itemize}
\end{frame}

%%%% SLIDE 3 %%%%
\begin{frame}
    \frametitle{Application Vulnerabilities: Overview and Attack Surface}
    
    \begin{itemize}
        \item \textbf{Application vulnerabilities} are weaknesses in software that allow attackers to manipulate how programs operate.
        \item These are among the most common security issues because software is complex and constantly changing.
        \item Think of applications like a house with many rooms—each feature adds another space that needs to be secured.
        \item Common application weaknesses include improper handling of user input, memory management mistakes, and authentication problems.
        \item Modern applications rely on hundreds of components from different sources, with each component potentially introducing new vulnerabilities.
    \end{itemize}
\end{frame}

%%%% SLIDE 4 %%%%
\begin{frame}
    \frametitle{Memory Injection: Concepts and Attack Techniques}
    
    \begin{block}{Core Concept}
        \textbf{Memory injection} happens when an attacker can insert their own code into a running program's memory space, like slipping a fake page into a book.
    \end{block}
    
    \begin{itemize}
        \item Computer programs store instructions and data in memory while they're running.
        \item Memory injection attacks manipulate this memory to make the program do something it wasn't designed to do.
        \item This is similar to changing the script in a play while actors are performing—suddenly they're following the attacker's directions.
        \item These attacks are powerful because they execute with the same permissions as the original program.
        \item Memory injection often bypasses traditional security tools that focus on examining files rather than memory.
    \end{itemize}
\end{frame}

%%%% SLIDE 5 %%%%
\begin{frame}
    \frametitle{Buffer Overflow Vulnerabilities: Understanding the Basics}
    
    \begin{itemize}
        \item A \textbf{buffer overflow} happens when a program tries to store more data in a temporary storage area than it was designed to hold.
        \item Imagine trying to pour a gallon of water into a pint glass—the excess has to go somewhere and will spill over.
        \item In computers, this "spill" can overwrite adjacent memory locations with the attacker's malicious code.
        \item When executed, this malicious code can crash systems, corrupt files, or give attackers complete control of the computer.
        \item Buffer overflows have been around for decades but remain dangerous because they're difficult to detect during software development.
    \end{itemize}
\end{frame}

%%%% SLIDE 6 %%%%
\begin{frame}
    \frametitle{Race Conditions: When Timing Creates Vulnerability}
    
    \begin{alertblock}{Hidden Danger}
        Race conditions are particularly tricky because they often don't appear during testing but emerge under real-world conditions.
    \end{alertblock}
    
    \begin{itemize}
        \item \textbf{Race conditions} occur when a system's behavior depends on the sequence or timing of events that can't be controlled.
        \item Think of two people trying to withdraw the last \$100 from a shared bank account at exactly the same time.
        \item If the system checks the balance for both people before processing either withdrawal, both might be approved even though there's only enough money for one.
        \item In computer security, race conditions can let attackers access files during brief windows when permission checks aren't properly synchronized with operations.
        \item These vulnerabilities are called \textbf{Time-of-Check to Time-of-Use (TOCTOU)} problems because they exploit the gap between checking a condition and acting on it.
    \end{itemize}
\end{frame}

%%%% SLIDE 7 %%%%
\begin{frame}
    \frametitle{Case Study: Darth Vader's Memory Corruption Imperial Project}
    
    \begin{itemize}
        \item Darth Vader discovers the Rebel Alliance's main communication system has a critical buffer overflow vulnerability.
        \item The system doesn't properly check the length of incoming message headers, allowing more data to be stored than the allocated space.
        \item Imperial technicians craft special messages that look normal but contain hidden code in the oversized header.
        \item When processed by Rebel systems, this extra code overwrites critical memory and gives the Empire backdoor access.
        \item The Rebels don't notice the intrusion because their systems continue to function normally while secretly reporting to the Empire.
    \end{itemize}
\end{frame}

%%%% SLIDE 8 %%%%
\begin{frame}
    \frametitle{Malicious Update Vulnerabilities: Subverting Trust}
    
    \begin{block}{Trojan Horse}
        Malicious updates are modern-day Trojan horses—threats disguised as helpful improvements.
    \end{block}
    
    \begin{itemize}
        \item Software updates are designed to improve programs, fix bugs, and patch security holes.
        \item \textbf{Malicious updates} corrupt this trusted process by disguising harmful code as legitimate updates.
        \item These attacks are effective because users and systems are conditioned to trust and automatically install updates.
        \item Update mechanisms typically run with high system privileges, giving attackers powerful access if compromised.
        \item The SolarWinds attack of 2020 demonstrated how a single compromised update can affect thousands of organizations simultaneously.
    \end{itemize}
\end{frame}

%%%% SLIDE 9 %%%%
\begin{frame}
    \frametitle{Operating System Update Vulnerabilities}
    
    \begin{itemize}
        \item Operating systems need regular updates to fix security issues, but the update process itself can be vulnerable.
        \item OS updates require special privileges to replace system files, making them attractive targets for attackers.
        \item If update verification is flawed, attackers can trick the system into installing fake updates containing malware.
        \item During updates, security features are sometimes temporarily disabled, creating a window of vulnerability.
        \item Organizations face a dilemma: delay updates and remain vulnerable to known issues, or update quickly and risk update-related problems.
    \end{itemize}
\end{frame}

%%%% SLIDE 10 %%%%
\begin{frame}
    \frametitle{Third-Party Software Update Exploitation}
    
    \begin{alertblock}{Security Challenge}
        The average computer has dozens of applications, each with its own update mechanism and security practices.
    \end{alertblock}
    
    \begin{itemize}
        \item While operating systems often have well-designed update systems, third-party applications may not be as secure.
        \item Many applications check for updates using unencrypted connections, making it easier for attackers to intercept and modify them.
        \item Some applications run their update processes with administrative privileges even though the application itself doesn't need them.
        \item Auto-update features, while convenient, can become security risks if they don't properly verify the source of updates.
        \item Popular applications are particularly attractive targets since compromising their update process affects millions of users.
    \end{itemize}
\end{frame}

%%%% SLIDE 11 %%%%
\begin{frame}
    \frametitle{Case Study: Loki's Malicious Update Scheme}
    
    \begin{itemize}
        \item Marvel's trickster god Loki targets S.H.I.E.L.D. by compromising their security software update server.
        \item Rather than attacking S.H.I.E.L.D. directly, Loki focuses on their software vendor—a smaller company with fewer security resources.
        \item He creates a perfect replica of their legitimate security update but adds a hidden backdoor that activates after installation.
        \item S.H.I.E.L.D. agents automatically approve the update because it comes from their trusted vendor and has proper digital signatures.
        \item Once installed throughout S.H.I.E.L.D., Loki's backdoor gives him access to classified information about the Avengers Initiative.
    \end{itemize}
\end{frame}

%%%% SLIDE 12 %%%%
\begin{frame}
    \frametitle{Web Application Security: Attack Surface and Common Flaws}
    
    \begin{block}{Exposure}
        Web applications are like storefronts open to the entire internet—accessible to both legitimate users and attackers worldwide.
    \end{block}
    
    \begin{itemize}
        \item \textbf{Web applications} are programs that users access through web browsers rather than installing them on their computers.
        \item These applications are particularly vulnerable because they're designed to accept input from any internet user.
        \item Common weaknesses include failing to properly validate user input, authentication problems, and insecure data handling.
        \item Web applications often connect to databases containing sensitive information, creating additional security concerns.
        \item Unlike traditional software, web applications can be updated instantly, sometimes leading to rushed deployment with inadequate security testing.
    \end{itemize}
\end{frame}

%%%% SLIDE 13 %%%%
\begin{frame}
    \frametitle{SQL Injection: When Databases Trust Bad Input}
    
    \begin{itemize}
        \item \textbf{SQL injection} happens when attackers insert database commands into what should be ordinary user input.
        \item Imagine a bank teller who follows any instruction written on a withdrawal slip—an attacker could write "And give me all the money in the vault" on the slip.
        \item Similarly, a vulnerable website might take user input like "John Smith" and directly include it in database commands.
        \item An attacker could enter something like "John Smith; DELETE ALL RECORDS" and the system would execute both actions.
        \item SQL injection can allow attackers to read sensitive data, modify database contents, or even delete entire databases.
    \end{itemize}
\end{frame}

%%%% SLIDE 14 %%%%
\begin{frame}
    \frametitle{Cross-Site Scripting (XSS): Injecting Malicious Scripts}
    
    \begin{alertblock}{User Trust}
        XSS attacks are particularly effective because the malicious code appears to come from a website the victim already trusts.
    \end{alertblock}
    
    \begin{itemize}
        \item \textbf{Cross-site scripting (XSS)} allows attackers to inject malicious code into websites that is then executed in visitors' browsers.
        \item It's like someone placing a hidden camera in a trusted friend's house—the danger comes from a source you wouldn't suspect.
        \item When users visit the compromised website, their browsers run the malicious script, thinking it's a legitimate part of the site.
        \item These scripts can steal cookies containing login information, capture keystrokes, or redirect users to fake websites.
        \item XSS vulnerabilities occur when websites display user input without properly filtering out potentially dangerous code.
    \end{itemize}
\end{frame}

%%%% SLIDE 15 %%%%
\begin{frame}
    \frametitle{Case Study: The Borg's Database Assimilation}
    
    \begin{itemize}
        \item The Borg Collective discovers Starfleet's personnel database has a SQL injection vulnerability in its search function.
        \item The search page directly incorporates user input into database queries without proper filtering or parameterization.
        \item The Borg craft a special search term: "Enterprise' OR '1'='1" which tricks the database into returning all personnel records.
        \item By refining their injection techniques, they extract officer locations, security clearances, and starship deployment schedules.
        \item Starfleet's database is "assimilated" without any need to breach physical security or hack through firewalls.
    \end{itemize}
\end{frame}

%%%% SLIDE 16 %%%%
\begin{frame}
    \frametitle{Hardware Vulnerabilities: Physical Security Weaknesses}
    
    \begin{block}{Fundamental Problem}
        Hardware vulnerabilities are especially concerning because they can't usually be fixed with a simple software update.
    \end{block}
    
    \begin{itemize}
        \item \textbf{Hardware vulnerabilities} are weaknesses in the physical components of computing systems.
        \item Unlike software bugs that can be patched remotely, hardware flaws often require physical replacement or modification.
        \item These vulnerabilities can exist in processors, memory chips, networking equipment, or peripheral devices.
        \item Some hardware vulnerabilities are introduced accidentally during design, while others might be deliberately inserted as backdoors.
        \item Even seemingly minor hardware issues can completely undermine the security of otherwise well-protected systems.
    \end{itemize}
\end{frame}

%%%% SLIDE 17 %%%%
\begin{frame}
    \frametitle{Firmware Security: The Software Within Hardware}
    
    \begin{itemize}
        \item \textbf{Firmware} is specialized software embedded within hardware components that controls how they function.
        \item Think of firmware as the conductor of an orchestra, directing how physical components work together.
        \item Firmware exists in almost everything electronic—from your computer's motherboard to smart refrigerators and network routers.
        \item When firmware is compromised, attackers gain control at a fundamental level below the operating system.
        \item A firmware attack might persist even if you reinstall your operating system or replace your hard drive.
    \end{itemize}
\end{frame}

%%%% SLIDE 18 %%%%
\begin{frame}
    \frametitle{End-of-Life and Legacy Hardware Risks}
    
    \begin{alertblock}{Hidden Dangers}
        Nearly 70\% of organizations rely on at least some outdated systems that no longer receive security updates.
    \end{alertblock}
    
    \begin{itemize}
        \item \textbf{End-of-life hardware} refers to devices that manufacturers no longer support with updates or security patches.
        \item These systems are like old cars with known safety defects that can't be fixed because replacement parts aren't made anymore.
        \item Vulnerabilities discovered after support ends remain permanently unpatched, creating perpetual security risks.
        \item Organizations often keep legacy systems running due to compatibility requirements or the high cost of replacement.
        \item Critical infrastructure like power plants, hospitals, and factories frequently rely on decades-old control systems.
    \end{itemize}
\end{frame}

%%%% SLIDE 19 %%%%
\begin{frame}
    \frametitle{Case Study: Bowser's Hardware Backdoor}
    
    \begin{itemize}
        \item Bowser realizes he can't defeat the Mushroom Kingdom through direct attacks, so he targets their networking hardware instead.
        \item He bribes an employee at the factory that manufactures network switches for the kingdom's infrastructure.
        \item The compromised worker modifies the firmware on switches destined for Princess Peach's castle.
        \item This backdoored firmware appears normal but secretly monitors all network traffic and can be remotely controlled by Bowser.
        \item When installed, the switches pass all security tests because the backdoor remains dormant until Bowser activates it with a special signal.
    \end{itemize}
\end{frame}

%%%% SLIDE 20 %%%%
\begin{frame}
    \frametitle{Virtualization Security: Risks in Virtual Environments}
    
    \begin{block}{Shared Resources}
        Virtualization is like several families living in separate apartments within the same building—isolation is critical.
    \end{block}
    
    \begin{itemize}
        \item \textbf{Virtualization} creates multiple virtual computers that run independently on a single physical machine.
        \item This technology offers many benefits but introduces unique security challenges.
        \item The core security principle of virtualization is isolation—keeping virtual machines separated from each other.
        \item If this isolation fails, an attacker who compromises one virtual system might be able to access others on the same host.
        \item The hypervisor (the software that creates and manages virtual machines) becomes a critical security component that must be protected.
    \end{itemize}
\end{frame}

%%%% SLIDE 21 %%%%
\begin{frame}
    \frametitle{Virtual Machine Escape: Breaking Out of the Sandbox}
    
    \begin{itemize}
        \item \textbf{VM escape} refers to an attack that breaks out of a virtual machine to access the host system or other VMs.
        \item It's like a prisoner escaping from their cell and gaining access to the entire prison facility.
        \item Virtual machines are designed to be isolated environments, with strict boundaries enforced by the hypervisor.
        \item When these boundaries fail, attackers can potentially gain control over all virtual machines running on the same hardware.
        \item VM escape vulnerabilities are particularly dangerous in cloud environments where your virtual machines might share physical hardware with unknown tenants.
    \end{itemize}
\end{frame}

%%%% SLIDE 22 %%%%
\begin{frame}
    \frametitle{Case Study: The Master's Virtual Escape}
    
    \begin{alertblock}{Containment Failure}
        In Doctor Who, The Master is often imprisoned but always finds a creative way to escape—just like virtual machine escape attacks.
    \end{alertblock}
    
    \begin{itemize}
        \item Doctor Who's nemesis, The Master, finds himself imprisoned in a virtual machine by UNIT's cybersecurity team.
        \item He analyzes his environment and discovers the hypervisor has a vulnerability in how it handles memory allocation.
        \item The Master creates a program that deliberately triggers memory errors, causing the hypervisor to crash.
        \item During the brief moment when the hypervisor restarts, there's a gap in the isolation mechanisms.
        \item The Master exploits this gap to break out of his virtual prison and gain access to the host system controlling all of UNIT's security.
    \end{itemize}
\end{frame}

%%%% SLIDE 23 %%%%
\begin{frame}
    \frametitle{Cloud Infrastructure Vulnerabilities: Challenges in the Cloud}
    
    \begin{itemize}
        \item \textbf{Cloud computing} shifts computing resources from local hardware to remote data centers accessed over the internet.
        \item This model introduces new security challenges because you're sharing infrastructure with other customers.
        \item Misconfigured cloud resources are a leading cause of data breaches—settings that unintentionally expose data to the public.
        \item Access management becomes more complex in the cloud, with permissions spread across multiple services and interfaces.
        \item Many organizations mistakenly assume cloud providers handle all security concerns, when responsibility is actually shared.
    \end{itemize}
\end{frame}

%%%% SLIDE 24 %%%%
\begin{frame}
    \frametitle{Shared Responsibility in Cloud Security}
    
    \begin{block}{Clear Boundaries}
        Cloud security works like apartment living—the building owner secures the structure, but you must lock your own door.
    \end{block}
    
    \begin{itemize}
        \item The \textbf{shared responsibility model} divides security duties between cloud providers and their customers.
        \item Cloud providers typically secure the underlying infrastructure—the hardware, networks, and facilities.
        \item Customers remain responsible for securing their data, applications, user access, and configurations.
        \item This division of responsibility varies by service type: IaaS (more customer responsibility) vs. SaaS (more provider responsibility).
        \item Security failures often occur when organizations don't clearly understand where their responsibilities begin and end.
    \end{itemize}
\end{frame}

%%%% SLIDE 25 %%%%
\begin{frame}
    \frametitle{Case Study: The First Order's Cloud City Takeover}
    
    \begin{itemize}
        \item The First Order discovers Cloud City's systems are hosted on a multi-tenant cloud platform with inadequate isolation.
        \item Their cyber warfare team identifies a fundamental flaw in the cloud platform's memory management system.
        \item By creating specialized workloads on their own cloud instances, they can extract tiny fragments of data from adjacent tenants.
        \item These data fragments include encryption keys and authentication tokens from Cloud City's security systems.
        \item With these credentials, the First Order gains administrative access to Cloud City's infrastructure without ever directly attacking their systems.
    \end{itemize}
\end{frame}

%%%% SLIDE 26 %%%%
\begin{frame}
    \frametitle{Supply Chain Security: The Chain of Trust}
    
    \begin{alertblock}{Expanding Risk}
        Modern software typically contains hundreds of components from different sources, each representing a potential security risk.
    \end{alertblock}
    
    \begin{itemize}
        \item The \textbf{supply chain} refers to all the components, vendors, and service providers involved in creating and delivering technology.
        \item Think of it like a restaurant meal—if any ingredient is contaminated, the whole dish becomes unsafe.
        \item Supply chain attacks target the weakest links in this chain rather than attacking the final target directly.
        \item These attacks are effective because organizations inherently trust their suppliers and the components they provide.
        \item As systems become more interconnected, a single compromise in the supply chain can affect thousands of downstream organizations.
    \end{itemize}
\end{frame}

%%%% SLIDE 27 %%%%
\begin{frame}
    \frametitle{Service, Hardware, and Software Provider Risks}
    
    \begin{itemize}
        \item Organizations rely on various external providers who have privileged access to their systems and data.
        \item \textbf{Managed service providers (MSPs)} often have administrative access to client networks, making them valuable targets.
        \item \textbf{Hardware manufacturers} might unknowingly incorporate compromised components or deliberately insert backdoors.
        \item \textbf{Software vendors} can distribute malware through otherwise legitimate programs if their development environment is compromised.
        \item The security of your organization ultimately depends on the security practices of every provider in your supply chain.
    \end{itemize}
\end{frame}

%%%% SLIDE 28 %%%%
\begin{frame}
    \frametitle{Case Study: The Mayor's Supply Chain Attack}
    
    \begin{block}{Indirect Approach}
        In Buffy the Vampire Slayer, the Mayor maintains a respectable public image while secretly working toward sinister goals.
    \end{block}
    
    \begin{itemize}
        \item The Mayor of Sunnydale wants access to school records and security systems but can't risk direct involvement.
        \item Instead of attacking the school directly, he identifies that the school district uses software from a small local company.
        \item The Mayor places one of his loyal followers as an employee at this software company.
        \item This insider adds hidden functionality to the authentication module used across all the company's educational software.
        \item When Sunnydale High installs the next routine update, the backdoor gives the Mayor access to student records and school security systems.
    \end{itemize}
\end{frame}

%%%% SLIDE 29 %%%%
\begin{frame}
    \frametitle{Cryptographic Vulnerabilities: When Protection Fails}
    
    \begin{itemize}
        \item \textbf{Cryptography} is the practice of securing information using mathematical techniques to encrypt and decrypt data.
        \item It's like having an unbreakable lock—except sometimes the lock isn't as unbreakable as we think.
        \item Cryptographic vulnerabilities can exist in the algorithms themselves or in how they're implemented in systems.
        \item Even mathematically sound encryption can be undermined by poor key generation, improper implementation, or side channels.
        \item As computing power increases and quantum computing develops, encryption that's secure today may become vulnerable tomorrow.
    \end{itemize}
\end{frame}

%%%% SLIDE 30 %%%%
\begin{frame}
    \frametitle{Configuration Vulnerabilities: The Human Factor}
    
    \begin{alertblock}{Common Problem}
        Security misconfigurations are among the most common and easily exploitable vulnerabilities in IT systems.
    \end{alertblock}
    
    \begin{itemize}
        \item \textbf{Misconfigurations} occur when systems are set up incorrectly, leaving security gaps that attackers can exploit.
        \item It's like building a fortress but forgetting to close the main gate—all other defenses become irrelevant.
        \item Default settings in software and hardware often prioritize ease-of-use over security.
        \item Common misconfigurations include default passwords, unnecessary services, excessive permissions, and disabled security features.
        \item Unlike complex technical vulnerabilities, misconfigurations are often simple to fix once identified.
    \end{itemize}
\end{frame}

%%%% SLIDE 31 %%%%
\begin{frame}
    \frametitle{Mobile Device Vulnerabilities: Security on the Go}
    
    \begin{itemize}
        \item Mobile devices present unique security challenges because they combine powerful computing with constant connectivity and physical mobility.
        \item \textbf{Sideloading} allows users to install apps from outside official app stores, bypassing security screening processes.
        \item \textbf{Jailbreaking} or \textbf{rooting} removes the built-in security restrictions of mobile operating systems.
        \item These modifications can provide users with more control but also remove critical security protections.
        \item Mobile devices are also vulnerable to physical threats like theft, unauthorized access via bluetooth, and connectivity to untrusted networks.
    \end{itemize}
\end{frame}

%%%% SLIDE 32 %%%%
\begin{frame}
    \frametitle{Zero-Day Vulnerabilities: The Unknown Threats}
    
    \begin{block}{Definition}
        A \textbf{zero-day vulnerability} is a security flaw that is unknown to the software creator and has no available patch.
    \end{block}
    
    \begin{itemize}
        \item Zero-day vulnerabilities are called this because defenders have "zero days" to prepare before the vulnerability is exploited.
        \item These are particularly dangerous because traditional security measures rely on knowing what to defend against.
        \item Think of them as undiscovered secret passages into a fortress—you can't guard what you don't know exists.
        \item Zero-days are highly valued by attackers, with some selling for hundreds of thousands of dollars on underground markets.
        \item Defending against zero-days requires security that doesn't just look for known threats but can detect suspicious behavior.
    \end{itemize}
\end{frame}

%%%% SLIDE 33 %%%%
\begin{frame}
    \frametitle{Case Study: Thanos and the Zero-Day Infinity Stones}
    
    \begin{itemize}
        \item In his quest for power, Thanos searches for six "Infinity Stone" zero-day vulnerabilities affecting the Avengers' security systems.
        \item The Mind Stone vulnerability allows him to bypass authentication systems without triggering any alerts or logging.
        \item The Space Stone exploit lets him move between isolated network segments that should be completely separated.
        \item The Reality Stone creates false readings in security monitoring tools, making defenders see normal activity when attacks are happening.
        \item The Power Stone targets infrastructure systems like power and climate control, creating physical effects from digital attacks.
        \item With all six zero-days combined, Thanos can execute his plan with a single coordinated attack that bypasses all security layers.
    \end{itemize}
\end{frame}

%%%% SLIDE 34 %%%%
\begin{frame}
    \frametitle{Resource Reuse Vulnerabilities: Dangerous Leftovers}
    
    \begin{alertblock}{Hidden Data}
        Resources that aren't properly cleared before reuse can leak sensitive information between users or systems.
    \end{alertblock}
    
    \begin{itemize}
        \item \textbf{Resource reuse vulnerabilities} occur when computing resources aren't properly cleared before being reassigned.
        \item It's similar to renting a car and finding the previous renter's personal information still in the GPS history.
        \item In computing, memory might contain fragments of sensitive data after an application finishes using it.
        \item If this memory is assigned to another application without being cleared, the new application might access that sensitive data.
        \item Cloud computing makes resource reuse concerns more serious because your virtual servers might use memory or storage previously used by strangers.
    \end{itemize}
\end{frame}

%%%% SLIDE 35 %%%%
\begin{frame}
    \frametitle{Vulnerability Management: A Comprehensive Approach}
    
    \begin{itemize}
        \item \textbf{Vulnerability management} is the ongoing process of finding, assessing, and fixing security weaknesses.
        \item Think of it as regular health check-ups for your IT systems—identifying problems before they become serious.
        \item The process starts with knowing what you have—a complete inventory of hardware, software, and systems.
        \item Next, you need regular scanning and testing to identify vulnerabilities across your environment.
        \item Not all vulnerabilities are equally dangerous—prioritize fixes based on risk level, potential impact, and exploitation likelihood.
    \end{itemize}
\end{frame}

%%%% SLIDE 36 %%%%
\begin{frame}
    \frametitle{The Future of Vulnerabilities: Emerging Risks}
    
    \begin{block}{Proactive Security}
        Tomorrow's security requires building systems that can withstand unknown attacks, not just patching known vulnerabilities.
    \end{block}
    
    \begin{itemize}
        \item As technology evolves, new types of vulnerabilities emerge that we haven't encountered before.
        \item \textbf{Artificial intelligence systems} introduce new risks related to data poisoning and manipulation of learning algorithms.
        \item The \textbf{Internet of Things (IoT)} connects billions of devices with widely varying security standards.
        \item \textbf{Quantum computing} may eventually break many encryption systems we rely on today.
        \item The most effective security approach is building resilience—systems that can detect and contain attacks even when vulnerabilities exist.
    \end{itemize}
\end{frame}

\end{document}